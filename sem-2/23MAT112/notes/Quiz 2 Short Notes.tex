\documentclass{article}

\input{preamble1}
\title{\Huge{Quiz 2}\\ Quick Reference Notes}
\author{\huge{Adithya Nair}}
\date{}
\begin{document}
\section{Inner Products For Functions} % (fold)
\[L^P: \int f g dx = <f,g>, \text{Where P = 2}\]

Here, $L^2$ is the defined norm we are going to take, and $\langle f,g\rangle$ is the notation for the inner product.

When we take the inner product for a function on itself,

\[\int f f dx = \langle f,f\rangle dx = \|f\|^2\]

\begin{note}
   $\cos{x}$ and $\sin{x}$ are orthogonal to each other. $\cos{(kx)}$ and $\cos{(lx)}$ as well as $\sin{(kx)}$ and $\sin{(lx)}$ are also orthogonal to each other.
\end{note}



\section{Fourier Series}
\begin{definition}[Fourier Series]
   The Fourier series of a function $f(x)$ is defined on $-\pi \leq x \leq \pi$ where    \[
   	 f(x) \approxeq \frac{a_0}{2}+\sum_{k=1}^{\infty}[a_k \cos{kx} + b_k \sin{kx}]
   \]
   where the coefficients are given by the inner product formulae:
   \begin{align*}
      a_k = \langle f, \cos{kx} \rangle = \frac{1}{\pi}\int_{-\pi}^{\pi} f(x) \cos{(kx)} dx \ \forall k = 0,1,2\dots \\
      b_k = \langle f, \sin{kx} \rangle = \frac{1}{\pi}\int_{-\pi}^{\pi} f(x) \sin{(kx)} dx \ \forall k = 0,1,2\dots
   \end{align*}
\end{definition}

\section{Change Of Scale} % (fold)
For a periodic function f(x) such that the period p = 2L, where L is some natural number, we scale the Fourier Series to fit the interval of $2\pi$, to do this we set $x = \frac{p}{2\pi}v$, then $f(x) = f(\frac{p}{2\pi}v) = f(\frac{p}{2\pi}(v+2\pi))$

Then, 
\[
     a_0 = \frac{1}{L} \int_{-L}^{L} f(x) dx
\]
\[
  a_k = \frac{1}{L} \int_{-L }^{L} f(x) \cos {\frac{k \pi x}{L}} dx  
\]
\[
 b_k = \frac{1}{L} \int_{-L }^{L} f(x) \sin {\frac{k \pi x}{L}} dx  
\] \section{Odd And Even Functions} % (fold)
\begin{definition}[Even Function]
   A function such that $f(x) = f(-x)$
   For such a function,
   \[
   	b_k = 0
   \]
   and it only has $\cos$ terms
   \label{dfn2}
\end{definition} % (fold)
\begin{definition}[Odd Function]
   A function such that $f(-x) = -f(x)$
   For such a function,
   \[
   	a_0, a_k = 0
   \]
   And it only has $\sin$ terms
\end{definition}

\begin{theorem}
    The Fourier coefficients of the functions $f_1 + f_2$ are the sum of the corresponding coefficients of $f_1$ and $f_2$, The Fourier coefficients of cf is c times the Fourier coefficients of f.

\end{theorem}

\section{Differentiation And Integration Of Fourier Series}

\subsection{Half Range Expansions}
When we're finding the Fourier Series of a function for some interval [0,L], we convert it into a periodic function by limiting them to the range $[-\pi,\pi]$ and then extend the function with a period $2\pi$.
When a range is given, we assume both even and odd periodic extensions, and solve for the Fourier Series.
\section{Piecewise Continuous Functions}

\begin{definition}[Piecewise Continuous Functions]
   A function f(x) is said to be piecewise continuous on an interval [a,b] if it is defined and continuous ecxept possibly at a finite number of points $a\leq x_1 \leq x_2 \leq \dots \leq x_n \leq b$ Furthermore, at each point of discontinuity, we require that the left and right hand limits exists. 
   \[
      f(x_k^-) = \lim_{x \rightarrow x_k^-} f(x); f(x_k^+) = \lim_{x\rightarrow x_k^+} f(x)
   \]
   At the ends of the domain, the left hand is ignored at a and the right hand limit is ignored at b.
\end{definition}
We take the basis,
\[
   \sigma(x) = 1, x>0 \ and \ 0, x<0
\]
Then
\[
	h(x) = \beta \sigma(x - \xi) = \beta, x> \xi \ and \ 0, x<\xi
\]
\begin{definition}[Piecewise $C^1$]
   A function is called piecewise $C^1(U \subset R)$ (continuous and continuously differentiable) on the interval [a,b] except at a finite number of points $a\leq x_1 \leq x_2 \leq \dots \leq x_n \leq b$. At each exceptional point, the left and the right hand limits of both the function and its derivative exists.
\end{definition}
\section{Complex Fourier Series}
\[
   f(x) \approxeq \frac{a_0}{2}+\sum_{k=1}^{\infty}[a_k \cos{kx} + b_k \sin{kx}]
\]

The complex form of this is
\[
   f(x) = \sum_{k=-\infty}^{\infty} c_k e^{-ikx}
\]
Here the vector space is complex functions.
% section Complex Fourier Series (end)
The inner product is 
\[
   \langle f,g \rangle = \frac{1}{2\pi}\int_{-\pi}^{\pi} f(x) \overline{g(x)} dx
\]
And, $\cos$ and $\sin$ can be written as:
\[
\cos{(kx)} = \frac{e^{ikx} + e^{-ikx}}{2} \ , \ \ \sin{(kx)} = \frac{e^{ikx} - e^{-ikx}}{2i}
\]
\[
   f(x) = \sum_{k=-\infty}^{\infty} c_k e^{ikx}
\]
\[
   c_k = \langle f, e^{ikx} \rangle = \frac{1}{2\pi} \int_{-\pi}^{\pi} f(x) e^{-ikx}dx
\]
\[
   \langle e^{ikx},e^{ilx} \rangle = \frac{1}{2\pi} \int_{-\pi}^\pi e^{ikx} e^{-ilx}dx
\]
\section{Differentiation And Integration} % (fold)
\[
   f(x) = \frac{a_0}{2} + \sum_{k=1}^{\infty} [a_k cos(kx) + b_k sin(kx)]
\]
The discussion was surrounding whether or not we can differentiate or integrate functions that have been expressed with their Fourier series... Only when these functions converge. 
Looking at integration first, the integral of sin x is -cos x, and the cos term gets converted to sin. 

The problem arises with the constant term $\frac{a_0}{2}$, which is not periodic.
\[
   \frac{a_0}{2} = \frac{1}{2\pi} \int_{-\pi}^{\pi} f(x) dx
\]
This term is known as the mean of the function. A Fourier Series is only integrable when the mean is zero.
\begin{lemma}
   If f(x)  is $2\pi$- periodic, then its integral $g(x) = \int_0^{x} f(y) dy$ is $2\pi$-periodic if and only if $\int_{-\pi}^\pi f(x) dx = 0$, so that f(x) has zero mean on the interval $[-\pi,\pi]$.
\end{lemma}
\begin{theorem}
   If f is piecewise continuous and has mean zero on the interval $[-\pi,\pi]$ then its Fourier Series  
	\begin{displaymath}
	   f(x) \approxeq \sum_{k=1}^{\infty} [a_k cos(kx) + b_k sin(kx)]
	\end{displaymath}
	can be integrated term by term to produce the Fourier Series. 
	\[
	   g(x) = \int_0^{x} f(y) dy \approxeq m + \sum_{k=1}^{\infty} - \frac{b_k}{k} cos kx + \frac{a_k}{k} sin kx
	\]
	where $m = \int_{-\pi}^{\pi}$g(x) dx
\end{theorem}
\begin{note}
   m is the inner product of g(x) with itself. The remarkable thing about this is it is equal to $\sum_{k=1}^{\infty} \frac{b_k}{k}$
\end{note}
\section{Integration Formulas} % (fold)
\textbf{Integration By Parts}
\[
   \int u dv = uv - \int v du
\]
\label{sec:}
\textbf{Trignometric Formulas}
\begin{align*}
   \int \sin{x} dx = - \cos{x} + C \\
   \int \cos{x} dx = \sin{x} + C \\
   \int \sec^2{x} dx = \tan{x} + C \\ 
   \int \csc^2{x} dx = -\cot{x} + C \\ 
   \int \sec{x} \tan{x} dx = \sec{x} + C \\ 
   \int \csc{x} \cot{x} dx = -\csc{x} + C \\
\end{align*}
\textbf{Trigonometric Identities}
\begin{align*}
   \sin{-x} = -\sin{x} \\
   \cos{-x} = \cos{x} \\
   \tan{-x} = -\tan{x} \\ 
   \cos{2x} = \cos^2{x} - \sin^2{x} \\
   \sin{2x} = 2 \sin{x} \cos{x} \\ 
   \sin{A + B} = \sin{A}\cos{B} + \cos{A}\sin{B}\\
   \sin{A - B} = \sin{A}\cos{B} - \cos{A}\sin{B} \\
   \cos{A + B} = \cos{A}\cos{B} - \sin{A}\sin{B} \\
   \cos{A - B} = \cos{A}\cos{B} + \sin{A}\sin{B} \\
   \tan{A + B} = \frac{\tan{A} + \tan{B}}{1-\tan{A}\tan{B}} \\
   \tan{A - B} = \frac{\tan{A} - \tan{B}}{1+\tan{A}\tan{B}} \\
\end{align*} 
\end{document}
