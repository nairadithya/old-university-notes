\documentclass{report}

%%%%%%%%%%%%%%%%%%%%%%%%%%%%%%%%%%%%%%%%%%%%%%%%%%%%%%%%%%%%%%%%%%%%%%%%%%%%%%%
%                                Basic Packages                               %
%%%%%%%%%%%%%%%%%%%%%%%%%%%%%%%%%%%%%%%%%%%%%%%%%%%%%%%%%%%%%%%%%%%%%%%%%%%%%%%

% Gives us multiple colors.
\usepackage[dvipsnames,pdftex]{xcolor}
% Lets us style link colors.
\usepackage{hyperref}
% Lets us import images and graphics.
\usepackage{graphicx}
% Lets us use figures in floating environments.
\usepackage{float}
% Lets us create multiple columns.
\usepackage{multicol}
% Gives us better math syntax.
\usepackage{amsmath,amsfonts,mathtools,amsthm,amssymb}
% Lets us strikethrough text.
\usepackage{cancel}
% Lets us edit the caption of a figure.
\usepackage{caption}
% Lets us import pdf directly in our tex code.
\usepackage{pdfpages}
% Lets us do algorithm stuff.
\usepackage[ruled,vlined,linesnumbered]{algorithm2e}
% Use a smiley face for our qed symbol.
\usepackage{tikzsymbols}
\renewcommand\qedsymbol{$\Laughey$}

\def\class{article}


%%%%%%%%%%%%%%%%%%%%%%%%%%%%%%%%%%%%%%%%%%%%%%%%%%%%%%%%%%%%%%%%%%%%%%%%%%%%%%%
%                                Basic Settings                               %
%%%%%%%%%%%%%%%%%%%%%%%%%%%%%%%%%%%%%%%%%%%%%%%%%%%%%%%%%%%%%%%%%%%%%%%%%%%%%%%

%%%%%%%%%%%%%
%  Symbols  %
%%%%%%%%%%%%%

\let\implies\Rightarrow
\let\impliedby\Leftarrow
\let\iff\Leftrightarrow
\let\epsilon\varepsilon

%%%%%%%%%%%%
%  Tables  %
%%%%%%%%%%%%

\setlength{\tabcolsep}{5pt}
\renewcommand\arraystretch{1.5}

%%%%%%%%%%%%%%
%  SI Unitx  %
%%%%%%%%%%%%%%

\usepackage{siunitx}
\sisetup{locale = FR}

%%%%%%%%%%
%  TikZ  %
%%%%%%%%%%

\usepackage[framemethod=TikZ]{mdframed}
\usepackage{tikz}
\usepackage{tikz-cd}
\usepackage{tikzsymbols}

\usetikzlibrary{intersections, angles, quotes, calc, positioning}
\usetikzlibrary{arrows.meta}

\tikzset{
  force/.style={thick, {Circle[length=2pt]}-stealth, shorten <=-1pt}
}

%%%%%%%%%%%%%%%
%  PGF Plots  %
%%%%%%%%%%%%%%%

\usepackage{pgfplots}
\pgfplotsset{compat=1.13}

%%%%%%%%%%%%%%%%%%%%%%%
%  Center Title Page  %
%%%%%%%%%%%%%%%%%%%%%%%

\usepackage{titling}
\renewcommand\maketitlehooka{\null\mbox{}\vfill}
\renewcommand\maketitlehookd{\vfill\null}

%%%%%%%%%%%%%%%%%%%%%%%%%%%%%%%%%%%%%%%%%%%%%%%%%%%%%%%
%  Create a grey background in the middle of the PDF  %
%%%%%%%%%%%%%%%%%%%%%%%%%%%%%%%%%%%%%%%%%%%%%%%%%%%%%%%

\usepackage{eso-pic}
\newcommand\definegraybackground{
  \definecolor{reallylightgray}{HTML}{FAFAFA}
  \AddToShipoutPicture{
    \ifthenelse{\isodd{\thepage}}{
      \AtPageLowerLeft{
        \put(\LenToUnit{\dimexpr\paperwidth-222pt},0){
          \color{reallylightgray}\rule{222pt}{297mm}
        }
      }
    }
    {
      \AtPageLowerLeft{
        \color{reallylightgray}\rule{222pt}{297mm}
      }
    }
  }
}

%%%%%%%%%%%%%%%%%%%%%%%%
%  Modify Links Color  %
%%%%%%%%%%%%%%%%%%%%%%%%

\hypersetup{
  % Enable highlighting links.
  colorlinks,
  % Change the color of links to blue.
  linkcolor=blue,
  % Change the color of citations to black.
  citecolor={black},
  % Change the color of url's to blue with some black.
  urlcolor={blue!80!black}
}

%%%%%%%%%%%%%%%%%%
% Fix WrapFigure %
%%%%%%%%%%%%%%%%%%

\newcommand{\wrapfill}{\par\ifnum\value{WF@wrappedlines}>0
    \parskip=0pt
    \addtocounter{WF@wrappedlines}{-1}%
    \null\vspace{\arabic{WF@wrappedlines}\baselineskip}%
    \WFclear
\fi}

%%%%%%%%%%%%%%%%%
% Multi Columns %
%%%%%%%%%%%%%%%%%

\let\multicolmulticols\multicols
\let\endmulticolmulticols\endmulticols

\RenewDocumentEnvironment{multicols}{mO{}}
{%
  \ifnum#1=1
    #2%
  \else % More than 1 column
    \multicolmulticols{#1}[#2]
  \fi
}
{%
  \ifnum#1=1
\else % More than 1 column
  \endmulticolmulticols
\fi
}

\newlength{\thickarrayrulewidth}
\setlength{\thickarrayrulewidth}{5\arrayrulewidth}


%%%%%%%%%%%%%%%%%%%%%%%%%%%%%%%%%%%%%%%%%%%%%%%%%%%%%%%%%%%%%%%%%%%%%%%%%%%%%%%
%                           School Specific Commands                          %
%%%%%%%%%%%%%%%%%%%%%%%%%%%%%%%%%%%%%%%%%%%%%%%%%%%%%%%%%%%%%%%%%%%%%%%%%%%%%%%

%%%%%%%%%%%%%%%%%%%%%%%%%%%
%  Initiate New Counters  %
%%%%%%%%%%%%%%%%%%%%%%%%%%%

\newcounter{lecturecounter}

%%%%%%%%%%%%%%%%%%%%%%%%%%
%  Helpful New Commands  %
%%%%%%%%%%%%%%%%%%%%%%%%%%

\makeatletter

\newcommand\resetcounters{
  % Reset the counters for subsection, subsubsection and the definition
  % all the custom environments.
  \setcounter{subsection}{0}
  \setcounter{subsubsection}{0}
  \setcounter{paragraph}{0}
  \setcounter{subparagraph}{0}
  \setcounter{theorem}{0}
  \setcounter{claim}{0}
  \setcounter{corollary}{0}
  \setcounter{lemma}{0}
  \setcounter{exercise}{0}

  \@ifclasswith\class{nocolor}{
    \setcounter{definition}{0}
  }{}
}

%%%%%%%%%%%%%%%%%%%%%
%  Lecture Command  %
%%%%%%%%%%%%%%%%%%%%%

\usepackage{xifthen}

% EXAMPLE:
% 1. \lesson{Oct 17 2022 Mon (08:46:48)}{Lecture Title}
% 2. \lesson[4]{Oct 17 2022 Mon (08:46:48)}{Lecture Title}
% 3. \lesson{Oct 17 2022 Mon (08:46:48)}{}
% 4. \lesson[4]{Oct 17 2022 Mon (08:46:48)}{}
% Parameters:
% 1. (Optional) Lesson number.
% 2. Time and date of lecture.
% 3. Lecture Title.
\def\@lesson{}
\newcommand\lesson[3][\arabic{lecturecounter}]{
  % Add 1 to the lecture counter.
  \addtocounter{lecturecounter}{1}

  % Set the section number to the lecture counter.
  \setcounter{section}{#1}
  \renewcommand\thesubsection{#1.\arabic{subsection}}

  % Reset the counters.
  \resetcounters

  % Check if user passed the lecture title or not.
  \ifthenelse{\isempty{#3}}{
    \def\@lesson{Lecture \arabic{lecturecounter}}
  }{
    \def\@lesson{Lecture \arabic{lecturecounter}: #3}
  }

  % Display the information like the following:
  %                                                  Oct 17 2022 Mon (08:49:10)
  % ---------------------------------------------------------------------------
  % Lecture 1: Lecture Title
  \hfill\small{#2}
  \hrule
  \vspace*{-0.3cm}
  \section*{\@lesson}
  \addcontentsline{toc}{section}{\@lesson}
}

%%%%%%%%%%%%%%%%%%%%
%  Import Figures  %
%%%%%%%%%%%%%%%%%%%%

\usepackage{import}
\pdfminorversion=7

% EXAMPLE:
% 1. \incfig{limit-graph}
% 2. \incfig[0.4]{limit-graph}
% Parameters:
% 1. The figure name. It should be located in figures/NAME.tex_pdf.
% 2. (Optional) The width of the figure. Example: 0.5, 0.35.
\newcommand\incfig[2][1]{%
  \def\svgwidth{#1\columnwidth}
  \import{./figures/}{#2.pdf_tex}
}

\begingroup\expandafter\expandafter\expandafter\endgroup
\expandafter\ifx\csname pdfsuppresswarningpagegroup\endcsname\relax
\else
  \pdfsuppresswarningpagegroup=1\relax
\fi

%%%%%%%%%%%%%%%%%
% Fancy Headers %
%%%%%%%%%%%%%%%%%

\usepackage{fancyhdr}

% Force a new page.

\newcommand\forcenewpage{\clearpage\mbox{~}\clearpage\newpage}
\newcommand\createintro{
  \pagestyle{fancy}
  \fancyhead{}
  \fancyhead[C]{23MAT112}
  \fancyfoot[L]{Adithya Nair}
  \fancyfoot[R]{AID23002}
  % Create a new page.
}
  \newpage
\makeatother


%%%%%%%%%%%%%%%%%%%%%%%%%%%%%%%%%%%%%%%%%%%%%%%%%%%%%%%%%%%%%%%%%%%%%%%%%%%%%%%
%                               Custom Commands                               %
%%%%%%%%%%%%%%%%%%%%%%%%%%%%%%%%%%%%%%%%%%%%%%%%%%%%%%%%%%%%%%%%%%%%%%%%%%%%%%%

%%%%%%%%%%%%
%  Circle  %
%%%%%%%%%%%%

\newcommand*\circled[1]{\tikz[baseline=(char.base)]{
  \node[shape=circle,draw,inner sep=1pt] (char) {#1};}
}

%%%%%%%%%%%%%%%%%%%
%  Todo Commands  %
%%%%%%%%%%%%%%%%%%%

\usepackage{xargs}
\usepackage[colorinlistoftodos]{todonotes}

\makeatletter

\@ifclasswith\class{working}{
  \newcommandx\unsure[2][1=]{\todo[linecolor=red,backgroundcolor=red!25,bordercolor=red,#1]{#2}}
  \newcommandx\change[2][1=]{\todo[linecolor=blue,backgroundcolor=blue!25,bordercolor=blue,#1]{#2}}
  \newcommandx\info[2][1=]{\todo[linecolor=OliveGreen,backgroundcolor=OliveGreen!25,bordercolor=OliveGreen,#1]{#2}}
  \newcommandx\improvement[2][1=]{\todo[linecolor=Plum,backgroundcolor=Plum!25,bordercolor=Plum,#1]{#2}}

  \newcommand\listnotes{
    \newpage
    \listoftodos[Notes]
  }
}{
  \newcommandx\unsure[2][1=]{}
  \newcommandx\change[2][1=]{}
  \newcommandx\info[2][1=]{}
  \newcommandx\improvement[2][1=]{}

  \newcommand\listnotes{}
}

\makeatother

%%%%%%%%%%%%%
%  Correct  %
%%%%%%%%%%%%%

% EXAMPLE:
% 1. \correct{INCORRECT}{CORRECT}
% Parameters:
% 1. The incorrect statement.
% 2. The correct statement.
\definecolor{correct}{HTML}{009900}
\newcommand\correct[2]{{\color{red}{#1 }}\ensuremath{\to}{\color{correct}{ #2}}}


%%%%%%%%%%%%%%%%%%%%%%%%%%%%%%%%%%%%%%%%%%%%%%%%%%%%%%%%%%%%%%%%%%%%%%%%%%%%%%%
%                                 Environments                                %
%%%%%%%%%%%%%%%%%%%%%%%%%%%%%%%%%%%%%%%%%%%%%%%%%%%%%%%%%%%%%%%%%%%%%%%%%%%%%%%

\usepackage{varwidth}
\usepackage{thmtools}
\usepackage[most,many,breakable]{tcolorbox}

\tcbuselibrary{theorems,skins,hooks}
\usetikzlibrary{arrows,calc,shadows.blur}

%%%%%%%%%%%%%%%%%%%
%  Define Colors  %
%%%%%%%%%%%%%%%%%%%

\definecolor{myblue}{RGB}{45, 111, 177}
\definecolor{mygreen}{RGB}{56, 140, 70}
\definecolor{myred}{RGB}{199, 68, 64}
\definecolor{mypurple}{RGB}{197, 92, 212}

\definecolor{definition}{HTML}{228b22}
\definecolor{theorem}{HTML}{00007B}
\definecolor{example}{HTML}{2A7F7F}
\definecolor{definition}{HTML}{228b22}
\definecolor{prop}{HTML}{191971}
\definecolor{lemma}{HTML}{983b0f}
\definecolor{exercise}{HTML}{88D6D1}

\colorlet{definition}{mygreen!85!black}
\colorlet{claim}{mygreen!85!black}
\colorlet{corollary}{mypurple!85!black}
\colorlet{proof}{theorem}

%%%%%%%%%%%%%%%%%%%%%%%%%%%%%%%%%%%%%%%%%%%%%%%%%%%%%%%%%
%  Create Environments Styles Based on Given Parameter  %
%%%%%%%%%%%%%%%%%%%%%%%%%%%%%%%%%%%%%%%%%%%%%%%%%%%%%%%%%

\mdfsetup{skipabove=1em,skipbelow=0em}

%%%%%%%%%%%%%%%%%%%%%%
%  Helpful Commands  %
%%%%%%%%%%%%%%%%%%%%%%

% EXAMPLE:
% 1. \createnewtheoremstyle{thmdefinitionbox}{}{}
% 2. \createnewtheoremstyle{thmtheorembox}{}{}
% 3. \createnewtheoremstyle{thmproofbox}{qed=\qedsymbol}{
%       rightline=false, topline=false, bottomline=false
%    }
% Parameters:
% 1. Theorem name.
% 2. Any extra parameters to pass directly to declaretheoremstyle.
% 3. Any extra parameters to pass directly to mdframed.
\newcommand\createnewtheoremstyle[3]{
  \declaretheoremstyle[
  headfont=\bfseries\sffamily, bodyfont=\normalfont, #2,
  mdframed={
    #3,
  },
  ]{#1}
}

% EXAMPLE:
% 1. \createnewcoloredtheoremstyle{thmdefinitionbox}{definition}{}{}
% 2. \createnewcoloredtheoremstyle{thmexamplebox}{example}{}{
%       rightline=true, leftline=true, topline=true, bottomline=true
%     }
% 3. \createnewcoloredtheoremstyle{thmproofbox}{proof}{qed=\qedsymbol}{backgroundcolor=white}
% Parameters:
% 1. Theorem name.
% 2. Color of theorem.
% 3. Any extra parameters to pass directly to declaretheoremstyle.
% 4. Any extra parameters to pass directly to mdframed.
\newcommand\createnewcoloredtheoremstyle[4]{
  \declaretheoremstyle[
  headfont=\bfseries\sffamily\color{#2}, bodyfont=\normalfont, #3,
  mdframed={
    linewidth=2pt,
    rightline=false, leftline=true, topline=false, bottomline=false,
    linecolor=#2, backgroundcolor=#2!5, #4,
  },
  ]{#1}
}

%%%%%%%%%%%%%%%%%%%%%%%%%%%%%%%%%%%
%  Create the Environment Styles  %
%%%%%%%%%%%%%%%%%%%%%%%%%%%%%%%%%%%

\makeatletter
\@ifclasswith\class{nocolor}{
  % Environments without color.

  \createnewtheoremstyle{thmdefinitionbox}{}{}
  \createnewtheoremstyle{thmtheorembox}{}{}
  \createnewtheoremstyle{thmexamplebox}{}{}
  \createnewtheoremstyle{thmclaimbox}{}{}
  \createnewtheoremstyle{thmcorollarybox}{}{}
  \createnewtheoremstyle{thmpropbox}{}{}
  \createnewtheoremstyle{thmlemmabox}{}{}
  \createnewtheoremstyle{thmexercisebox}{}{}
  \createnewtheoremstyle{thmdefinitionbox}{}{}
  \createnewtheoremstyle{thmquestionbox}{}{}
  \createnewtheoremstyle{thmsolutionbox}{}{}

  \createnewtheoremstyle{thmproofbox}{qed=\qedsymbol}{}
  \createnewtheoremstyle{thmexplanationbox}{}{}
}{
  % Environments with color.

  \createnewcoloredtheoremstyle{thmdefinitionbox}{definition}{}{}
  \createnewcoloredtheoremstyle{thmtheorembox}{theorem}{}{}
  \createnewcoloredtheoremstyle{thmexamplebox}{example}{}{
    rightline=true, leftline=true, topline=true, bottomline=true
  }
  \createnewcoloredtheoremstyle{thmclaimbox}{claim}{}{}
  \createnewcoloredtheoremstyle{thmcorollarybox}{corollary}{}{}
  \createnewcoloredtheoremstyle{thmpropbox}{prop}{}{}
  \createnewcoloredtheoremstyle{thmlemmabox}{lemma}{}{}
  \createnewcoloredtheoremstyle{thmexercisebox}{exercise}{}{}

  \createnewcoloredtheoremstyle{thmproofbox}{proof}{qed=\qedsymbol}{backgroundcolor=white}
  \createnewcoloredtheoremstyle{thmexplanationbox}{example}{qed=\qedsymbol}{backgroundcolor=white}
}
\makeatother

%%%%%%%%%%%%%%%%%%%%%%%%%%%%%
%  Create the Environments  %
%%%%%%%%%%%%%%%%%%%%%%%%%%%%%

\declaretheorem[numberwithin=section, style=thmtheorembox,     name=Theorem]{theorem}
\declaretheorem[numbered=no,          style=thmexamplebox,     name=Example]{example}
\declaretheorem[numberwithin=section, style=thmclaimbox,       name=Claim]{claim}
\declaretheorem[numberwithin=section, style=thmcorollarybox,   name=Corollary]{corollary}
\declaretheorem[numberwithin=section, style=thmpropbox,        name=Proposition]{prop}
\declaretheorem[numberwithin=section, style=thmlemmabox,       name=Lemma]{lemma}
\declaretheorem[numberwithin=section, style=thmexercisebox,    name=Exercise]{exercise}
\declaretheorem[numbered=no,          style=thmproofbox,       name=Proof]{replacementproof}
\declaretheorem[numbered=no,          style=thmexplanationbox, name=Proof]{expl}

\makeatletter
\@ifclasswith\class{nocolor}{
  % Environments without color.

  \newtheorem*{note}{Note}

  \declaretheorem[numberwithin=section, style=thmdefinitionbox, name=Definition]{definition}
  \declaretheorem[numberwithin=section, style=thmquestionbox,   name=Question]{question}
  \declaretheorem[numberwithin=section, style=thmsolutionbox,   name=Solution]{solution}
}{
  % Environments with color.

  \newtcbtheorem[number within=section]{Definition}{Definition}{
    enhanced,
    before skip=2mm,
    after skip=2mm,
    colback=red!5,
    colframe=red!80!black,
    colbacktitle=red!75!black,
    boxrule=0.5mm,
    attach boxed title to top left={
      xshift=1cm,
      yshift*=1mm-\tcboxedtitleheight
    },
    varwidth boxed title*=-3cm,
    boxed title style={
      interior engine=empty,
      frame code={
        \path[fill=tcbcolback]
        ([yshift=-1mm,xshift=-1mm]frame.north west)
        arc[start angle=0,end angle=180,radius=1mm]
        ([yshift=-1mm,xshift=1mm]frame.north east)
        arc[start angle=180,end angle=0,radius=1mm];
        \path[left color=tcbcolback!60!black,right color=tcbcolback!60!black,
        middle color=tcbcolback!80!black]
        ([xshift=-2mm]frame.north west) -- ([xshift=2mm]frame.north east)
        [rounded corners=1mm]-- ([xshift=1mm,yshift=-1mm]frame.north east)
        -- (frame.south east) -- (frame.south west)
        -- ([xshift=-1mm,yshift=-1mm]frame.north west)
        [sharp corners]-- cycle;
      },
    },
    fonttitle=\bfseries,
    title={#2},
    #1
  }{def}

  \NewDocumentEnvironment{definition}{O{}O{}}
    {\begin{Definition}{#1}{#2}}{\end{Definition}}

  \newtcolorbox{note}[1][]{%
    enhanced jigsaw,
    colback=gray!20!white,%
    colframe=gray!80!black,
    size=small,
    boxrule=1pt,
    title=\textbf{Note:-},
    halign title=flush center,
    coltitle=black,
    breakable,
    drop shadow=black!50!white,
    attach boxed title to top left={xshift=1cm,yshift=-\tcboxedtitleheight/2,yshifttext=-\tcboxedtitleheight/2},
    minipage boxed title=1.5cm,
    boxed title style={%
      colback=white,
      size=fbox,
      boxrule=1pt,
      boxsep=2pt,
      underlay={%
        \coordinate (dotA) at ($(interior.west) + (-0.5pt,0)$);
        \coordinate (dotB) at ($(interior.east) + (0.5pt,0)$);
        \begin{scope}
          \clip (interior.north west) rectangle ([xshift=3ex]interior.east);
          \filldraw [white, blur shadow={shadow opacity=60, shadow yshift=-.75ex}, rounded corners=2pt] (interior.north west) rectangle (interior.south east);
        \end{scope}
        \begin{scope}[gray!80!black]
          \fill (dotA) circle (2pt);
          \fill (dotB) circle (2pt);
        \end{scope}
      },
    },
    #1,
  }

  \newtcbtheorem{Question}{Question}{enhanced,
    breakable,
    colback=white,
    colframe=myblue!80!black,
    attach boxed title to top left={yshift*=-\tcboxedtitleheight},
    fonttitle=\bfseries,
    title=\textbf{Question:-},
    boxed title size=title,
    boxed title style={%
      sharp corners,
      rounded corners=northwest,
      colback=tcbcolframe,
      boxrule=0pt,
    },
    underlay boxed title={%
      \path[fill=tcbcolframe] (title.south west)--(title.south east)
      to[out=0, in=180] ([xshift=5mm]title.east)--
      (title.center-|frame.east)
      [rounded corners=\kvtcb@arc] |-
      (frame.north) -| cycle;
    },
    #1
  }{def}

  \NewDocumentEnvironment{question}{O{}O{}}
  {\begin{Question}{#1}{#2}}{\end{Question}}

  \newtcolorbox{Solution}{enhanced,
    breakable,
    colback=white,
    colframe=mygreen!80!black,
    attach boxed title to top left={yshift*=-\tcboxedtitleheight},
    title=\textbf{Solution:-},
    boxed title size=title,
    boxed title style={%
      sharp corners,
      rounded corners=northwest,
      colback=tcbcolframe,
      boxrule=0pt,
    },
    underlay boxed title={%
      \path[fill=tcbcolframe] (title.south west)--(title.south east)
      to[out=0, in=180] ([xshift=5mm]title.east)--
      (title.center-|frame.east)
      [rounded corners=\kvtcb@arc] |-
      (frame.north) -| cycle;
    },
  }

  \NewDocumentEnvironment{solution}{O{}O{}}
  {\vspace{-10pt}\begin{Solution}{#1}{#2}}{\end{Solution}}
}
\makeatother

%%%%%%%%%%%%%%%%%%%%%%%%%%%%
%  Edit Proof Environment  %
%%%%%%%%%%%%%%%%%%%%%%%%%%%%

\renewenvironment{proof}[1][\proofname]{\vspace{-10pt}\begin{replacementproof}}{\end{replacementproof}}
\newenvironment{explanation}[1][\proofname]{\vspace{-10pt}\begin{expl}}{\end{expl}}

\theoremstyle{definition}

\newtheorem*{notation}{Notation}
\newtheorem*{previouslyseen}{As previously seen}
\newtheorem*{problem}{Problem}
\newtheorem*{observe}{Observe}
\newtheorem*{property}{Property}
\newtheorem*{intuition}{Intuition}

\title{Mathematics For Intelligent Systems 2 \\ Quick Reference Notes}
\author{Adithya Nair}
\createintro

\begin{document}
\maketitle
\chapter{Eigenvalues And Eigenvectors}



\textbf{\[
   A\vec{x} = \lambda \vec{x}
\]}

\begin{note}
	\begin{itemize}
	\item If $Ax = \lambda x$, then $A^2 x = \lambda^2 x$ and $A^{-1}x = \lambda^{-1}x$ and $(A+cI)x = (\lambda + c)x$ 
	\item $det(A) = \lambda_1 \lambda_2 \lambda_3 \dots \lambda_n$
	\item trace(A) = $\lambda_1 + \lambda_2 \dots \lambda_n$
	\item Projection matrices have $\lambda = 1,0$. 
	\item Reflection matrices have $\lambda = -1,1$
	\item Rotation matrices have $\lambda = \pm i$
	\end{itemize}
\end{note}
\section{Determinants}
Important Properties: 
\begin{note}
   \begin{itemize}
   	\item determinant of a $2 \times 2$ matrix is $ad - bc$
	\item determinant of a singular matrix is 0 
	\item Row exchange of a matrix gives -det(A)
	\item For $\begin{bmatrix}
	      xa + yA & xb + yB \\
	      c & d \\
	\end{bmatrix}$
	The determinant is $x(ad - bc) + y(Ad-bc)$
   \end{itemize}	
\end{note}
\section{Finding Eigenvalues And Eigenvectors For A Given Matrix}
The eigenvalues for a given matrix can be found by solving the equation,

\[
	\det(A- \lambda I) = 0
\]
The roots of this equation would be the eigenvalues of the matrix A. 

The eigenvectors can be found by substituting $\lambda$ and solving for 
\[
   (A - \lambda I)\vec{x} = 0
\]
The eigenvectors fill up the nullspace of $A - \lambda I$
\section{Similar Matrices}
\begin{definition}[Similar And Diagonalizable Matrices]
	Let A and B be two square matrices of size $n \times n$. We say that A and B are similar if there is an invertible matrix of the same size P such that: 
	\[
	   A = PBP^{-1}
	\]
	Then we can say that A is \textbf{diagonalizable} if A is similar to a diagonal matrix D
\end{definition}
where, 
P's columns consists of the eigenvectors, and B consists of the eigenvalues of A.
\begin{note}
	\begin{itemize}
	\item If $A = PBP^{-1}$, then $A^n = PB^nP^{-1}$
	\item For symmetric matrices, $A = PBP^T$, since the eigenvectors of a symmetric matrix are orthogonal, and the eigenvalues are real.
	\end{itemize}
\end{note}
\section{Positive Definite Matrix}
\begin{definition}[Positive Definite Matrices]
Any matrix A for which the following is true:	
	\begin{displaymath}
   \vec{x}^T A\vec{x} > 0, \forall \vec{x} \neq \vec{0}
	\end{displaymath}
is called a \textbf{positive definite matrix}
\end{definition}
\pagebreak
For such a matrix: 
\begin{itemize}
	\item The matrix is always a full-rank matrix. 
	\item The eigenvalues are all positive
	\item $A+B$ is also a positive definite matrix, when A and B are positive definite. 
	\item For a rectangular $m \times n$ matrix, we check for $A^TA \text{ or } AA^T$, \[
	      \vec{x}^T(A^TA)\vec{x} = (\vec{x}^TA^T)(A\vec{x}) = (A\vec{x})^T(A\vec{x}) = \|{A\vec{x}}\|^2
	\]
	This means that $A^TA$ or $AA^T$ is always a positive definite matrix, unless x lies in the nullspace of A, in which case $Ax = 0$(in other words, $A^T A$ is always positive semi-definite)
\end{itemize}
\section{Singular Value Decomposition}
\[
	A = U \Sigma V^T
\]
Another way to write this is,
\[
   A = \sum_{i=1}^{n} u_i\sigma_i v^T_i
\]
\subsection{The Procedure For Finding This Decomposition}
\begin{enumerate}
   \item Find the eigenvectors for $A^TA$, this forms the column vectors for V. 
   \item Normalize these column vectors of V.
   \item The corresponding eigenvalues of V are the diagonal entries of $\Sigma^2$
   \item Evaluate $AV = U \Sigma$
\end{enumerate}
\section{Important Theorems And Lemmas}
\begin{lemma}
   Suppose that A and B are two $n \times n$ matrices and P is an invertible matrix, such that $A = PBP^{-1}.$ Then, $A^{n} = PB^{n}P^{-1}$ 
\end{lemma}
\begin{proof}
	Using the principle of mathematical induction.\\
	We are given, $A = PBP^{-1}$
	to show, $A^{n} = PB^{n}P^{-1}$

	\textbf{Base step - } $n = 1, A^{1} = P^1 B^1 P^{-1}$, which is true 

	\textbf{Induction step - } Supppose $A^{n} = PB^{n}P^{-1}$. We need to show that $A^{n+1} = PB^{n+1}P^{-1}$
	\[
	   A^{n+1} = A \dot A^{n}
	\]
	\[
	   = (PBP^{-1})(PB^n P^{-1})
	\]
	\[
	= PBB^nP^{-1} = PB^{n+1}P^{-1} \]
\end{proof}
\begin{theorem}
   Let A be an $n \times n$ matrix and let $v_1,v_2,\dots,v_k$ be eigenvectors of A with distinct eigenvalues $\lambda_1, \lambda_2,\dots\lambda_k$. Then $v_1,v_2\dots v_k$ are independent. In particular, if k = n, then $v_1,v_2\dots v_k$ are a basis of eigenvectors for $\mathbb{R}^n$
\end{theorem}
\begin{proof}
	Suppose $v_1,v_2\dots v_n$ are dependent such that $\exists r_i$  such that,
	\begin{equation}
	    \sum_{i=1}^{k} r_i v_i = 0
		\label{eq:1}
	\end{equation}
	Assume that k is minimal with this property and $r_k$ is all non-zero,
	\[
		A.0 = r_1 A v_1 + r_2 A v_2 \dots + r_k A v_k
	\]
	\begin{equation}
	0 = r_1\lambda_1 v_1 + r_2 \lambda_2 v_2 \dots r_k \lambda_k v_k = 0	
		\label{eq:2}
	\end{equation}
	\begin{equation}
	   \lambda_k \times \ref{eq:1}: r_1 \lambda_k v_1 + \dots + r_k \lambda_k v_k = 0	
		\label{eq:3}
	\end{equation}
	\ref{eq:3} - \ref{eq:2}
	\[
	   r_1(\lambda_k - \lambda_1)v_1 + \dots + r_{k-1}(\lambda_k - \lambda_{k-1})v_{k-1} = 0
	\]
	$\lambda_k - \lambda_i$ is non-zero(distinct eigenvalues)\\
	This forms a linear combination of vectors $v_1,v_2\dots v_{k-1}$ that equate to zero. \\
	This contradicts the assumption that $v_1,v_2\dots v_{k-1}$ is the minimal set which is dependent. \\
	Therefore the eigenvectors are independent.
     \end{proof}
\begin{theorem}
   Let A be a n $\times$ n matrix, Then A is diagonalizable if and only if we can find a basis $v_{1},\cdots, v_{n}$ of eigen vectors for $R^{n}$. In this case,
   \begin{equation}
      A = PDP^{1}	
   \end{equation}
   where P is the matrix whose eigenvectors $v_{1},\cdots, v_{n}$ and D is the diagonal matrix whose diagonal entries are the corresponding eigenvalues $\lambda_{1}, \cdots, \lambda_{n}$
\end{theorem}
\begin{proof}
   \[
   Av_{i} = (PDP^{-1}v_{i})
   \]
where P is the matrix with column vectors $v_{1},v_{2} \cdots v_{n}$ and D is a diagonal matrix with the entries $\lambda_{1},\cdots, \lambda_{n}$
\[
Av_i = (PDP^{-1})P \hat{e}_{i} \\
\]
\[
   Av_{i} = PD\hat{e}_i
\]
\[
Av_{i} = P\lambda_{i} \hat{e}_{i} = \lambda_{i} P \hat{e}_{i} = \lambda_i v_{i}
\]
This proves that $v_{i}$ is the eigenvector of A, and that $\lambda_{i}$ is the corresponding eigenvalue.
Because $P^{-1}$ exists, $v_{1},\cdots,v_{n}$ are independent, based on the theorem proven earlier, this is the basis for $R^{n}$ \\
\textbf{Part b}
Suppose $v_{1},\cdots, v_{n}$ are an eigenvector basis with corresponding eigenvalues $\lambda_{1},\cdots,\lambda_{n}$
Suppose that P is the matrix with column vectors $v_{1},\cdots,v_{n}$ \[
   \text{Let} D = P^{-1}AP 
\]
\[
   D \hat{e}_{i} = (P^{-1}AP)\hat{e}_i
\]
\[
   = P^{-1} A v_{i} = \lambda_{i} P^{-1}v_{i} = \lambda_{i} \hat{e}_{i}
\]
Thus D is the diagonal matrix with the diagonal entries $\lambda_{i}$
\end{proof}
%
\begin{theorem}
   The eigenvalues for a symmetric matrix are real.
\end{theorem}
\begin{proof}
   The claim:
	\[
	For \\	Ax = \lambda x, A = A^T \\
	\]
	\[
		\lambda \in R
	\]
	Take:
	\[
		A x = \lambda x
	\]
	And the complex conjugate:
	\[
	   \overline{A}\overline{x} = \overline{\lambda} \overline{x} \indent  (\overline{A} = A)(real \\ matrix)
	\]
\begin{equation}
	   (\overline{A}\overline{x})^T = (\overline{\lambda} \overline{x})^T \iff \overline{x}^T A^T = \overline{\lambda} \overline{x}^T
	\label{eq:Conjugate}
\end{equation}	
\ref{eq:Conjugate} $\times$ x
\[
   \overline{x}^T A^T x = \lambda \overline{x^T} x \iff \overline{x}^T A x = \overline{\lambda} \overline{x}^T x \iff \overline{x}^T \lambda x = \overline{\lambda} \overline{x}^T x \iff \lambda = \overline{\lambda}
\]
Then $\lambda$ is real, when $\overline{x}^T x \neq 0$
\end{proof}
\begin{lemma}
   Let A be a symmetric matrix. If v and w are eigenvectors with distinct eigenvalues $\lambda \& \mu$ then v \& w are orthogonal.
\end{lemma}

\begin{proof}
	\[
		Av .  w = (Av)^T w = v^T A^Tw = v^T Aw = v . Aw
	\]
	\[
	\iff \lambda v . w = \mu v . w \iff (\lambda - \mu) v.w = 0
	\]
	\[
		\iff v . w  = 0
	\]
\end{proof}
% section Symmetric Matrices (end)
\begin{theorem}
   Let A be a symmetric matrix. Then we can find a diagonal matrix D and an orthogonal matrix P such that,
\[
   A = QDQ^T
\]	
In particular, every symmetric matrix is diagonalisable.
\end{theorem}
\begin{lemma}
	Similar matrices have the same eigenvalues
\end{lemma}
\begin{proof}[Proof Other Information]
   \[
   A\vec{x} - \lambda{x}	
   \]
   \[
      AI\vec{x} = APP^{-1}x = \lambda \vec{x}
   \]
   \[
      \implies P^{-1} A P P^{-1} \vec{x} = \lambda P^{-1} x
   \]
   \[
   	B(P^{-1}x) = \lambda(P^{-1}\vec{x})
   \]
   
\end{proof}

\section{Inner Products For Functions} % (fold)
\[L^P: \int f g dx = <f,g>, \text{Where P = 2}\]

Here, $L^2$ is the defined norm we are going to take, and $\langle f,g\rangle$ is the notation for the inner product.

When we take the inner product for a function on itself,

\[\int f f dx = \langle f,f\rangle dx = \|f\|^2\]

\begin{note}
   $\cos{x}$ and $\sin{x}$ are orthogonal to each other. $\cos{(kx)}$ and $\cos{(lx)}$ as well as $\sin{(kx)}$ and $\sin{(lx)}$ are also orthogonal to each other.
\end{note}



\section{Fourier Series}
\begin{definition}[Fourier Series]
   The Fourier series of a function $f(x)$ is defined on $-\pi \leq x \leq \pi$ where    \[
   	 f(x) \approxeq \frac{a_0}{2}+\sum_{k=1}^{\infty}[a_k \cos{kx} + b_k \sin{kx}]
   \]
   where the coefficients are given by the inner product formulae:
   \begin{align*}
      a_k = \langle f, \cos{kx} \rangle = \frac{1}{\pi}\int_{-\pi}^{\pi} f(x) \cos{(kx)} dx \ \forall k = 0,1,2\dots \\
      b_k = \langle f, \sin{kx} \rangle = \frac{1}{\pi}\int_{-\pi}^{\pi} f(x) \sin{(kx)} dx \ \forall k = 0,1,2\dots
   \end{align*}
\end{definition}

\section{Change Of Scale} % (fold)
For a periodic function f(x) such that the period p = 2L, where L is some natural number, we scale the Fourier Series to fit the interval of $2\pi$, to do this we set $x = \frac{p}{2\pi}v$, then $f(x) = f(\frac{p}{2\pi}v) = f(\frac{p}{2\pi}(v+2\pi))$

Then, 
\[
     a_0 = \frac{1}{L} \int_{-L}^{L} f(x) dx
\]
\[
  a_k = \frac{1}{L} \int_{-L }^{L} f(x) \cos {\frac{k \pi x}{L}} dx  
\]
\[
 b_k = \frac{1}{L} \int_{-L }^{L} f(x) \sin {\frac{k \pi x}{L}} dx  
\] \section{Odd And Even Functions} % (fold)
\begin{definition}[Even Function]
   A function such that $f(x) = f(-x)$
   For such a function,
   \[
   	b_k = 0
   \]
   and it only has $\cos$ terms
   \label{dfn2}
\end{definition} % (fold)
\begin{definition}[Odd Function]
   A function such that $f(-x) = -f(x)$
   For such a function,
   \[
   	a_0, a_k = 0
   \]
   And it only has $\sin$ terms
\end{definition}

\begin{theorem}
    The Fourier coefficients of the functions $f_1 + f_2$ are the sum of the corresponding coefficients of $f_1$ and $f_2$, The Fourier coefficients of cf is c times the Fourier coefficients of f.

\end{theorem}

\section{Differentiation And Integration Of Fourier Series}

\subsection{Half Range Expansions}
When we're finding the Fourier Series of a function for some interval [0,L], we convert it into a periodic function by limiting them to the range $[-\pi,\pi]$ and then extend the function with a period $2\pi$.
When a range is given, we assume both even and odd periodic extensions, and solve for the Fourier Series. 

To do this, we take the even periodic extension, where $f(x) = f(-x)$, and similary for odd periodic extensions, where $f(-x) = -f(x)$ 
\section{Piecewise Continuous Functions}

\begin{definition}[Piecewise Continuous Functions]
   A function f(x) is said to be piecewise continuous on an interval [a,b] if it is defined and continuous ecxept possibly at a finite number of points $a\leq x_1 \leq x_2 \leq \dots \leq x_n \leq b$ Furthermore, at each point of discontinuity, we require that the left and right hand limits exists. 
   \[
      f(x_k^-) = \lim_{x \rightarrow x_k^-} f(x); f(x_k^+) = \lim_{x\rightarrow x_k^+} f(x)
   \]
   At the ends of the domain, the left hand is ignored at a and the right hand limit is ignored at b.
\end{definition}
We take the basis,
\[
   \sigma(x) = 1, x>0 \ and \ 0, x<0
\]
Then
\[
	h(x) = \beta \sigma(x - \xi) = \beta, x> \xi \ and \ 0, x<\xi
\]
\begin{definition}[Piecewise $C^1$]
   A function is called piecewise $C^1(U \subset R)$ (continuous and continuously differentiable) on the interval [a,b] except at a finite number of points $a\leq x_1 \leq x_2 \leq \dots \leq x_n \leq b$. At each exceptional point, the left and the right hand limits of both the function and its derivative exists.
\end{definition}
\section{Complex Fourier Series}
\[
   f(x) \approxeq \frac{a_0}{2}+\sum_{k=1}^{\infty}[a_k \cos{kx} + b_k \sin{kx}]
\]

The complex form of this is
\[
   f(x) = \sum_{k=-\infty}^{\infty} c_k e^{-ikx}
\]
Here the vector space is complex functions.
% section Complex Fourier Series (end)
The inner product is 
\[
   \langle f,g \rangle = \frac{1}{2\pi}\int_{-\pi}^{\pi} f(x) \overline{g(x)} dx
\]
And, $\cos$ and $\sin$ can be written as:
\[
\cos{(kx)} = \frac{e^{ikx} + e^{-ikx}}{2} \ , \ \ \sin{(kx)} = \frac{e^{ikx} - e^{-ikx}}{2i}
\]
\[
   f(x) = \sum_{k=-\infty}^{\infty} c_k e^{ikx}
\]
\[
   c_k = \langle f, e^{ikx} \rangle = \frac{1}{2\pi} \int_{-\pi}^{\pi} f(x) e^{-ikx}dx
\]
\[
   \langle e^{ikx},e^{ilx} \rangle = \frac{1}{2\pi} \int_{-\pi}^\pi e^{ikx} e^{-ilx}dx
\]
\section{Differentiation And Integration} % (fold)
\[
   f(x) = \frac{a_0}{2} + \sum_{k=1}^{\infty} [a_k cos(kx) + b_k sin(kx)]
\]
The discussion was surrounding whether or not we can differentiate or integrate functions that have been expressed with their Fourier series... Only when these functions converge. 
Looking at integration first, the integral of sin x is -cos x, and the cos term gets converted to sin. 

The problem arises with the constant term $\frac{a_0}{2}$, which is not periodic.
\[
   \frac{a_0}{2} = \frac{1}{2\pi} \int_{-\pi}^{\pi} f(x) dx
\]
This term is known as the mean of the function. A Fourier Series is only integrable when the mean is zero.
\begin{lemma}
   If $f(x)$  is $2\pi$- periodic, then its integral $g(x) = \int_0^{x} f(y) dy$ is $2\pi$-periodic if and only if $\int_{-\pi}^\pi f(x) dx = 0$, so that f(x) has zero mean on the interval $[-\pi,\pi]$.
\end{lemma}
\begin{theorem}
   If f is piecewise continuous and has mean zero on the interval $[-\pi,\pi]$ then its Fourier Series  
	\begin{displaymath}
	   f(x) \approxeq \sum_{k=1}^{\infty} [a_k cos(kx) + b_k sin(kx)]
	\end{displaymath}
	can be integrated term by term to produce the Fourier Series. 
	\[
	   g(x) = \int_0^{x} f(y) dy \approxeq m + \sum_{k=1}^{\infty} - \frac{b_k}{k} cos kx + \frac{a_k}{k} sin kx
	\]
	where $m = \int_{-\pi}^{\pi}$g(x) dx
\end{theorem}
\begin{note}
   m is the inner product of g(x) with itself. The remarkable thing about this is it is equal to $\sum_{k=1}^{\infty} \frac{b_k}{k}$
\end{note}
\section{Integration Formulas} % (fold)
\textbf{Integration By Parts}
\[
   \int u dv = uv - \int v du
\]
\label{sec:}
\textbf{Trignometric Formulas}
\begin{align*}
   \int \sin{x} dx & = - \cos{x} + C \\
   \int \cos{x} dx & = \sin{x} + C \\
   \int \sec^2{x} dx & = \tan{x} + C \\ 
   \int \csc^2{x}  dx & = -\cot{x} + C \\ 
   \int \sec{x} \tan{x} dx & = \sec{x} + C \\ 
   \int \csc{x} \cot{x} dx & = -\csc{x} + C \\
\end{align*}
\textbf{Trigonometric Identities}
\begin{align*}
   \sin{-x} & = -\sin{x} \\
   \cos{-x} &= \cos{x} \\
   \tan{-x} &= -\tan{x} \\ 
   \cos{2x} &= \cos^2{x} - \sin^2{x} \\
   \sin{2x} &= 2 \sin{x} \cos{x} \\ 
   \sin{A + B} & = \sin{A}\cos{B} + \cos{A}\sin{B}\\
   \sin{A - B} &= \sin{A}\cos{B} - \cos{A}\sin{B} \\
   \cos{A + B}& = \cos{A}\cos{B} - \sin{A}\sin{B} \\
   \cos{A - B} &= \cos{A}\cos{B} + \sin{A}\sin{B} \\
   \tan{A + B} &= \frac{\tan{A} + \tan{B}}{1-\tan{A}\tan{B}} \\
   \tan{A - B} &= \frac{\tan{A} - \tan{B}}{1+\tan{A}\tan{B}} \\
\end{align*} 
\section{Fourier Transform}
\begin{definition}[Fourier Transform]
   The Fourier Transform of $f(x)$ $f:\mathbb{R} \rightarrow \mathbb{C}$ is denoted by $\hat{f}(k):\mathbb{R} \rightarrow \mathbb{C}$ and is defined as:
   \[
      \hat{f}(k) = \frac{1}{\sqrt{2\pi}}\int_{-\infty}^{\infty} f(x) e^{-ikx} dx
   \]
\end{definition}
   When does $\hat{f}(k)$ exist (for all $k \in \mathbb{R}$) 
   \begin{enumerate}
   	\item f must be piecewise $C^1$
	\item $f(x) \rightarrow 0$ as $|x| \rightarrow \infty$, f must be absolutely convergent. i.e. \[\int_{-\infty}^{\infty}|f(x)| dx < \infty\]
   \end{enumerate}
   \textbf{The Notation} - 
   \begin{align*}
      &\hat{f}(k) = \mathcal{F}[f(x)] & \text{\quad Fourier Transform} \\
      & f(x) = \mathcal{F}^{-1}[\hat{f}(k)] & \text{\quad Inverse Fourier Transform}
   \end{align*}
   \subsection{Properties}
   \begin{align*}
	& \mathcal{F}[f(x) + g(x)] = \mathcal{F}[f(x)] + \mathcal{F}[g(x)] = \hat{f}(k) + \hat{g}(k) & \text{\quad Linear Operator}  \\
	& \mathcal{F}[cf(x)] = c\mathcal{F}[f(x)] = c \hat{f}(k) & \text{Scalar Multiplication}
   \end{align*}
   \begin{definition}[Convolution]
   The convolutoin of scalar functions $f(x)$ and $g(x)$ is the scalar function $h = f*g$ is defined by the formula	
   \[
      h(x) = f*g(x) = \int_{-\infty}^{\infty} f(x-\xi)g(\xi) d\xi
   \]
   \end{definition}
   \begin{theorem}
      The Fourier Transform of the convolution $h(x) = f*g(x)$ of two functions is a multiple of the product of their Fourier Trasnforms.
      \[
	 \hat{h}(k) = \sqrt{2\pi} \hat{f}(k) \hat{g}(k)
      \]
   	
   \end{theorem}
\end{document}
