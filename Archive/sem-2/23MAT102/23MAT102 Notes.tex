\documentclass{report}

%%%%%%%%%%%%%%%%%%%%%%%%%%%%%%%%%%%%%%%%%%%%%%%%%%%%%%%%%%%%%%%%%%%%%%%%%%%%%%%
%                                Basic Packages                               %
%%%%%%%%%%%%%%%%%%%%%%%%%%%%%%%%%%%%%%%%%%%%%%%%%%%%%%%%%%%%%%%%%%%%%%%%%%%%%%%

% Gives us multiple colors.
\usepackage[dvipsnames,pdftex]{xcolor}
\usepackage[tmargin=2cm,rmargin=1in,lmargin=1in,margin=0.85in,bmargin=2cm,footskip=.2in]{geometry}

% Lets us style link colors.
\usepackage{hyperref}
% Lets us import images and graphics.
\usepackage{graphicx}
% Lets us use figures in floating environments.
\usepackage{float}
% Lets us create multiple columns.
\usepackage{multicol}
% Gives us better math syntax.
\usepackage{amsmath,amsfonts,mathtools,amsthm,amssymb}
% Lets us strikethrough text.
\usepackage{cancel}
% Lets us edit the caption of a figure.
\usepackage{caption}
% Lets us import pdf directly in our tex code.
\usepackage{pdfpages}
% Lets us do algorithm stuff.
\usepackage[ruled,vlined,linesnumbered]{algorithm2e}
% Use a smiley face for our qed symbol.
\usepackage{tikzsymbols}
\renewcommand\qedsymbol{$\Laughey$}

\def\class{article}


%%%%%%%%%%%%%%%%%%%%%%%%%%%%%%%%%%%%%%%%%%%%%%%%%%%%%%%%%%%%%%%%%%%%%%%%%%%%%%%
%                                Basic Settings                               %
%%%%%%%%%%%%%%%%%%%%%%%%%%%%%%%%%%%%%%%%%%%%%%%%%%%%%%%%%%%%%%%%%%%%%%%%%%%%%%%

%%%%%%%%%%%%%
%  Symbols  %
%%%%%%%%%%%%%

\let\implies\Rightarrow
\let\impliedby\Leftarrow
\let\iff\Leftrightarrow
\let\epsilon\varepsilon

%%%%%%%%%%%%
%  Tables  %
%%%%%%%%%%%%

\setlength{\tabcolsep}{5pt}
\renewcommand\arraystretch{1.5}

%%%%%%%%%%%%%%
%  SI Unitx  %
%%%%%%%%%%%%%%

\usepackage{siunitx}
\sisetup{locale = FR}

%%%%%%%%%%
%  TikZ  %
%%%%%%%%%%

\usepackage[framemethod=TikZ]{mdframed}
\usepackage{tikz}
\usepackage{tikz-cd}
\usepackage{tikzsymbols}

\usetikzlibrary{intersections, angles, quotes, calc, positioning}
\usetikzlibrary{arrows.meta}

\tikzset{
  force/.style={thick, {Circle[length=2pt]}-stealth, shorten <=-1pt}
}

%%%%%%%%%%%%%%%
%  PGF Plots  %
%%%%%%%%%%%%%%%

\usepackage{pgfplots}
\pgfplotsset{compat=1.13}

%%%%%%%%%%%%%%%%%%%%%%%
%  Center Title Page  %
%%%%%%%%%%%%%%%%%%%%%%%

\usepackage{titling}
\renewcommand\maketitlehooka{\null\mbox{}\vfill}
\renewcommand\maketitlehookd{\vfill\null}

%%%%%%%%%%%%%%%%%%%%%%%%%%%%%%%%%%%%%%%%%%%%%%%%%%%%%%%
%  Create a grey background in the middle of the PDF  %
%%%%%%%%%%%%%%%%%%%%%%%%%%%%%%%%%%%%%%%%%%%%%%%%%%%%%%%

\usepackage{eso-pic}
\newcommand\definegraybackground{
  \definecolor{reallylightgray}{HTML}{FAFAFA}
  \AddToShipoutPicture{
    \ifthenelse{\isodd{\thepage}}{
      \AtPageLowerLeft{
        \put(\LenToUnit{\dimexpr\paperwidth-222pt},0){
          \color{reallylightgray}\rule{222pt}{297mm}
        }
      }
    }
    {
      \AtPageLowerLeft{
        \color{reallylightgray}\rule{222pt}{297mm}
      }
    }
  }
}

%%%%%%%%%%%%%%%%%%%%%%%%
%  Modify Links Color  %
%%%%%%%%%%%%%%%%%%%%%%%%

\hypersetup{
  % Enable highlighting links.
  colorlinks,
  % Change the color of links to blue.
  linkcolor=blue,
  % Change the color of citations to black.
  citecolor={black},
  % Change the color of url's to blue with some black.
  urlcolor={blue!80!black}
}

%%%%%%%%%%%%%%%%%%
% Fix WrapFigure %
%%%%%%%%%%%%%%%%%%

\newcommand{\wrapfill}{\par\ifnum\value{WF@wrappedlines}>0
    \parskip=0pt
    \addtocounter{WF@wrappedlines}{-1}%
    \null\vspace{\arabic{WF@wrappedlines}\baselineskip}%
    \WFclear
\fi}

%%%%%%%%%%%%%%%%%
% Multi Columns %
%%%%%%%%%%%%%%%%%

\let\multicolmulticols\multicols
\let\endmulticolmulticols\endmulticols

\RenewDocumentEnvironment{multicols}{mO{}}
{%
  \ifnum#1=1
    #2%
  \else % More than 1 column
    \multicolmulticols{#1}[#2]
  \fi
}
{%
  \ifnum#1=1
\else % More than 1 column
  \endmulticolmulticols
\fi
}

\newlength{\thickarrayrulewidth}
\setlength{\thickarrayrulewidth}{5\arrayrulewidth}


%%%%%%%%%%%%%%%%%%%%%%%%%%%%%%%%%%%%%%%%%%%%%%%%%%%%%%%%%%%%%%%%%%%%%%%%%%%%%%%
%                           School Specific Commands                          %
%%%%%%%%%%%%%%%%%%%%%%%%%%%%%%%%%%%%%%%%%%%%%%%%%%%%%%%%%%%%%%%%%%%%%%%%%%%%%%%

%%%%%%%%%%%%%%%%%%%%%%%%%%%
%  Initiate New Counters  %
%%%%%%%%%%%%%%%%%%%%%%%%%%%

\newcounter{lecturecounter}

%%%%%%%%%%%%%%%%%%%%%%%%%%
%  Helpful New Commands  %
%%%%%%%%%%%%%%%%%%%%%%%%%%

\makeatletter

\newcommand\resetcounters{
  % Reset the counters for subsection, subsubsection and the definition
  % all the custom environments.
  \setcounter{subsection}{0}
  \setcounter{subsubsection}{0}
  \setcounter{paragraph}{0}
  \setcounter{subparagraph}{0}
  \setcounter{theorem}{0}
  \setcounter{claim}{0}
  \setcounter{corollary}{0}
  \setcounter{lemma}{0}
  \setcounter{exercise}{0}

  \@ifclasswith\class{nocolor}{
    \setcounter{definition}{0}
  }{}
}

%%%%%%%%%%%%%%%%%%%%%
%  Lecture Command  %
%%%%%%%%%%%%%%%%%%%%%

\usepackage{xifthen}

% EXAMPLE:
% 1. \lesson{Oct 17 2022 Mon (08:46:48)}{Lecture Title}
% 2. \lesson[4]{Oct 17 2022 Mon (08:46:48)}{Lecture Title}
% 3. \lesson{Oct 17 2022 Mon (08:46:48)}{}
% 4. \lesson[4]{Oct 17 2022 Mon (08:46:48)}{}
% Parameters:
% 1. (Optional) Lesson number.
% 2. Time and date of lecture.
% 3. Lecture Title.
\def\@lesson{}
\newcommand\lesson[3][\arabic{lecturecounter}]{
  % Add 1 to the lecture counter.
  \addtocounter{lecturecounter}{1}

  % Set the section number to the lecture counter.
  \setcounter{section}{#1}
  \renewcommand\thesubsection{#1.\arabic{subsection}}

  % Reset the counters.
  \resetcounters

  % Check if user passed the lecture title or not.
  \ifthenelse{\isempty{#3}}{
    \def\@lesson{Lecture \arabic{lecturecounter}}
  }{
    \def\@lesson{Lecture \arabic{lecturecounter}: #3}
  }

  % Display the information like the following:
  %                                                  Oct 17 2022 Mon (08:49:10)
  % ---------------------------------------------------------------------------
  % Lecture 1: Lecture Title
  \hfill\small{#2}
  \hrule
  \vspace*{-0.3cm}
  \section*{\@lesson}
  \addcontentsline{toc}{section}{\@lesson}
}

%%%%%%%%%%%%%%%%%%%%
%  Import Figures  %
%%%%%%%%%%%%%%%%%%%%

\usepackage{import}
\pdfminorversion=7

% EXAMPLE:
% 1. \incfig{limit-graph}
% 2. \incfig[0.4]{limit-graph}
% Parameters:
% 1. The figure name. It should be located in figures/NAME.tex_pdf.
% 2. (Optional) The width of the figure. Example: 0.5, 0.35.
\newcommand\incfig[2][1]{%
  \def\svgwidth{#1\columnwidth}
  \import{./figures/}{#2.pdf_tex}
}

\begingroup\expandafter\expandafter\expandafter\endgroup
\expandafter\ifx\csname pdfsuppresswarningpagegroup\endcsname\relax
\else
  \pdfsuppresswarningpagegroup=1\relax
\fi

%%%%%%%%%%%%%%%%%
% Fancy Headers %
%%%%%%%%%%%%%%%%%

\usepackage{fancyhdr}

% Force a new page.

\newcommand\forcenewpage{\clearpage\mbox{~}\clearpage\newpage}
\newcommand\createintro{
  \pagestyle{fancy}
  \fancyhead{}
  \fancyhead[C]{23PHY114}
  \fancyfoot[L]{Adithya Nair}
  \fancyfoot[R]{AID23002}
  % Create a new page.
}
  \newpage
\makeatother

%%%%%%%%%%%%%%%%%%%%%%%%%%%%%%%%%%%%%%%%%%%%%%%%%%%%%%%%%%%%%%%%%%%%%%%%%%%%%%%
%                               Custom Commands                               %
%%%%%%%%%%%%%%%%%%%%%%%%%%%%%%%%%%%%%%%%%%%%%%%%%%%%%%%%%%%%%%%%%%%%%%%%%%%%%%%

%%%%%%%%%%%%
%  Circle  %
%%%%%%%%%%%%

\newcommand*\circled[1]{\tikz[baseline=(char.base)]{
  \node[shape=circle,draw,inner sep=1pt] (char) {#1};}
}

%%%%%%%%%%%%%%%%%%%
%  Todo Commands  %
%%%%%%%%%%%%%%%%%%%

\usepackage{xargs}
\usepackage[colorinlistoftodos]{todonotes}

\makeatletter

\@ifclasswith\class{working}{
  \newcommandx\unsure[2][1=]{\todo[linecolor=red,backgroundcolor=red!25,bordercolor=red,#1]{#2}}
  \newcommandx\change[2][1=]{\todo[linecolor=blue,backgroundcolor=blue!25,bordercolor=blue,#1]{#2}}
  \newcommandx\info[2][1=]{\todo[linecolor=OliveGreen,backgroundcolor=OliveGreen!25,bordercolor=OliveGreen,#1]{#2}}
  \newcommandx\improvement[2][1=]{\todo[linecolor=Plum,backgroundcolor=Plum!25,bordercolor=Plum,#1]{#2}}

  \newcommand\listnotes{
    \newpage
    \listoftodos[Notes]
  }
}{
  \newcommandx\unsure[2][1=]{}
  \newcommandx\change[2][1=]{}
  \newcommandx\info[2][1=]{}
  \newcommandx\improvement[2][1=]{}

  \newcommand\listnotes{}
}

\makeatother

%%%%%%%%%%%%%
%  Correct  %
%%%%%%%%%%%%%

% EXAMPLE:
% 1. \correct{INCORRECT}{CORRECT}
% Parameters:
% 1. The incorrect statement.
% 2. The correct statement.
\definecolor{correct}{HTML}{009900}
\newcommand\correct[2]{{\color{red}{#1 }}\ensuremath{\to}{\color{correct}{ #2}}}


%%%%%%%%%%%%%%%%%%%%%%%%%%%%%%%%%%%%%%%%%%%%%%%%%%%%%%%%%%%%%%%%%%%%%%%%%%%%%%%
%                                 Environments                                %
%%%%%%%%%%%%%%%%%%%%%%%%%%%%%%%%%%%%%%%%%%%%%%%%%%%%%%%%%%%%%%%%%%%%%%%%%%%%%%%

\usepackage{varwidth}
\usepackage{thmtools}
\usepackage[most,many,breakable]{tcolorbox}

\tcbuselibrary{theorems,skins,hooks}
\usetikzlibrary{arrows,calc,shadows.blur}

%%%%%%%%%%%%%%%%%%%
%  Define Colors  %
%%%%%%%%%%%%%%%%%%%

\definecolor{myblue}{RGB}{45, 111, 177}
\definecolor{mygreen}{RGB}{56, 140, 70}
\definecolor{myred}{RGB}{199, 68, 64}
\definecolor{mypurple}{RGB}{197, 92, 212}

\definecolor{definition}{HTML}{228b22}
\definecolor{theorem}{HTML}{00007B}
\definecolor{example}{HTML}{2A7F7F}
\definecolor{definition}{HTML}{228b22}
\definecolor{prop}{HTML}{191971}
\definecolor{lemma}{HTML}{983b0f}
\definecolor{exercise}{HTML}{88D6D1}

\colorlet{definition}{mygreen!85!black}
\colorlet{claim}{mygreen!85!black}
\colorlet{corollary}{mypurple!85!black}
\colorlet{proof}{theorem}

%%%%%%%%%%%%%%%%%%%%%%%%%%%%%%%%%%%%%%%%%%%%%%%%%%%%%%%%%
%  Create Environments Styles Based on Given Parameter  %
%%%%%%%%%%%%%%%%%%%%%%%%%%%%%%%%%%%%%%%%%%%%%%%%%%%%%%%%%

\mdfsetup{skipabove=1em,skipbelow=0em}

%%%%%%%%%%%%%%%%%%%%%%
%  Helpful Commands  %
%%%%%%%%%%%%%%%%%%%%%%

% EXAMPLE:
% 1. \createnewtheoremstyle{thmdefinitionbox}{}{}
% 2. \createnewtheoremstyle{thmtheorembox}{}{}
% 3. \createnewtheoremstyle{thmproofbox}{qed=\qedsymbol}{
%       rightline=false, topline=false, bottomline=false
%    }
% Parameters:
% 1. Theorem name.
% 2. Any extra parameters to pass directly to declaretheoremstyle.
% 3. Any extra parameters to pass directly to mdframed.
\newcommand\createnewtheoremstyle[3]{
  \declaretheoremstyle[
  headfont=\bfseries\sffamily, bodyfont=\normalfont, #2,
  mdframed={
    #3,
  },
  ]{#1}
}

% EXAMPLE:
% 1. \createnewcoloredtheoremstyle{thmdefinitionbox}{definition}{}{}
% 2. \createnewcoloredtheoremstyle{thmexamplebox}{example}{}{
%       rightline=true, leftline=true, topline=true, bottomline=true
%     }
% 3. \createnewcoloredtheoremstyle{thmproofbox}{proof}{qed=\qedsymbol}{backgroundcolor=white}
% Parameters:
% 1. Theorem name.
% 2. Color of theorem.
% 3. Any extra parameters to pass directly to declaretheoremstyle.
% 4. Any extra parameters to pass directly to mdframed.
\newcommand\createnewcoloredtheoremstyle[4]{
  \declaretheoremstyle[
  headfont=\bfseries\sffamily\color{#2}, bodyfont=\normalfont, #3,
  mdframed={
    linewidth=2pt,
    rightline=false, leftline=true, topline=false, bottomline=false,
    linecolor=#2, backgroundcolor=#2!5, #4,
  },
  ]{#1}
}

%%%%%%%%%%%%%%%%%%%%%%%%%%%%%%%%%%%
%  Create the Environment Styles  %
%%%%%%%%%%%%%%%%%%%%%%%%%%%%%%%%%%%

\makeatletter
\@ifclasswith\class{nocolor}{
  % Environments without color.

  \createnewtheoremstyle{thmdefinitionbox}{}{}
  \createnewtheoremstyle{thmtheorembox}{}{}
  \createnewtheoremstyle{thmexamplebox}{}{}
  \createnewtheoremstyle{thmclaimbox}{}{}
  \createnewtheoremstyle{thmcorollarybox}{}{}
  \createnewtheoremstyle{thmpropbox}{}{}
  \createnewtheoremstyle{thmlemmabox}{}{}
  \createnewtheoremstyle{thmexercisebox}{}{}
  \createnewtheoremstyle{thmdefinitionbox}{}{}
  \createnewtheoremstyle{thmquestionbox}{}{}
  \createnewtheoremstyle{thmsolutionbox}{}{}

  \createnewtheoremstyle{thmproofbox}{qed=\qedsymbol}{}
  \createnewtheoremstyle{thmexplanationbox}{}{}
}{
  % Environments with color.

  \createnewcoloredtheoremstyle{thmdefinitionbox}{definition}{}{}
  \createnewcoloredtheoremstyle{thmtheorembox}{theorem}{}{}
  \createnewcoloredtheoremstyle{thmexamplebox}{example}{}{
    rightline=true, leftline=true, topline=true, bottomline=true
  }
  \createnewcoloredtheoremstyle{thmclaimbox}{claim}{}{}
  \createnewcoloredtheoremstyle{thmcorollarybox}{corollary}{}{}
  \createnewcoloredtheoremstyle{thmpropbox}{prop}{}{}
  \createnewcoloredtheoremstyle{thmlemmabox}{lemma}{}{}
  \createnewcoloredtheoremstyle{thmexercisebox}{exercise}{}{}

  \createnewcoloredtheoremstyle{thmproofbox}{proof}{qed=\qedsymbol}{backgroundcolor=white}
  \createnewcoloredtheoremstyle{thmexplanationbox}{example}{qed=\qedsymbol}{backgroundcolor=white}
}
\makeatother

%%%%%%%%%%%%%%%%%%%%%%%%%%%%%
%  Create the Environments  %
%%%%%%%%%%%%%%%%%%%%%%%%%%%%%

\declaretheorem[numberwithin=section, style=thmtheorembox,     name=Theorem]{theorem}
\declaretheorem[numbered=no,          style=thmexamplebox,     name=Example]{example}
\declaretheorem[numberwithin=section, style=thmclaimbox,       name=Claim]{claim}
\declaretheorem[numberwithin=section, style=thmcorollarybox,   name=Corollary]{corollary}
\declaretheorem[numberwithin=section, style=thmpropbox,        name=Proposition]{prop}
\declaretheorem[numberwithin=section, style=thmlemmabox,       name=Lemma]{lemma}
\declaretheorem[numberwithin=section, style=thmexercisebox,    name=Exercise]{exercise}
\declaretheorem[numbered=no,          style=thmproofbox,       name=Proof]{replacementproof}
\declaretheorem[numbered=no,          style=thmexplanationbox, name=Proof]{expl}

\makeatletter
\@ifclasswith\class{nocolor}{
  % Environments without color.

  \newtheorem*{note}{Note}

  \declaretheorem[numberwithin=section, style=thmdefinitionbox, name=Definition]{definition}
  \declaretheorem[numberwithin=section, style=thmquestionbox,   name=Question]{question}
  \declaretheorem[numberwithin=section, style=thmsolutionbox,   name=Solution]{solution}
}{
  % Environments with color.

  \newtcbtheorem[number within=section]{Definition}{Definition}{
    enhanced,
    before skip=2mm,
    after skip=2mm,
    colback=red!5,
    colframe=red!80!black,
    colbacktitle=red!75!black,
    boxrule=0.5mm,
    attach boxed title to top left={
      xshift=1cm,
      yshift*=1mm-\tcboxedtitleheight
    },
    varwidth boxed title*=-3cm,
    boxed title style={
      interior engine=empty,
      frame code={
        \path[fill=tcbcolback]
        ([yshift=-1mm,xshift=-1mm]frame.north west)
        arc[start angle=0,end angle=180,radius=1mm]
        ([yshift=-1mm,xshift=1mm]frame.north east)
        arc[start angle=180,end angle=0,radius=1mm];
        \path[left color=tcbcolback!60!black,right color=tcbcolback!60!black,
        middle color=tcbcolback!80!black]
        ([xshift=-2mm]frame.north west) -- ([xshift=2mm]frame.north east)
        [rounded corners=1mm]-- ([xshift=1mm,yshift=-1mm]frame.north east)
        -- (frame.south east) -- (frame.south west)
        -- ([xshift=-1mm,yshift=-1mm]frame.north west)
        [sharp corners]-- cycle;
      },
    },
    fonttitle=\bfseries,
    title={#2},
    #1
  }{def}

  \NewDocumentEnvironment{definition}{O{}O{}}
    {\begin{Definition}{#1}{#2}}{\end{Definition}}

  \newtcolorbox{note}[1][]{%
    enhanced jigsaw,
    colback=gray!20!white,%
    colframe=gray!80!black,
    size=small,
    boxrule=1pt,
    title=\textbf{Note:-},
    halign title=flush center,
    coltitle=black,
    breakable,
    drop shadow=black!50!white,
    attach boxed title to top left={xshift=1cm,yshift=-\tcboxedtitleheight/2,yshifttext=-\tcboxedtitleheight/2},
    minipage boxed title=1.5cm,
    boxed title style={%
      colback=white,
      size=fbox,
      boxrule=1pt,
      boxsep=2pt,
      underlay={%
        \coordinate (dotA) at ($(interior.west) + (-0.5pt,0)$);
        \coordinate (dotB) at ($(interior.east) + (0.5pt,0)$);
        \begin{scope}
          \clip (interior.north west) rectangle ([xshift=3ex]interior.east);
          \filldraw [white, blur shadow={shadow opacity=60, shadow yshift=-.75ex}, rounded corners=2pt] (interior.north west) rectangle (interior.south east);
        \end{scope}
        \begin{scope}[gray!80!black]
          \fill (dotA) circle (2pt);
          \fill (dotB) circle (2pt);
        \end{scope}
      },
    },
    #1,
  }

  \newtcbtheorem{Question}{Question}{enhanced,
    breakable,
    colback=white,
    colframe=myblue!80!black,
    attach boxed title to top left={yshift*=-\tcboxedtitleheight},
    fonttitle=\bfseries,
    title=\textbf{Question:-},
    boxed title size=title,
    boxed title style={%
      sharp corners,
      rounded corners=northwest,
      colback=tcbcolframe,
      boxrule=0pt,
    },
    underlay boxed title={%
      \path[fill=tcbcolframe] (title.south west)--(title.south east)
      to[out=0, in=180] ([xshift=5mm]title.east)--
      (title.center-|frame.east)
      [rounded corners=\kvtcb@arc] |-
      (frame.north) -| cycle;
    },
    #1
  }{def}

  \NewDocumentEnvironment{question}{O{}O{}}
  {\begin{Question}{#1}{#2}}{\end{Question}}

  \newtcolorbox{Solution}{enhanced,
    breakable,
    colback=white,
    colframe=mygreen!80!black,
    attach boxed title to top left={yshift*=-\tcboxedtitleheight},
    title=\textbf{Solution:-},
    boxed title size=title,
    boxed title style={%
      sharp corners,
      rounded corners=northwest,
      colback=tcbcolframe,
      boxrule=0pt,
    },
    underlay boxed title={%
      \path[fill=tcbcolframe] (title.south west)--(title.south east)
      to[out=0, in=180] ([xshift=5mm]title.east)--
      (title.center-|frame.east)
      [rounded corners=\kvtcb@arc] |-
      (frame.north) -| cycle;
    },
  }

  \NewDocumentEnvironment{solution}{O{}O{}}
  {\vspace{-10pt}\begin{Solution}{#1}{#2}}{\end{Solution}}
}
\makeatother

%%%%%%%%%%%%%%%%%%%%%%%%%%%%
%  Edit Proof Environment  %
%%%%%%%%%%%%%%%%%%%%%%%%%%%%

\renewenvironment{proof}[1][\proofname]{\vspace{-10pt}\begin{replacementproof}}{\end{replacementproof}}
\newenvironment{explanation}[1][\proofname]{\vspace{-10pt}\begin{expl}}{\end{expl}}

\theoremstyle{definition}

\newtheorem*{notation}{Notation}
\newtheorem*{previouslyseen}{As previously seen}
\newtheorem*{problem}{Problem}
\newtheorem*{observe}{Observe}
\newtheorem*{property}{Property}
\newtheorem*{intuition}{Intuition}

%%%%%%%%%%%%%%%%%%%%%%%%%%%%%%%%%%%%%%%%%%%%%%%%%%%%%%%%%%%%%%%
%                 Code Highlighting                           %
%%%%%%%%%%%%%%%%%%%%%%%%%%%%%%%%%%%%%%%%%%%%%%%%%%%%%%%%%%%%%%%
\usepackage{listings}
\lstset{
language=Octave,
backgroundcolor=\color{white},   % choose the background color; you must add \usepackage{color} or \usepackage{xcolor}
basicstyle=\footnotesize\ttfamily,        % the size of the fonts that are used for the code
breakatwhitespace=false,         % sets if automatic breaks should only happen at whitespace
breaklines=true,                 % sets automatic line breaking
captionpos=b,                    % sets the caption-position to bottom
commentstyle=\color{gray},    % comment style
%escapeinside={\%*}{*)},          % if you want to add LaTeX within your code
extendedchars=true,            % lets you use non-ASCII characters; for 8-bits encodings only, does not work with UTF-8
frame=single,                    % adds a frame around the code
% frameround=fttt,
keepspaces=true,                 % keeps spaces in text, useful for keeping indentation of code (possibly needs columns=flexible)
columns=flexible,
classoffset=0,
keywordstyle=\color{RoyalBlue},       % keyword style
deletekeywords={function,endfunction, if,endif},
classoffset=1,
morekeywords={function,endfunction, if,endif},
keywordstyle=\bf\color{Red},       % keyword style
classoffset=2,
morekeywords={persistent},            % if you want to add more keywords to the set
keywordstyle=\bf\color{ForestGreen},       % keyword style
classoffset=0,
literate=
{/}{{{\color{Mahogany}/}}}1
{*}{{{\color{Mahogany}*}}}1
{.*}{{{\color{Mahogany}.*}}}2
{+}{{{\color{Mahogany}+{}}}}1
{=}{{{\bf\color{Mahogany}=}}}1
{-}{{{\color{Mahogany}-}}}1
{[}{{{\bf\color{RedOrange}[}}}1
{]}{{{\bf\color{RedOrange}]}}}1
{ç}{{\c{c}}}1 % Cedilha
{á}{{\'{a}}}1 % Acentos agudos
{é}{{\'{e}}}1
{í}{{\'{i}}}1
{ó}{{\'{o}}}1
{ú}{{\'{u}}}1
{â}{{\^{a}}}1 % Acentos circunflexos
{ê}{{\^{e}}}1
{î}{{\^{i}}}1
{ô}{{\^{o}}}1
{û}{{\^{u}}}1
{à}{{\`{a}}}1 % Acentos graves
{è}{{\`{e}}}1
{ì}{{\`{i}}}1
{ò}{{\`{o}}}1
{ù}{{\`{u}}}1
{ã}{{\~{a}}}1 % Tils
{ẽ}{{\~{e}}}1
{ĩ}{{\~{i}}}1
{õ}{{\~{o}}}1
{ũ}{{\~{u}}}1,
numbers=left,                    % where to put the line-numbers; possible values are (none, left, right)
numbersep=6pt,                   % how far the line-numbers are from the code
numberstyle=\tiny\color{gray}, % the style that is used for the line-numbers
rulecolor=\color{black},         % if not set, the frame-color may be changed on line-breaks within not-black text (e.g. comments (green here))
showspaces=false,                % show spaces everywhere adding particular underscores; it overrides 'showstringspaces'
showstringspaces=false,          % underline spaces within strings only
showtabs=false,                  % show tabs within strings adding particular underscores
stepnumber=1,                    % the step between two line-numbers. If it's 1, each line will be numbered
stringstyle=\color{purple},     % string literal style
tabsize=2,                       % sets default tabsize to 2 spaces
}


\title{\Huge{23MAT102}\\ Class Notes}
\author{\huge{Adithya Nair}}
\date{}

\begin{document}

\maketitle
\newpage% or \cleardoublepage
% \pdfbookmark[<level>]{<title>}{<dest>}
\tableofcontents

\pagebreak
\section{A Basic Order Of Importance}
\begin{itemize}
    \item \textbf{Axiom} - Statements taken as fact
    \item \textbf{Theorem} - Statements that are proven using axioms
    \item \textbf{Lemma} - Statements proven using theorems
    \item \textbf{Proposition} - Statements, regardless of whether it is true or false, is assumed to be true
    \item \textbf{Corollary} - A theorem that is proven using another theorem.*
\end{itemize}
\chapter{A Revision Of Sets And Functions}
Sets are assumed to be sets on the basis of a theory known as \textbf{Naive Set Theory}. according to this theory, A set is defined as,

\begin{definition}[Sets]
A set is a collection of objects
\end{definition}

e.g. - \[\mathbb{N}, \mathbb{Z}, \mathbb{Q}, \mathbb{R}, \mathbb{C}\]

\section{Notations}
A,B...Z will denote sets
a,b...z will denote elements
a $\in$ A, a is an element of A
a $\notin$  A, a is not an element of A

\section{Roster Notation}
\[\mathbb{N} = \{1,2...\}\]
\[A = \{2,4,6,8\]
\[B = \{ x \in Z+| x < 10 \} \]
 B is written in set builder form

\section{Basic Concepts Of Sets}
\begin{definition}[Subsets]
A and B are two sets. A is a subset of B, and we write A $\subset$ B, if every element of A is also an element of B
\end{definition}

\begin{theorem}
    Two sets A and B are equal and we write A=B if and only if A $\subset$ B and B $\subset$ A
\end{theorem}

\begin{definition}[Unions]
    The union of two sets A and B, denoted by A $\cup$ B, is
    \[ A \cup B = \{x|\ x \in A \ and \ x \in B\}\]
\end{definition}
\begin{definition}[Intersections]
    The intersection of two sets denoted by A $\cap$ B, is
    \[A \cap B = \{x|\ x \in A \ or \ x \in B\}\]
\end{definition}
\begin{definition}[Set Difference]
    The difference of two sets denoted by A$\setminus$ B is
    \[A\setminus B = {x|\ x \in A \ and \ x \notin B}\]
\end{definition}

\begin{definition}[Set Complement]
    The complement of a set A, denoted by A $^{C}$ is,
    \[A^{C} = \{x \in X |\ x \notin A \}\]
\end{definition}
\begin{itemize}
    \item $(B \cup C)^{C} = B^{C}  \cap C^{C}$
    \item $(B \cap C)^{C} = B^{C} \cap C^{C}$
    \item $A\setminus (B\cup C) = (A\setminus B) \cap (A \setminus C)$
    \item $A\setminus (B\cap C) = (A\setminus B) \cup (A \setminus C)$
\end{itemize}
\section{Logical Notation}
$\forall$ - for all
$\exists$ - there exists
$\exists!$ - there exists a unique

\section{Functions}
f: A $\rightarrow$ B
f(a) = b, a $\in$ A, b $\in$ B
A is the \textbf{domain} of the function, B is the \textbf{codomain} of the function and, 
\{b$\in$ B \textbar f(a) = b \} - Range
\section{Cartesian Product}
\[A \times B = \{(a,b) | a \in A, b \in B\}\]



\section{Composition Of Functions}
(g$\circ$f)(x) = (g(f(x))

A function is the same as a mapping, which is the same as a transformation 
\section{Types Of Functions}
\begin{enumerate}
    \item f is injective(one-one) if,
    \[f(a) = f(a')\ then\ a\ =\ a'\]
    \item f is surjective(onto) if, 
    \[\forall b \in B, \exists \ a \in A, \  f(a) = b\]
    \item f is bijective if f is injective and surjective
    \end{enumerate}   
\section*{Reference}
Knowles - Linear Vector Spaces and Cartesian Tensors\\
Halmos - Finite Dimensional Linear Spaces\\
Gelfand - Linear Algebra
\chapter{Linear Algebra}
A vector space over a field F = $\mathbb{R}$ or $\mathbb{C}$ is a set V with two operations:
\begin{enumerate}
\item +:V$\times$ V $\rightarrow$ V i.e. "+" is closed under addition.

\item $\cdot$:F $\times$ V $\rightarrow$ V, i.e. $"\cdot"$ is closed under multiplication
\end{enumerate}
having the following properties

\begin{enumerate}
    \item \textbf{Associativity}
    \[\forall \ v_1, v_2,v_3 \in V, (v_1 + v_2) + v_3 = v_1 + (v_2 + v_3)\]
    \item \textbf{Existence of identity element}
    \[\exists! \ 0 \in V, \forall v\ in V, such that 0 + v = v\]
    \item \textbf{Existence of additive inverse}
    \[\forall \ v \in V \exists (-v) \in V, v + (-v) = 0\]
    \item \textbf{Commutativity}
    \[\forall \ u, v \in V, u+v = v+u \]
        Properties 1 to 4 constitute a group known as the "abelian group" or "commutative group"
    \item \textbf{Existence of multiplactive identity}
    \[\exists! \ 1 \in V, \ such \ that \ \forall \ v \in V, 1\cdot v = v\]
    \item \textbf{Associativity}
    \[\mu, \lambda \in F, v \in V, \lambda(\mu \cdot v) = (\lambda\mu)\cdot v\]
    \item \textbf{Distribution of + over $\cdot$}
    \[(\lambda + \mu) \cdot v = \lambda \cdot v + \mu \cdot v, \forall \ \mu, \lambda \in F\]
    \item \textbf{Distribution of $\cdot$ over +}
    \[\lambda \cdot(u + v)\ = \lambda \cdot u + \lambda \cdot v, \forall \lambda \in \ F, u,v \in V\]    
\end{enumerate}
\section{Examples Of Vector Spaces}
\begin{enumerate}
    \item V = {0}
    \item $\mathbb{R}$
    \item All polynomials of order \textbf{at most} n
\end{enumerate}
\section*{Reference}
    \begin{itemize}
        \item Donald Knuth
        \item Marvin Mirsky, MIT
        \item Web Of Stories, Youtube Channel
        \item Axler, Chapter 1
        \item Olver, Shakiban, Chapter 2
        \item Terrence Tao Notes - AMS Open Math
    \end{itemize}
    \section{Some Theorems And Proofs Regarding Vector Spaces}
\begin{theorem}
    Additive identity is unique
\end{theorem}
    \begin{proof}
        Suppose $\exists \ additive \ identities \ 0_1$, $0_2$ such that
        \[\forall u \in V, 0_1 + u \ \& \ 0_2 + u = u\]
        \[0_1 + 0_2 = 0_2\]
        \[0_2 + 0_1 = 0_1\]
        \[\therefore 0_1 = 0_2\]
    \end{proof}
\begin{theorem}
    Additive inverse is unique
\end{theorem}
    \begin{proof}
        Suppose additive inverses of u are $v_1, v_2$
        \[u + v_1 = 0, u + v_2 = 0\]
        \[v_2 + (u+v_1) = v_2 + 0\]
        \[(v_2 + u) + v_1 = v_2\]
        \[0 + v_1 = v_2\]
        \[v_1 = v_2\]
        \end{proof}
\begin{theorem}
    $0\cdot u = 0$
\end{theorem}
\begin{proof}

   Let 0 . u = 0 
   Consider,
   \[
   v + v = 0.u + 0.u = (0+0).u
   \]
\[
	= 0.u = v
\]	
\[
	\implies v+v = v \\
\]
\[
       v+(v + (-v)) = v + -v \\
\]
\[
	\implies v = 0 
\]
\end{proof}
\begin{theorem}[Scalars And Inverses]
	\[
	   (-\lambda)u = -(\lambda.u) = \lambda . (-u)
	\]
\end{theorem}
\begin{proof}
	Let \[
	   v = (-\lambda).u
	\]
	Consider,
	\begin{displaymath}
	v + \lambda. u = (-\lambda).u + \lambda. u
	\end{displaymath}
        \[
            \indent = (\lambda + \lambda).u
        \]
        \[
            = 0.u = 0
        \]	
        \[
            \therefore (\lambda .u) + -(\lambda.u) = 0
        \]
        \[
            =(-\lambda.u) + (\lambda.u) + -(\lambda.u) = (-\lambda.u)
        \]
        \[
            = (-\lambda).u + 0 = (-\lambda.u)  
        \]
        \[
            (-\lambda).u = -(\lambda.u)
        \]
\end{proof}

\section{Fields}
A) To every pair $\alpha$ and $\beta$ of scalars, there corresponds a scalar $\alpha + \beta$ called the sum of $\alpha \text{ and } \beta$, in such a way that 
\begin{enumerate}
	\item addition is commutative, $\alpha + \beta = \beta + \alpha$
	\item addition is associative, $(\alpha + \beta)+\gamma = \alpha + (\beta + \gamma)$   
	\item There exists a unique scalar 0, called zero, such that $\alpha + 0 = \alpha$ for every scalar $\alpha$, and 
	\item to every scalar $\alpha$, there corresponds a unique scalar $(-\alpha)$ such that $\alpha + (-\alpha) = 0$ 
\end{enumerate}
B) To every pair $\alpha$ and $\beta$ of scalars there corresponds a scalar $\alpha\beta$, called the product of $\alpha$ and $\beta$ in such a way that
\begin{enumerate}
    \item multiplication is commutative, $\alpha\beta = \beta\alpha$
    \item multiplication is associative, $(\alpha\beta)\gamma = \alpha(\beta\gamma)$
    \item there exists a unique non-zero scalar 1(called one) such that $\alpha 1 = \alpha$ for every scalar $\alpha$ and
    \item to every non-zero scalar $\alpha$, there corresponds a unique scalar $\alpha^{-1} = \frac{1}{\alpha}$ such that $\alpha \alpha^{-1} = 1$
\end{enumerate}
C) Multiplication is distributive with respect to addition,
\[
    \alpha(\beta + \gamma) = (\alpha\beta + \alpha \gamma )
\]
If addition and multiplication are defined within same set of objects(scalars) so that the conditions A,B and C are satisfied, then that set is called a field.

\begin{note}
    The main difference between vector spaces and fields: \\
    All fields are vector spaces over themselves. 
    The main difference arises in the operation. 
    Vector spaces have operations:
    \[
        +: V \times V \rightarrow V
    \]
    \[
        \cdot : F \times V \rightarrow V
    \]
    While fields have both operations:
    \[
        +, \cdot : F \times F \rightarrow F
    \]
    Another key difference is that fields have an axiom regarding a unique scalar known as the multiplicative inverse $(\alpha^{-1})$ which is not an axiom for vector spaces.
\end{note}
\section{Examples Of Vector Spaces And Fields}
Examples of fields include: $\mathbb{Q},\mathbb{R},\mathbb{C}$
\begin{itemize}
    \item For complex numbers \[\mathbb{C}\text{(Complex Numbers)} = \{(a,b): a,b \in R\}\]
        \[
            (a,b)+(c,d) = (a+c,b+d) \\
        \]
        \[
            (a,b)(c,d) = (ac-bd,ad+bc)
        \]
        This is a field.
    \item $\mathbb{R}^n$, n-tuple where $x = (x_1,x_2,x_2 \dots x_n)$
        This is a vector space over $\mathbb{R}$
    \item $P_n(\mathbb{R})$, the set of all polynomials with degree n is a vector space over $\mathbb{R}$.
    \begin{note}
        $P_n(\mathbb{R})$ has a direct correlation to $(\mathbb{R}^{n+1})$ which means they are \textbf{isomorphic} spaces
    \end{note}
    \item C($\mathbb{R}$) - the space of all continuous functions is a vector space over field $\mathbb{R}$
    \item $M_{m \times n}$ : set of all $m \times n$ matrices is a vector space
    \item \textbf{Linear maps/operations/transformations/functions} - 
        Suppose U,V are two vector spaces over a field F, then $T:u \rightarrow v$ is linear for some scalars $\alpha,\beta \in F$, we have $T(\alpha u_1 + \beta u_2 = \alpha T(u_1)) + \beta T(u_2)$
\end{itemize}

\section{Subspaces}
(V,+, $\cdot$) - is a vector space \\
(W $\subset$ V, +, $\cdot$) - is a subspace of V

A subspace is a vector space, where the set is a subset of another vector space.

\begin{note}
    All lines that pass through the origin in a 2-dimensional plane are subspaces of $\mathbb{R}^2$
\end{note}
\pagebreak
\begin{lemma}
    Let V be a vector space and let W be a subset of V . Then W is a subspace of V if and only if the following properties hold:
    \begin{enumerate}
        \item W is closed under addition, 
            \begin{align*}
                \text{If } w_1, w_2 \in W, \text{ then} \\
            w_1 + w_2 \in W
            \end{align*}
        \item W is closed under scalar multiplication.
            \[
                \text{If }  \alpha \in F, w \in W, \text{ then } \alpha w \in W
            \]
    \end{enumerate}
\end{lemma}
\begin{proof}
    \begin{enumerate}
        \item W is a subspace $\implies$ 1. and 2.\\
            W is closed under addition and scalar multiplication, because this is implicit in the definition of subspaces.
        \item 1. and 2. $\implies$ W is a subspace  \\
            $-u \in W$, because $-1.u = -u$ \\
            $0 \in W$ because $0.u = 0$
    \end{enumerate}
\end{proof}
\section{Span}
\begin{definition}[Spans]
    Let S be a finite collection of vectors in a vector space V. 
    A linaer combination of S is defined to be any vector in V of the form 
    \[
        \alpha_1 v_1 + \alpha_2 v_2 \dots + \alpha_n v_n
    \]
\end{definition}
\begin{theorem}
    Let S be a subset of a vector space V. Then span(S) is a subspace of V which contains S is a subspace of V which contains S. Moreoever, ny subspace of V which contains S as a subset must in fact contain all of span(S)
\end{theorem}
\begin{proof}
    To prove, 
    i) span(S) is a subspace of V 
    ii) span(S) $\subseteq V$ 

  Proving  span(S) $\subseteq V$,
    Let $\alpha_1v_1 + \alpha_2 v_2 \dots + \alpha_nv_n$ be any element in span(S), where $\alpha_i \in F, v_i \in S \text{ and } i = \{1,\dots n\}$ 
    Then
    \[
        \alpha_1v_1 + \alpha_2 v_2 + \dots + \alpha_nv_n \in span(S)
    \]
    Since, $\cdot$ and + are closed under V, $\alpha_1v_1 + \alpha_2 v_2 + \dots + \alpha_nv_n \in V$ \\
    $\therefore span(S) \in V$
\end{proof}

\section{Linear dependence and independence}
\begin{definition}[Linear dependence]
    Any collection S of vectors in a vector space V are linearly dependent, if we can find scalars \[
        \alpha_1, \alpha_2 \dots, \alpha_n \in \text{ F not all zero, such that}
    \]
    \begin{displaymath}
      \alpha_1 v_1 + \alpha_2v_2 + \dots \alpha_n v_n = 0  
    \end{displaymath}
\end{definition}
\section{Basis}
\begin{definition}[Basis]
    A basis in a vector space V is a set S of linearly independent vectors such that every vector in V is a linear combination of elements in S, i.e. S is the set of linearly independent vectors and is the spanning set of V.
\end{definition}
\begin{definition}[Dimension]
    The dimension of a finite dimensional vector space V is the number of elements in a basis of V.
\end{definition}
\begin{lemma}
    Let $\{v_1,v_2,\dots v_n\}$ be a basis for a vector space V. Then every vector v can be written in the form,
    \begin{equation}
        v = \alpha_1 v_1 + \alpha_2 v_2 + \dots + \alpha_nv_n 
        \label{eq:1}
    \end{equation}
\end{lemma}
\begin{proof}
    Let, 
    \begin{equation}
        v = \beta_1v_1+\beta_2 v_2 + \dots \beta_n v_n
        \label{eq:2}
    \end{equation}
    be another representation of v in $v_1,v_2 \dots v_n$
    \[
        \ref{eq:2} - \ref{eq:1}
    \]
    \[
        0 = (\beta_1 - \alpha_1)v_1 + (\beta_2 - \alpha_2)v_2 \dots + (\beta_n - \alpha_n)v_n
    \]    
    \[
        \therefore \alpha_1 = \beta_1, \alpha_2 = \beta_2 \dots \alpha_n = \beta_n
    \]
    ii) W is any subspace of V, which contains S. 

    Take any term v, which is a linear combination of $\alpha_i v_i$
    \[
        v = \alpha_1 v_1 + \alpha_2 v_2 \dots + \alpha_n v_n ,
    \]
    \[
       \therefore v \in span(S) \in W
    \]
\end{proof}
\begin{theorem}
    Let V be a vector space, and let S be a linearly independent subset of V. Let v be a vector which does not lie in S.
    a) If v lies in span(S) then the set $S \cup \{v\}$ is linearly dependent and span($S \cup \{v\}$ = span(S)
\end{theorem}


\end{document}
