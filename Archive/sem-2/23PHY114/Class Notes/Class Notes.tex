\documentclass{report}
\input{preamble}
\usepackage{cancel}
\title{\Huge{23PHY114}\\ Class Notes}
\author{\huge{Adithya Nair}}
\date{}
\begin{document}

\maketitle
\newpage% or \cleardoublepage
% \pdfbookmark[<level>]{<title>}{<dest>}
\tableofcontents
\chapter{Solids}
\createintro
\section{Moment Of Area}
The moment of inertia is used to help find the "resistance" to the force, given a specific axis or direction.

\section{Resisting Force And Moments From Supports} % (fold)
\label{sec:resisting_force_and_moments_from_supports}
There are three main kinds of supports - 
\begin{enumerate}
  \item Pin/Hinge - Fixes linear motion but leaves rotation free. 
  \item Roller - Fixes rotation but leaves motion free. 
  \item Clamp - Fixes both linear and rotational motion.
\end{enumerate}
% section Resisting Force And Moments From Supports (end)
Take this case, with a bunch of forces being applied to the given load. 

There are three main things we need to find for this figure. 
\begin{enumerate}
  \item The resultant force acting on this bar fixed to a hinge. 
  \item The support reaction force and moment.
  \item The moment on the object (maximum)
\end{enumerate}
The way to approach the problem is as always,
\begin{enumerate}
  \item Free Body Diagram first. 
  \item Assuming $\Sigma f = 0 $, because the object has no acceleration currently, because it is fixed.
  \item Assuming $\Sigma M = 0$
  \item The point at which the resultant force is applied is found by, 
    \[
      \frac{\int |r|dm}{\int dm}
    \]
\end{enumerate}
\section{Derivation Of The Uniaxial Formula}

\section{Code For Uniaxial Deformation} % (fold)


\label{sec:for_uniaxial_deformation}
Main Subroutines For Uniaxial Deformation
% section For Uniaxial Deformation (end)
\subsection{Finding The Local Stiffness Matrix} % (fold)
\begin{lstlisting}{octave}
function stiffnessLocal = localStiffnessGenerator(E,A,l,theta);
    stiffnessConstant = E*A/l;
    R = [cos(theta) -sin(theta); sin(theta) cos(theta)];
    stiffnessMatrix = [stiffnessConstant 0 -stiffnessConstant 0; zeros(1,4);-stiffnessConstant 0 stiffnessConstant 0; zeros(1,4)];
    R4 = [R zeros(2,2); zeros(2,2) R];
    stiffnessLocal = R4*stiffnessMatrix*R4';
end
\end{lstlisting}
\label{sec:finding_the_local_stiffness_matrix}
% subsubsection Finding The Local Stiffness Matrix (end)
\subsection{Converting The Local Stiffness To A Global Stiffness Matrix}
\begin{lstlisting}{octave}
function stiffnessLocalGlobal = local2Global(stiffnessLocal,node1,node2,nodeCount)
    stiffnessLocalGlobal = zeros(2*nodeCount,2*nodeCount);
    i = 2*node1 - 1;
    j = 2*node2 - 1;
    stiffnessLocalGlobal(i:(i+1),i:(i+1)) = stiffnessLocal(1:2,1:2);
    stiffnessLocalGlobal(i:(i+1),j:(j+1)) = stiffnessLocal(1:2,3:4);
    stiffnessLocalGlobal(j:(j+1),i:(i+1)) = stiffnessLocal(3:4,1:2);
    stiffnessLocalGlobal(j:(j+1),j:(j+1)) = stiffnessLocal(3:4,3:4);
end
\end{lstlisting}
\subsection{Main Loop Through Evaluating The Global Stiffness Matrix}
\begin{lstlisting}{octave}
 A = 0; theta = 0; l = 0; stiffnessLocal = zeros(4,4); nodeAxialForces = zeros(nodeCount,1);
for element = 1:elementCount % For first three bars
    A = areaVector(element);
    theta = angleVector(element);
    l = lengthVector(element);
    stiffnessLocal = localStiffnessGenerator(E,A,l,theta);
    nodeCounter = element*2 - 1;
    stiffnessLocalGlobal = local2Global(stiffnessLocal,nodeVector(nodeCounter),nodeVector(nodeCounter+1),nodeCount);
    stiffnessLocal
    stiffnessLocalGlobal
    stiffnessGlobal += stiffnessLocalGlobal;
end
\end{lstlisting}
\subsection{Applying Boundary Conditions}
\begin{lstlisting}{octave}
selectedVector = [3 5:end]
forceEval = forceVector(selectedVector);
displacementEval = displacementVector(selectedVector);
stiffnessEval = stiffnessGlobal(selectedVector);
displacementEval = stiffnessEval\forceEval;
\end{lstlisting}
\section{Deriving For Bending Deformation}
Taking this example of a bar fixed to the wall, and examining small components of the bar and examining the forces at work on the component, we get a Tension, Moment and Stress.

Taking things along the $\hat{i}$ direction, we get
$$
-T + T + \Delta T = 0
$$
 What does this tell us? Tension is constant. 

While $-S + S + \Delta S - w\Delta x = 0$

$$
\frac{dS}{dx} = \lim_{\Delta x \rightarrow 0} \frac{\Delta S}{\Delta x} = w(x)
$$
We get an expression for the effects of the distributed force on the shear force of an individual object.

Looking at the angular momentum balance,

$$
\sum M_{/P} = 0
$$
$$
-M + (M + \Delta M) \hat{k} + (-\Delta x \hat{i} \times(-T\hat{i} - S \hat{j}) + (\frac{-\Delta x}{2} \hat{i} \times (-w(x) \Delta x \hat{j})) $$ $$
\implies \frac{\Delta M }{\Delta x} + S - w\frac{\Delta x}{2} = 0
$$
Taking the limit on the equations, we get,
$$
\frac{dM}{dx} + S = 0
$$

Thus we get,
$$
\frac{dT}{dx} = 0
$$
$$
\frac{dS}{dx} = w(x)
$$
$$
\frac{dM}{dx} = -S
$$
Now we get to the Finite Element Method for bending.
$$
\theta(x) = v'(x)
$$
We find by simplification,

$$
c_1 = v_l 
$$
$$
c_2 = \theta_l
$$
$$
c_3 = \frac{3}{l^2}(v_r-v_l) - \frac{1}{l} (2\theta_l + \theta_r)
$$
$$
c_4 = \frac{2}{l^3} (v_l-v_r) + \frac{\theta_l+\theta_r}{l^2}
$$

$$
v(x) = H_1(x) v_l + H_2(x)\theta_l (H_3(x)) v_r + (H_4(x)) \theta_r
$$
$$
H_1 = 2(\frac{x}{l}^3) - 3\frac{x}{l}^2 + 1
$$
$$
H_2 = x-2\frac{x^2}{l} + \frac{x^3}{l^2}
$$
$$
H_3 = 3(\frac{x}{l})^2 -2\frac{x}{l}^3
$$
$$
H_4 = \frac{x^3}{l^2} - \frac{x^2}{l}
$$



Since we know that,
$$
\int_0^l EIq''v''dx
$$
$$
v = \underline{H}(x)^T v
$$
$$
q = H(x)
$$

\section{Quick Reference Notes}
\subsection{Derivations}
The governing differential equation is 
\[
  EI v^{(4)} = w(x)
\]

Then we can form a linear relationship
\begin{align*}
  v(x) & - \text{deflection} \\
  v^{(2)}(x) = \theta(x) & - \text{slope} \\
  EIv^{(2)}(x) = M(x) & - \text{bending moment related to curvature} \\
  EIv^{(3)}(x) = \frac{dM}{dx} = V(x) & - \text{transverse shear} \\ 
  EIv^{(4)}(x) = \frac{dV}{dx} = w(x) & - \text{load}
\end{align*}

Where, v is the deflection or displacement of the beam. E is the Young's Modulus and I is 

\subsection{Deriving Stiffness Matrix}
Use Hermite's interpolation formula to derive cubic shape functions for the deflection of beams.

\section{Final Equation}
For a bar, with axial, bending and shear deformation... The final equation is, where R, S and M are Axial, Shear and Moment.

\[
  F = k\Delta x
\]

\[
  \begin{bmatrix}
    R_1 \\ 
    S_1 \\
    M_1\\
    R_2 \\ 
    S_2 \\
    M_2\\
  \end{bmatrix}
  = 
  \begin{bmatrix}
    C_1 & 0 & 0 & -C_1 & 0 & 0 \\ 
    0 & 12C_2 & 6C_2L & 0 & -12C_2 & 6C_2L \\
    0 & 6C_2L & 4C_2L^2 & 0 & -6C_2L & 2C_2L^2 \\
    -C_1 & 0 & 0 & C_1 & 0 & 0 \\ 
    0 & -12C_2 & -6C_2L & 0 & 12C_2 & -6C_2L \\
    0 & 6C_2L & 2C_2L^2 & 0 & -6C_2L & 2C_2L^2 \\
  \end{bmatrix}
  \begin{bmatrix}
    u_1 \\ 
    v_1 \\ 
    \theta_1 \\
    u_2 \\ 
    v_2 \\ 
    \theta_2 \\
  \end{bmatrix}
\]
Where, $C_1 = \frac{EA}{L}$ and $C_2 = \frac{EI}{L^3}$

Rotating the matrix by angle $\theta$

\[
  R = \begin{bmatrix}
    c & s & 0 & 0 & 0 & 0 \\ 
    -s & c & 0 & 0 & 0 & 0 \\ 
    0 & 0 & 1 & 0 & 0 & 0 \\
    0 & 0 & 0 & c & s & 0 \\ 
    0 & 0 & 0 & -s & c & 0 \\ 
    0 & 0 & 0 & 0 & 0 & 1 \\
  \end{bmatrix}
\]

\chapter{Fluids}
\section{Main Contents}
\textbf{Computational Fluids Dynamics}
\begin{enumerate}
  \item Flow in channels and pipes
  \item Flow over objects 
\end{enumerate}

\begin{definition}[Laminar Region]
The region in a fluid where the flow is in a straight line is known as the laminar region 
\end{definition}

\begin{definition}[Turbulent Region]
The region in a fluid where the flow is not smooth is known as the turbulent region
\end{definition}

\textbf{The motivation behind learning CFD}

Computational Fluid Dynamics has been around as long as computers have been around. Using computation to calculate and visualize the flow of fluids is the basis behind wind tunnels. Trials can be done to help optimize solutions for the best aerodynamics.

Examples of data used is the velocity profile, the force and the acceleration. These fluid simulations can take months and be used to simulate the effects of fluids on a given shape.

\section{Terminology}
In engineering terms, fluids are deformable bodies. They take up the shape of its container. Viscosity, is defined as the friction to flow - $\mu$. It dictates the rate of flow of a fluid.

The counterintuitive picture. Imagine a stack of papers on your palm.  On tilting your palm, the first stack of paper goes out then the next then the next. It is dependent on the friction of the piece of paper below it. The last piece of paper has zero friction, and flows out the easiest. But a fluid does not work that way, when some water is placed on your palm, even after flowing out, some imprint of the water remains. 

Think of fluids as layered.

Taking a co-ordinate system, plotting the horizontal and vertical velocity of a fluid, u and v.

\begin{definition}[No Slip Condition]
  A condition where the part of the fluid at the base or in contact with the surface has no horizontal velocity.
  \[
    U(x,0) = 0
  \]
\end{definition}

Unlike solids, fluids have a profile that is counterintuitive to the way we imagine physics.

Counterintuitively, a fluid's profile is parabolic, in a pipe. Closed on all sides, the middlemost component of the fluid moves with 

\begin{note}
\textbf{Lagrangian Method} -
The method of following a single point and discovering interesting properties about it is known as the Lagrangian Method

\textbf{Eulerian Method} -
The method of following a single location and performing analyses on that location.
\end{note}
The study of fluids is performed using the Eulerian Method. 

\section{Derivation Of Equations}
Taking the fluid flowing through the figure given. In the Eulerian Method, we focus on a given location. This location here is the outlined box. 

\[
\]

The mass flow rate for a given location is 
\begin{align*}
  \dot{m}_{in} - \dot{m}_{out} &= \frac{dm}{dt} &(\dot{m} = (frac{d \rho V}{dt})) \\
  \dot{m} &= \frac{d\rho V}{dt} \\
  \dot{m} &= \rho \frac{dV}{dt} \\
          &= \rho Q & \text{(Q - volume flow rate)} \\
\end{align*}
\[
  \rho \frac{dV}{dt} = \Delta y \Delta z . u
\]
\begin{align*}
  (\dot{m}_{intx} + \dot{m}_{inty}) - (\dot{m}_{outx} + \dot{m}_{outy}) &= \frac{dm_{inside}}{dt} \\
  (\rho \Delta Y \Delta Z u + \rho \Delta X \Delta Z v) - (\rho \Delta Y \Delta Z (U + \Delta U) + \rho \Delta X \Delta Z ( v + \Delta v) &= \frac{d(\rho \Delta X \Delta Y \Delta Z)}{dt} \\
  (\rho \Delta Y \Delta Z u + \rho \Delta X \Delta Z v) - (\rho \Delta Y \Delta Z (U + \Delta U) + \rho \Delta X \Delta Z ( v + \Delta v) &= \frac{d(\rho \Delta X \Delta Y \Delta Z)}{dt} \\
  - \rho \Delta Y \Delta Z \frac{\Delta U}{\Delta X} - \rho \Delta x \Delta z \frac{\Delta V}{\Delta Y} &= \Delta X \Delta Y \Delta Z \frac{d \rho}{dt} \\
  - \rho \frac{\partial U}{\partial x}  - \rho \frac{\partial v}{dy} &= \frac{d\rho}{dt} \\
\end{align*}
\section{Recall}
\begin{align*}
  \dot{m}_{in} - \dot{m}_{out} = \dot{m}_{within} & &\text{(Continuity Equation)} \\
  \frac{\partial \rho}{\partial t} = -\rho \frac{\partial u}{\partial x} - \rho \frac{\partial v}{\partial y} & &\text{(Mass Continuity Equation)} or \\
  \frac{\partial \rho}{\partial t} = -\rho \frac{\partial u}{\partial x} - \rho \frac{\partial v}{\partial y} -\rho \frac{\partial w}{\partial z} & &\text{(in 3 dimensions)} \\
\end{align*}

Density is a function of position and time, then it can be expressed as, $\rho(x,y,z,t)$. Instantaneously, density can be observed to change. Similarly with viscosity,

But the question might come about that the density in the mass continuity equation, is taken as constant, since they move outside the derivative. We assume that the density does not change with position rather than changing with time.

\section{Density As A Function Of Time And Position}
\begin{align*}
  1) &\frac{\partial \rho}{\partial t} + \frac{\partial (\rho u)}{\partial y} + \frac{\partial (\rho w)}{\partial z} = 0 \\
  2) &\frac{\partial \rho}{\partial t} + \rho (\frac{\partial u}{\partial x} + \frac{\partial v}{\partial y} + \frac{\partial w}{\partial z}) + \frac{\partial \rho}{\partial x} u + \frac{\partial \rho}{\partial y}v + \frac{\partial \rho}{\partial z} w &\text{(Applying chain rule)} \\
  3) &\frac{d \rho}{dt} = \frac{\partial \rho}{\partial t} + \frac{\partial \rho}{\partial x}\frac{\partial x}{\partial t} + \frac{\partial \rho}{\partial y}\frac{\partial y}{\partial t} + \frac{\partial \rho}{\partial z}\frac{\partial z}{\partial t} &\text{(General form of partial derivatives to total derivatives)} \\
  4) & d\frac{\rho}{dt} + \rho(\frac{\partial u}{\partial x} + \frac{\partial v}{\partial y} + \frac{\partial w}{\partial z}) &\text{(Final expression)} \\
  5) & \vec{\nabla}  = \frac{\partial }{\partial x} \hat{i} + \frac{\partial }{\partial y} \hat{j} + \frac{\partial }{\partial z} \hat{k} &\text{(Differential Operator)} \\
  6) &\vec{v} = u \hat{i} + v \hat{j} + w \hat{k} &\text{(Velocity Vector)} \\
  7)&\frac{d \rho}{dt} + \rho \vec{\nabla}\cdot{\vec{v}} = 0 &{\nabla \cdot v \text{(Divergence of $\vec{v}$)}}
\end{align*}

\section{External Forces On Fluid}
\[
  \vec{\sigma} =
  \begin{bmatrix}
    \sigma_{xx} & \sigma_{xy} & \sigma_{xz} \\
    \sigma_{yx} & \sigma_{yy} & \sigma_{yz} \\
    \sigma_{zx} & \sigma_{zy} & \sigma_{zz} \\
  \end{bmatrix}
\]

Given the fact that a fluid's smallest element does not rotate. Then,
\[
  \Sigma M_{/O} = 0
\]
Then looking at any given face for the element, the shear forces are equal and opposite to each other, from this we get. 
\begin{align*}
  \sigma_{xy} = \sigma_{yx} \\ 
  \sigma_{xz} = \sigma_{zx} \\ 
  \sigma_{yz} = \sigma_{zy} \\
\end{align*}

\section{Variation of stress across a small distance}
insert diagram here(of cube with all forces measured)

\begin{align*}
  &\Sigma \vec{f} = m \vec{a} \\
  &\hat{i}(-\sigma_{xx} \Delta y \Delta z + (\sigma_{xx} + \Delta\sigma_{xx})\Delta y \Delta z + (\sigma_{xy} + \Delta \sigma_{xy})\Delta x \Delta z - \sigma_{xy}\Delta x \Delta z))  + \\
  &\hat{j}(-\sigma_{yy} \Delta x \Delta z + (\sigma_{yy} + \Delta \sigma_{yy}) \Delta x \Delta z + (\sigma_{xy}+\Delta \sigma_{xy})\Delta y \Delta z - \sigma_{xy} \Delta y \Delta z) - (\rho \Delta x \Delta y \Delta z)g \hat{j} =  (\rho \Delta x \Delta y \Delta z)\frac{d \vec{v}}{dt}\\ 
  &\vec{v} = u \hat{i} + v \hat{j} + w \hat{k}  \\
\end{align*}

Dotting along $\hat{i}$ and $\hat{j}$, when $\Delta x \rightarrow 0, \Delta y \rightarrow 0, \Delta z \rightarrow 0$
\begin{align*}
  \frac{\partial \sigma_{xx}}{\partial x} + \frac{\partial \sigma_{xy}}{\partial y} = \rho \frac{du}{dt} \\
  \frac{\partial \sigma_{xy}}{\partial x} + \frac{\partial \sigma_{yy}}{\partial y} - \rho g = \rho \frac{dv}{dt}
\end{align*}
There are certain equations that cannot be derived, they are simply observed and a relation is formed. Such equations are known as the constituent equation. Such as Hooke's Law. 

If the fluid is Newtonian(which means that stress is directly proportional to strain) and isotropic(properties are same in all directions)
\[
  \sigma_{xx} = -p + 2 \mu \frac{\partial u}{\partial x} + \lambda \vec{\nabla}\cdot\vec{v}
\]
$\mu$-viscosity, $\lambda = -\frac{2\mu}{3}$ This constant is related to $\mu$ for many fluids(including air). 
\section{Navier-Stokes Equation For Newtonian Isotropic Fluids} % (fold)
\begin{align*}
  &-\frac{\partial P}{\partial x} + \mu(\frac{\partial^2 u}{\partial x^2} + \frac{\partial^2 u}{\partial y^2} + \frac{\partial^2 u}{\partial z^2}) = \rho \frac{du}{dt} \\
  &-\frac{\partial P}{\partial y} + \mu(\frac{\partial^2 v}{\partial x^2} + \frac{\partial^2 v}{\partial y^2} + \frac{\partial^2 v}{\partial z^2}) - \rho g = \rho \frac{dv}{dt} \\
  &-\frac{\partial P}{\partial z} + \mu(\frac{\partial^2 w}{\partial x^2} + \frac{\partial^2 w}{\partial y^2} + \frac{\partial^2 w}{\partial z^2}) = \rho \frac{dw}{dt} \\
  &\frac{du}{dt} = \frac{\partial u}{\partial t} + u \frac{\partial u}{\partial x} + v\frac{\partial u}{\partial x} + v \frac{\partial u}{\partial y} + w \frac{\partial u}{\partial z}
\end{align*}
Analyzing the given equations, here's a breakdown of what the individual terms mean, 

\begin{align*}
  &\frac{\partial P}{\partial x} &\text{(The force due to the pressure on the fluid)} \\
  &\mu(\frac{\partial^2 u}{\partial x^2} + \frac{\partial^2 u}{\partial y^2} + \frac{\partial^2 u}{\partial z^2}) &\text{(Viscous Forces)} \\
  &-\rho g & \text{(Gravitational forces)}
\end{align*}

Simplying using the gradient or ``nabla" operator, 
\begin{align*}
-\frac{\partial P}{\partial x} \hat{i} - \frac{\partial P}{\partial y}- \frac{\partial P}{\partial z}\hat{k} &= -\vec{\nabla} p \\ 
    \mu(\frac{\partial^2}{\partial x^2} + \frac{\partial^2}{\partial y^2} + \frac{\partial^2}{\partial z^2})u &= \mu \vec{\nabla} \cdot \vec{\nabla} u  \\
    \rho \frac{d \vec{v}}{dt} = \rho(\frac{\partial \vec{v}}{\partial t} + u \frac{\partial \vec{v}}{\partial x} + v \frac{\partial \vec{v}}{\partial y} + w \frac{\partial \vec{v}}{\partial z}) &= \rho(\frac{\partial \vec{v}}{\partial t} + \vec{v} \cdot \vec{\nabla} \vec{v})
\end{align*}

Resulting with, 
\[
  - \vec{\nabla}p + \mu \nabla^2\vec{v} - \rho g \hat{j} = \rho \frac{d \vec{v}}{dt} = \rho (\frac{\partial \vec{v}}{\partial t} + \vec{v} \cdot \vec{\nabla} \vec{v})
\]
The continuity equation is,
\[
  \frac{\partial \rho}{\partial t} + \vec{\nabla} \cdot (\rho \vec{v}) = \frac{d \rho}{dt} + \rho \vec{\nabla} \cdot \vec{\nabla} = 0
\]
Refer to \textit{Schaum's Fluid Dynamics Chapter 5} for derivations
\section{Special Flows}
\begin{enumerate}
  \item Bernoulli's Equation \[
   \frac{d}{dx}(p + \rho g + \frac{v^2}{2})  = 0
  \]
\item \textbf{Flow between parallel plates} - in this case, $\vec{v} = u \hat{i}$ only, since the water only flows horizontally, which means $\frac{\partial u}{\partial x} = 0$. 
  \begin{align*}
    \cancel{\frac{d \rho}{d t}} + \rho (\vec{\nabla} \cdot \vec{v}) = 0 \\ 
    \rho(\frac{\partial u}{\partial x} + \cancel{\frac{\partial v}{\partial y}} + \cancel{\frac{\partial w}{\partial z}}) = 0 \\
    \frac{\partial u}{\partial x} = 0
  \end{align*}
  The second thing to look at is, 
  \begin{align*}
    -\frac{dp}{dx} + \mu(\frac{\partial^2 u}{\partial y^2}) &= \rho \cancel{\frac{du}{dt}}+ u \cancel{\frac{\partial u}{\partial x}} + v \frac{\partial v}{\partial y} + w \cancel{\frac{\partial u}{\partial z}}) \\
  \end{align*}
  {These are fully developed flows}
  \begin{align*}
    =-\frac{\partial p}{\partial x} + \mu \frac{\partial^2 u}{\partial y^2} &= 0\\
    u \text{ is a function of y}
  \end{align*} 
  \[
    -\frac{d p}{dy} + \mu(\cancel{\frac{d^2v}{dx^2}}+ \cancel{\frac{d^2v}{dy^2}} + \cancel{\frac{d^2v}{dz^2}} - \rho g = \rho \cancel{\rho\frac{dv}{dt}} - \rho g) = \rho \frac{dv}{dt}
  \]
  \begin{align*}
    \therefore -\frac{dP}{dy} - \rho g = 0 \\
    - \frac{-d P}{dy} = \rho g \\
   \int d p = - \int \rho g dy  \\
    p = - \rho g y + c
  \end{align*}
  p is a function of y, but we don't know whether $p$ is a function of x or not. 
  Now, 
  \[
    \mu \frac{d^2 u}{dy^2} = \frac{dp}{dx}
  \]
  Since u is  a function of y, and pressure is a partial derivative of x, the only possible value for both of them to have is a constant C.
  \[
    \frac{d^2 u}{dy^2} = \frac{C}{\mu} 
  \]
  Upon first integration, 
  \[
    \frac{du}{dy} = \frac{C}{\mu} y + C_1 
  \]
  Upon 2nd integration,
  \[
    U(y) = \frac{c}{2\mu} y^2 + C_1 y + C_2
  \]
Using boundary conditions,
\[
  U(0) = 0 \text{(No slip condition)}
\]

If the top part of the plates are fixed, then $U(L) = 0$ but $U(L) = u_0$
Since,
\[
  U(y) = \frac{c}{2\mu}y^2 + c_1 y + c_2
\]
\[
  U(0) = C_2 \implies C_2 = 0
\]
\[
  U(L) = \frac{C}{2\mu}L^2 + C_1 L + C_2
\]
In no slip condition,
\[
  0 = \frac{C}{2\mu}L^2 + C_1 L \implies \frac{C}{2\mu}L = -C_1
\]
If plate is moving, 
\[
  u_0 = \frac{c}{2\mu} L^2 + C_1 L
\]
\[
  \frac{u_0}{L} = \frac{c}{2\mu}L + c_1
\]
\[
  c_1 = \frac{u_0}{L} - \frac{cL}{2\mu}
\]
\[
  U(y) = \frac{c}{2\mu}(y^2) +  c_1 y + c_2 \text{ (Poiseuille flow)}
\]
\end{enumerate}

\section{Computational Fluid Dynamics}
This is a method to solve the Navier-Stokes equation, and all other fluids equations. This equation is a partial differential equation. We take finite difference approximations(using the Taylor series refer to Maths notes.) The beauty of this method is that every partial differential equation can be approximated to a Taylor series.

And we take this further,
\[
  A \vec{x} = \vec{b}
\]
Where, $x$ is a vector composed of $u,v,w,p$. These components are functions of $x,y,z$ and $t$
\subsection{Taylor Series}
Let's look at a case where $U$ is just a function of $y$, $U(y)$. What if I wanted to find what $U(y+\Delta y)$ or some point later. 
The Taylor series of $U(y+\Delta y)$ is,
\[
  U(y+\Delta y) = U(y) + U'(y)\frac{\Delta y}{1!} + U''(y)\frac{\Delta y^2}{2!} + U'''(y)\frac{\Delta y^3}{3!} + \dots
\]
For $U(y - \Delta y)$
\[
  U(y-\Delta y) = U(y) - U'(y)\frac{\Delta y}{1!} + U''(y)\frac{\Delta y^2}{2!} - U'''(y)\frac{\Delta y^3}{3!} + \dots
\]
Let's say I want to find $U'(y)$ 
\begin{align*}
  U'(y) &= \frac{\partial u}{\partial y} = [\frac{U(y+\Delta y) - U(y)}{\Delta y}] - \Delta y (\frac{U''(y)}{2!} - U'''(y)\frac{\Delta y}{3!} + \dots)
\end{align*}
I can assume the remaining terms after $[\frac{U(y+\Delta y) - U(y)}{\Delta y}] $ to be an error of the order of $\Delta y$. What this means is that, the further away I get from y, the higher the error is going to be. It is denoted by $O(\Delta y)$

We can find another way to express U'(y) using the second equation.
\[
  U'(y) = \frac{U(y) + U(y+\Delta y)}{\Delta y} + O(\Delta y)
\]
This is termed as the forward difference approximation
\[
  U'(y) = \frac{U(y) - U(y-\Delta y)}{\Delta y} - O(\Delta y)
\]
This is termed as the backward difference approximation

On subtracting $U(y-\Delta y)$ from $U(y+\Delta y)$
\[
  U(y + \Delta y) - U(y - \Delta y) = 2 U'(y) \frac{\Delta y}{1} + 2U''(y)\frac{\Delta y^3}{3!}
\]

An expression for $U'(y)$ with this equation
\[
  U'(y) = \frac{U(y + \Delta y) - U(y - \Delta y)}{2 \Delta y} - O(\Delta y^2) 
\]
This is termed as the central difference approximation
With these three approximations, we will solve the problem from earlier.
For $U''(y)$,
\[
U''(y) = \frac{U(y+\Delta y) - 2U(y) + U(y-\Delta y)}{\Delta y^2}
\]

\section{For Unidirectional Flow}
Let's look at two cases and solve,
\begin{enumerate}
  \item $\frac{\partial p}{\partial x}$, the pressure gradient is given
    We assume, to be general, that the plates are moving with a speed, $U_h$ and $U_0$ respectively for the top and bottom.

    The equation is,
    \[
      -\frac{\partial p}{\partial x} + \mu \frac{\partial^2 u}{\partial x^2} + \rho g_x = \rho \frac{\partial u}{\partial t}
    \]
    Substituting the equations we found using the Taylor series.
    \[
      -\frac{\partial p}{\partial x} + \mu(\frac{U(y+\Delta y) - 2U(y) + U(y-\Delta y)}{\Delta y^2}) + \rho g_y = \rho(\frac{U(t + \Delta t) - U(t-\Delta t)}{2\Delta t})
    \]
    Finding the partial derivative due to time,
    \[
      U'' = \frac{U(y+\Delta y, t)-2U(y,t) + U(y-\Delta y, t)}{\Delta y^2}
    \]
    \[
      \dot{U} = \frac{U(y, t+ \Delta t) - U(y,t - \Delta t)}{2\Delta t}
    \]

    We finally arrive at the final method for solving this. We break each point of the fluid vertically into discrete locations. Termed as ($U_1, U_2,\dots, U_n$)

    \[
      U_i(t) = \frac{-\partial p}{\partial x} + \mu(\frac{U_{i+1} - 2U_{i} + U_{i-1}}{\Delta y}) + \rho g_y
    \]
    As well as,
    \[
      U_i(t) = \rho(\frac{U_i(t + \Delta t)-U(t-\Delta t)}{2\Delta t})
    \]

    So,
    \[
    U_{i}(t+\Delta t) = 2\Delta t (-\frac{-1}{\rho} \frac{\partial p}{\partial x} + g_y) + \frac{2 \Delta t \mu}{\rho}(\frac{U_{i+1} - 2U_i + U_{i-1}}{\Delta y^2}) + U_{t - \Delta t}
    \]
  These two equations form the basis of what we are trying to code.

  The final equation to be is,
  \[
    U_{new}[i] = \frac{2 \Delta t}{\rho}(-\frac{\partial p}{\partial x} + \rho g_y) + \frac{2 \Delta t \mu}{\rho \Delta y^2}(U_{now}[i+1] - 2U_{now}[i] + U_{now}[i-1] + U_{old}[i])
  \]
  What these terms mean,
  \begin{align*}
    \frac{\mu}{\rho} & & \text{(Kinematic viscosity)}-\nu \\
    \frac{\Delta t\mu}{\rho \Delta y^2} & & \text{(Diffusiion number)}-\alpha
  \end{align*}
  How do we put boundary conditions, we use the velocities of the plates.
  \begin{align*}
   &U(0,t) = U_0 = U_1(t)  \\
   &U(h,t) = U_h = U_{N+1}(t)
  \end{align*}
\begin{lstlisting}{octave}
% Defining constants
mu = 0.6;
rho = 100;
g = 0.4;
h = 1;
N = 32;
deltaY = h/N;

% Terms in the equation
nu = mu/rho; % Kinematic Viscosity
alpha = 0.51; % Diffusion Number
deltaT = alpha*rho*(deltaY^2)/mu; % Evaluating time step from alpha
dpdx = -2; % Pressure Gradient

% Velocities of the plates
U0 = 0;
Uh = 0;

% Time for code to run
T = 5;
U = zeros(floor(T/deltaT),N+1);
% Initial Conditions
y = 0:deltaY:h;

% Equation
for k = 2:floor(T/deltaT)
  U(k,1) = U0;
  U(k,N+1) = Uh;
  for i = 2:N
    U(k,i) = deltaT/rho*(-dpdx    + rho*g) + alpha*(U(k-1,i+1)-2*U(k-1,i) + U(k-1,i-1)) + U(k-1,i);
  end
end

% Plotting
% Y axis - y, x axis - U(y), For a given time t
plot(U(end,:),y,'blue');
\end{lstlisting}
  \item Q(t) or the flow rate is given.
\end{enumerate}\end{document}
