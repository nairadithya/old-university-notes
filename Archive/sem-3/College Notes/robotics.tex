% Created 2024-09-04 Wed 08:27
% Intended LaTeX compiler: pdflatex
\documentclass[11pt]{report}
\usepackage[utf8]{inputenc}
\usepackage[T1]{fontenc}
\usepackage{graphicx}
\usepackage{longtable}
\usepackage{wrapfig}
\usepackage{rotating}
\usepackage[normalem]{ulem}
\usepackage{amsmath}
\usepackage{amssymb}
\usepackage{capt-of}
\usepackage{hyperref}
\input{preamble}
\author{Adithya Nair}
\date{\today}
\title{Introduction To Robotics}
\hypersetup{
 pdfauthor={Adithya Nair},
 pdftitle={Introduction To Robotics},
 pdfkeywords={},
 pdfsubject={},
 pdfcreator={Emacs 29.4 (Org mode 9.8)}, 
 pdflang={English}}
\begin{document}

\maketitle
\tableofcontents

\part{Robotics}
\label{sec:org17b0dc3}
\href{file:///home/adithya/university-notes/Introduction To Robotics/textbooks/Introduction-to-Robotics-Craig.pdf}{Textbook}
\chapter{Unit 1}
\label{sec:orgf0bc62c}
\section{Introduction}
\label{sec:orgd5bba91}
This course is mainly going to focus on \textbf{Manipulators}. These machines are used to manipulate positions and the state of the objects in an environment. We're going to break down their movements into Dynamics Analysis akin to the work done in Computational Mechanics.
\section{Syllabus}
\label{sec:org7f576fc}
\begin{itemize}
\item Overview Of Robotics
\item Kinematics Of Simple Robotic Systems
\item Dynamics And Control Of Simple Robotic Systems
\end{itemize}
\section{Glossary}
\label{sec:org72c08ba}
\begin{enumerate}
\item Actuator - Does work upon receiving voltage
\item Encoder - Sensor that measures raw angle data.
\end{enumerate}


Software for robotics - \textbf{v-rep}, MATLAB
This software involves making a CAD model, and apply a mathematical model. Unity can also be used in making such models.
\section{Degree Of Freedom}
\label{sec:org30839f3}
The degree of  freedom of a mechanical system is defined as the no. of independent paramets need to completely define its position in space at a given time..

The degree of freedom is defined with respect to a reference frame. If the object is free to rotate and move, it means it has 6 degrees of freedom.
Localization - Finding the position and orientation of an object in 3-dimensional space.

We call a system 'fully actuated' when there are as many actuators as there are degrees of freedom.
$$\text{No. of controlling inputs} < \text{No. of degrees of freedom}$$
Underactuated systems contain lesser actuators than the number of degrees of freedom
$$\text{No. of controlling inputs} = \text{No. of degrees of freedom}$$

Redundant systems contain more actuators than the number of degrees of freedom
$$\text{No. of controlling inputs} >\text{ No. of degrees of freedom}$$
\section{Kinematic Pair}
\label{sec:orgf889bcd}
Linkages are the basic elements of all mechanisms and robots. Links are rigid body member with nodes, and joints are connection between links at nodes. Allows relative motion between links.
\section{Robotic Manipulator}
\label{sec:orgce8c8c9}
\begin{itemize}
\item Why study kinematics and dynamics of robotic manipulator
\begin{itemize}
\item To manipulate an object in space
\item Understand the workspace and limitations of a robotic manipulator
\item Understand and estimate contact force between end-effector and object being manipulated.
\end{itemize}
\end{itemize}
\section{Pose of a rigid body}
\label{sec:orga473341}
A rigid body is completely defined in space by its position and orientation with respect to a reference.

We use the terminology `Inertial reference frame' to mean an observer where Newton's laws of physics apply. We use the terminology 'Inertial reference frame' to mean an observer where Newton's laws of physics apply and the frame itself does not accelerate. Generally the base of the robotic manipulator is treated as the inertial reference frame

We use unit vectors \(\hat{x},\hat{y},\hat{z}\) to describe the basis vectors. For the orientation of the rigid body, since they lie in 3d space, we must define new basis vectors to define the orientation, \(\hat{x'}, \hat{y'}, \hat{z'}\)

$$
\hat{x'} = x'_x \hat{x} + x'_y\hat{y} + x'_z\hat{z}
$$

$$
\hat{y'} = y'_x \hat{x} + y'_y\hat{y} + y'_z\hat{z}
$$

$$
\hat{z'} = z'_x \hat{x} + z'_y\hat{y} + z'_z\hat{z}
$$
\begin{enumerate}
\item When the frame is translationally different from the original frame.
\label{sec:orgcfba093}
$$
\vec{^AP} = \vec{^BP} + \vec{^AP}_{B org}
$$
Where \(^A\vec{P}\) is the position vector of P with respect to A, \(\vec{^BP}\) is the position vector of P with respect to B, and \(\vec{^AP}_{B org}\) is the vector of A to B.
\item When the frame is oriented differently from the original frame.
\label{sec:orgcc032d1}
Generating a rotation matrix \(^B_AR\) to rotate vectors from a frame B to A. Where A and B are reference frames, with B being oriented differently than A.

$$ \begin{bmatrix} \hat{X_b} \cdot \hat{X_a} & \hat{Y_b} \cdot \hat{X_a} & \hat{Z_b} \cdot \hat{X_a} \\ \hat{X_b} \cdot \hat{Y_a} & \hat{Y_b} \cdot \hat{Y_a} & \hat{Z_b} \cdot \hat{Y_a} \\ \hat{X_b} \cdot \hat{Z_a} & \hat{Y_b} \cdot \hat{Z_a} & \hat{Z_b} \cdot \hat{Z_a} \\ \end{bmatrix}$$

We can simply the matrix into 3 column vectors, with the notation \(^AX_B\) which means B with respect to A

$$\begin{bmatrix}\hat{^AX_B} & \hat{^AY_b} & \hat{^AZ_B}\end{bmatrix}$$
\item When the frame is both translationally and oriented different from the original frame
\label{sec:orgecfb49e}
$$ ^A\vec{P} = ^A_B R ^B\vec{P} + ^A\vec{P}_{B org}$$
To simplify the equations, we write.

$$^A\vec{P} = _B^AT ^BP$$
Where T becomes
$$^A_BT = \begin{bmatrix}^A_BR_{3 \times 3} & ^AP_{B org} \\ 0_{1 \times 3} & 1_{1 \times 1}\end{bmatrix}$$

This T is the homogeneous transformation matrix.

The Rotation Matrix belongs to a category of matrices called \(SO(3)\)(Special Orthogonal Matrices)
\end{enumerate}
\section{Denavit Hartenberg Parameters}
\label{sec:orgd994c86}
These parameters are used to assign the same co-ordinate frames while dealing with robotic manipulators to ensure that the entire scientific community are working under the same conventions.
\begin{itemize}
\item Number the link sequentially from \(0\) to \(n\). Giving us \(n+1\) links.
\item Number the joints sequentially from \(1\) to \(n\), Giving us \(n\) joints.
\item Number of co-ordinate frames: \(n+1\)
\item The \(Z_i\) axis is aligned with the \((i+1)^th\) joint axis.
\item \(X_i\) is defined along the common normal between the \(Z_i\) and \(Z_{i-1}\) axis.(Common normal is the line which is perpendicular to both of these axes.)
\item \(Y_i\) is obtained via cross product between \(Z_i\) and \(X_i\) axis.
\item The origin is placed at this point of intersection of \(x_i\), \(y_i\) and \(z_i\)
\item We define \(H_i\)(which is the point of intersection of the common normal of the next axis with the current axis.), and 4 new terms:
\begin{itemize}
\item \(a_i\) - Offset distance between two adjacent joint axes(The distance between \(O_i\) and \(H_{i-1}\))
\item \(d_i\) - Distance between \(H_{i-1}\) and \(O_{i-1}\)
\item \(\alpha_i\) Angle between \(Z_i\) and \(Z_{i-1}\) when viewed from \(X_i\), this is also known as the twist angle.
\item \(\theta_i\) Angle between \(X_i\) and \(X_{i-1}\) when viewed from \(Z_i\),
\end{itemize}
\end{itemize}

For a 3R Planar serial chain manipulator:

\begin{center}
\begin{tabular}{llrrl}
\hline
 & a\textsubscript{i} & d\textsubscript{i} & \(\alpha_{i}\) & \(\theta\)\textsubscript{i}\\
\hline
Joint i = 1 & L\textsubscript{1} & 0 & 0 & \(\theta\)\textsubscript{1}\\
Joint i = 2 & L\textsubscript{2} & 0 & 0 & \(\theta\)\textsubscript{2}\\
Joint i = 3 & L\textsubscript{3} & 0 & 0 & \(\theta\)\textsubscript{3}\\
\hline
\end{tabular}
\end{center}
\section{Rotation Matrix}
\label{sec:orgda06724}
We have three reference frames, with a common origin. With the notation we have setup we can say, \(\{0\},\{1\},\{2\}\) can be defined for a point \(P\)

\begin{align*}
^0P = ^0_1R ^1P \\
^0P = ^0_2R ^2P \\
^1P = ^1_2R ^2P \\
^0_2R = ^0_1R _2^1R
\end{align*}
\section{Derivation Using DH Parameters}
\label{sec:orga85619e}
These 4 parameters can be expressed by,

$$^{i-1}A_i = T(z,d)T(z,\Theta),T(x, \alpha), T(x, a)$$
\section{Skew Symmetric Matrix}
\label{sec:org0467e59}

A matrix such that \(A = -A^T\)

So this means that,

$$S + S^{T}=0$$

$$\vec{a} = a_x \hat{i} + \vec{a_y} \hat{j} \vec{a_z} \hat{k}$$

We define a matrix function,

$$S(\vec{a}) = \begin{bmatrix}
0 & -a_z & a_y \\
a_z & 0 & - a_x \\
-a_y & a_x 0
\end{bmatrix}
$$

So we can say for the basis vectors

$$S(\hat{i}) = \begin{bmatrix}
0 & 0 & 0 \\
0 & 0 & -1 \\
0 & 1 & 0
\end{bmatrix}
$$
$$S(\hat{j}) = \begin{bmatrix}
0 & 0 & 1 \\
0 & 0 & 0 \\
-1 & 0 & 0dt&i0
\end{bmatrix}
$$
$$S(\hat{k}) = \begin{bmatrix}
0 & -1 & 0 \\
1 & 0 & 0  \\
0 & 0 & 0
\end{bmatrix}
$$

Multiplying the matrix function's output with a vector is the same as doing a cross product with those two vectors.
$$S (\vec{a}) P = \vec{a} \times \vec{p} $$
This matrix operation also has linearity in operations.
$$S (\alpha\vec{a} + \beta \vec{b}) P = \alpha S(\vec{a}) +  \beta S(\vec{b}) $$
When multiplied with a rotation matrix.
$$^a_bR S(\vec{a})^a_bR^T = S(^a_bR \vec{a})$$

We know that,
\begin{align*}
R R^T &= I \\
R(\theta) R^t(\theta) &= I \\
\text{Taking the derivative} \\
\frac{d R(\theta)}{d \theta} R^T(\theta) + R(\theta) \frac{d R^{T}(\theta)}{d \theta} &= 0 \\
\frac{d R(\theta)}{d \theta} R^T(\theta) + R(\theta) \left( \frac{d R^{T}(\theta)}{d \theta} \right)^{T}&= 0 \\
\end{align*}

What this means is that the term \(\frac{d R(\theta)}{d \theta} R^T (\theta)\) is a skew-symmetric matrix.

This means that the the term can be written as \(S\),

$$S = \frac{d R(\theta)}{d \theta} R^T (\theta)$$

In other words, S becomes an operator on \(R\) to give us the derivative.

$$S(\hat{k})R(\theta) = \frac{d R(\theta)}{d\theta}$$

Taking both sides with respect to time,

$$ \frac{d R(\theta)}{d \theta} \frac{d \theta}{dt} = \dot{\theta} S(\hat{k})R(\theta) $$
In other words,
$$\dot{R} = S(\dot{\theta}\hat{k})R(\theta)$$

So we have,
$$\dot{R} = S(\vec{\omega})R(\theta)$$

$$S(\vec{\omega}) = \begin{bmatrix}
0 & -\omega_z & \omega_y \\
\omega_z & 0 & -\omega_x \\
\omega_y & \omega_{x} & 0
\end{bmatrix}$$

$$\Omega = \begin{bmatrix}
0 & -\omega_z & \omega_y \\
\omega_z & 0 & -\omega_x \\
\omega_y & \omega_{x} & 0
\end{bmatrix}$$

\begin{align*}
^0P = ^0_1R ^1P \\
^0P = R ^1P \\
^0 \dot{P} = \dot{R} ^1P + R^1 \dot{P} \\
^0 \dot{P} = S(\vec{\omega}) R ^1P \\
^0 \dot{P} = \omega \times R ^1 \vec{p} \\
^0 \dot{P} = \omega \times R ^1 \vec{p} \\
\vec{v} = \vec{\omega} \times \vec{r}
\end{align*}
\chapter{Unit 2 - Kinematics Of Robotic Manipulators}
\label{sec:orgb9c9c11}
\section{Forward And Inverse Kinematics}
\label{sec:orgb32b1d1}
Forward kinematics is the use of the joint space to get to the task space.
Inverse Kinematics is the use of the task space to get to the joint space.

Workspace - The space of all points that the end effector can reach.
\section{Inverse Kinematics Of 2R Planar Structures}
\label{sec:org1ebbfa6}

We solve for \(q_2\) and eliminate \(q_1\)

\begin{align*}
q_2 &= \cos^{-1} \kappa \\
\kappa &= \frac{^0X_{2}^2 + ^0Y_{2}^2 - L_1^2 - L_2^2}}{2L_1L_2} \\
\end{align*}

We have two cases,

Case 1: \(-1<\kappa<1\) We have two distinct and real solutions of \(q_2\)

Case 2: \(|\kappa| = 1\) We have one solution

Case 3: \(|\kappa| < 1\) No solution exists
\section{Forward Kinematics And Inverse Kinematics Of 3R Planar Structures}
\label{sec:org4de1a22}
$$X_e = L_1 \cos {q_1} + L_2 \cos(q_1+ q_2) + L_3 cos(q_1 + q_2 + q_3)$$

For a 3R Planar structure, the way to find the forward kinematics is straightforward.

What we get for the inverse kinematics is with 3 knowns, \(^0X_e, ^0Y_e, \phi\) where \((\phi = q_1 + q_2 + q_3)\) we reduce the problem down to \(^0X_{2}, ^0Y_2\) by finding,

\begin{align*}
^0X_e - L_3 \cos{\phi} &= L_{1} \cos q_1 + L_2 \cos q_{12}\\
^0Y_e - L_3 \sin{\phi} &= L_{1}\sin q_1 + L_2 \sin q_{12}
\end{align*}

Now we know that those terms are nothing but

\begin{align*}
^0X_{2}&= L_{1} \cos q_1 + L_2 \cos q_{12}\\
^0Y_{2} &= L_{1}\sin q_1 + L_2 \sin q_{12}
\end{align*}

Which we can solve the same as the 2R Planar Structure
\section{Jacobian Forward Kinematics}
\label{sec:orga385595}
Given \(\dot{q_{1}}, \dot{q_{2}}\), how do we get \(^0X_{e}, ^0Y_e\)? and similarly how do we get the angular velocities of a machine from the end effector's velocity.

We know that for 2R Planar SCMs,
\begin{align*}
^0X_e &= L_1 c q_1  + L_2 c(q_1 + q_2) \\
^0Y_e &= L_1 s q_1 + L_2 s(q_1+q_2) \\
^0\dot{X_e} &= -\dot{q_1}L_1 s q_1 - L_2s(q_{1}+q_2)\times(\dot{q_1} + \dot{q_2}) \\
^0\dot{Y_e} &= +\dot{q_1}L_1 c q_1 + L_2c(q_{1}+q_2)\times(\dot{q_1} + \dot{q_2})
\end{align*}

We have,
$$\dot{X}_{2\times1} = J_{2 \times2} \times \dot{q}_{2 \times1}$$


We now have this for a specific case, when we generalize,
$$ \dot{X} =  \begin{bmatrix} ^0_nv \\ ^0_n\omega \\ \end{bmatrix} = J \dot{q}$$

So now we can write the term

$$^0_{n}\omega = \sum_{i=1}^n \dot{\theta}_i z$$
\section{Lab Questions}
\label{sec:org08918d5}
\begin{enumerate}
\item Consider the 3R Planar SCM with \(L_1 = 2m\), \(L_2 = 3m, L_3 = 4m\)
\end{enumerate}
Forward Kinematics,
Find the co-ordinates of the end-effector in the base frame:
\begin{enumerate}
\item For \(q = [q_1 q_2 q_3]^T= [30\deg 45 \deg 60 \deg]^{T}\)
\item For \(q = [q_1 q_2 q_3]^T= [270\deg 75 \deg 10 \deg]^{T}\)
\end{enumerate}
Inverse kinematics,
Find the joint space variables,
\begin{enumerate}
\item For \(\phi = q_1 + q_2 + q_3 = 0\) and \([^0X_e, ^0Y_e] = [2,4]\)
\item For \(\phi = q_1 + q_2 + q_3 = 45\) and \([^0X_e, ^0Y_e] = [5,5]\)
\end{enumerate}

\begin{enumerate}
\item 
\end{enumerate}
\part{Practice Sheets}
\label{sec:org7c56995}
\href{file:///home/adithya/university-notes/Introduction To Robotics/assignments/23AID202\_ITR\_PracticeSheet01.pdf}{Practice Sheet 1}
The notations used in our lectures are similar with this book.

\begin{itemize}
\item For Rotation Matrices, Pose of a rigid body, Euler Angles, Homogeneous Transformation ---> read Chapter 2
\item For DH Parameters and Coordinate Convention ---> read Chapter 3, Sections 3.1 to 3.4
\end{itemize}
\end{document}
