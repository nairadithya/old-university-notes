% Created 2024-11-07 Thu 11:09
% Intended LaTeX compiler: pdflatex
\documentclass[11pt]{report}
\usepackage[utf8]{inputenc}
\usepackage[T1]{fontenc}
\usepackage{graphicx}
\usepackage{longtable}
\usepackage{wrapfig}
\usepackage{rotating}
\usepackage[normalem]{ulem}
\usepackage{amsmath}
\usepackage{amssymb}
\usepackage{capt-of}
\usepackage{hyperref}
%%%%%%%%%%%%%%%%%%%%%%%%%%%%%%%%%%%%%%%%%%%%%%%%%%%%%%%%%%%%%%%%%%%%%%%%%%%%%%%
%                                Basic Packages                               %
%%%%%%%%%%%%%%%%%%%%%%%%%%%%%%%%%%%%%%%%%%%%%%%%%%%%%%%%%%%%%%%%%%%%%%%%%%%%%%%

% Gives us multiple colors.
\usepackage[dvipsnames,pdftex]{xcolor}
\usepackage[tmargin=2cm,rmargin=1in,lmargin=1in,margin=0.85in,bmargin=2cm,footskip=.2in]{geometry}

% Lets us style link colors.
\usepackage{hyperref}
% Lets us import images and graphics.
\usepackage{graphicx}
% Lets us use figures in floating environments.
\usepackage{float}
% Lets us create multiple columns.
\usepackage{multicol}
% Gives us better math syntax.
\usepackage{amsmath,amsfonts,mathtools,amsthm,amssymb}
% Lets us strikethrough text.
\usepackage{cancel}
% Lets us edit the caption of a figure.
\usepackage{caption}
% Lets us import pdf directly in our tex code.
\usepackage{pdfpages}
% Lets us do algorithm stuff.
\usepackage[ruled,vlined,linesnumbered]{algorithm2e}
% Use a smiley face for our qed symbol.
\usepackage{tikzsymbols}
\renewcommand\qedsymbol{$\Laughey$}

\def\class{article}


%%%%%%%%%%%%%%%%%%%%%%%%%%%%%%%%%%%%%%%%%%%%%%%%%%%%%%%%%%%%%%%%%%%%%%%%%%%%%%%
%                                Basic Settings                               %
%%%%%%%%%%%%%%%%%%%%%%%%%%%%%%%%%%%%%%%%%%%%%%%%%%%%%%%%%%%%%%%%%%%%%%%%%%%%%%%

%%%%%%%%%%%%%
%  Symbols  %
%%%%%%%%%%%%%

\let\implies\Rightarrow
\let\impliedby\Leftarrow
\let\iff\Leftrightarrow
\let\epsilon\varepsilon

%%%%%%%%%%%%
%  Tables  %
%%%%%%%%%%%%

\setlength{\tabcolsep}{5pt}
\renewcommand\arraystretch{1.5}

%%%%%%%%%%%%%%
%  SI Unitx  %
%%%%%%%%%%%%%%

\usepackage{siunitx}
\sisetup{locale = FR}

%%%%%%%%%%
%  TikZ  %
%%%%%%%%%%

\usepackage[framemethod=TikZ]{mdframed}
\usepackage{tikz}
\usepackage{tikz-cd}
\usepackage{tikzsymbols}

\usetikzlibrary{intersections, angles, quotes, calc, positioning}
\usetikzlibrary{arrows.meta}

\tikzset{
  force/.style={thick, {Circle[length=2pt]}-stealth, shorten <=-1pt}
}

%%%%%%%%%%%%%%%
%  PGF Plots  %
%%%%%%%%%%%%%%%

\usepackage{pgfplots}
\pgfplotsset{compat=1.13}

%%%%%%%%%%%%%%%%%%%%%%%
%  Center Title Page  %
%%%%%%%%%%%%%%%%%%%%%%%

\usepackage{titling}
\renewcommand\maketitlehooka{\null\mbox{}\vfill}
\renewcommand\maketitlehookd{\vfill\null}

%%%%%%%%%%%%%%%%%%%%%%%%%%%%%%%%%%%%%%%%%%%%%%%%%%%%%%%
%  Create a grey background in the middle of the PDF  %
%%%%%%%%%%%%%%%%%%%%%%%%%%%%%%%%%%%%%%%%%%%%%%%%%%%%%%%

\usepackage{eso-pic}
\newcommand\definegraybackground{
  \definecolor{reallylightgray}{HTML}{FAFAFA}
  \AddToShipoutPicture{
    \ifthenelse{\isodd{\thepage}}{
      \AtPageLowerLeft{
        \put(\LenToUnit{\dimexpr\paperwidth-222pt},0){
          \color{reallylightgray}\rule{222pt}{297mm}
        }
      }
    }
    {
      \AtPageLowerLeft{
        \color{reallylightgray}\rule{222pt}{297mm}
      }
    }
  }
}

%%%%%%%%%%%%%%%%%%%%%%%%
%  Modify Links Color  %
%%%%%%%%%%%%%%%%%%%%%%%%

\hypersetup{
  % Enable highlighting links.
  colorlinks,
  % Change the color of links to blue.
  linkcolor=blue,
  % Change the color of citations to black.
  citecolor={black},
  % Change the color of url's to blue with some black.
  urlcolor={blue!80!black}
}

%%%%%%%%%%%%%%%%%%
% Fix WrapFigure %
%%%%%%%%%%%%%%%%%%

\newcommand{\wrapfill}{\par\ifnum\value{WF@wrappedlines}>0
    \parskip=0pt
    \addtocounter{WF@wrappedlines}{-1}%
    \null\vspace{\arabic{WF@wrappedlines}\baselineskip}%
    \WFclear
\fi}

%%%%%%%%%%%%%%%%%
% Multi Columns %
%%%%%%%%%%%%%%%%%

\let\multicolmulticols\multicols
\let\endmulticolmulticols\endmulticols

\RenewDocumentEnvironment{multicols}{mO{}}
{%
  \ifnum#1=1
    #2%
  \else % More than 1 column
    \multicolmulticols{#1}[#2]
  \fi
}
{%
  \ifnum#1=1
\else % More than 1 column
  \endmulticolmulticols
\fi
}

\newlength{\thickarrayrulewidth}
\setlength{\thickarrayrulewidth}{5\arrayrulewidth}


%%%%%%%%%%%%%%%%%%%%%%%%%%%%%%%%%%%%%%%%%%%%%%%%%%%%%%%%%%%%%%%%%%%%%%%%%%%%%%%
%                           School Specific Commands                          %
%%%%%%%%%%%%%%%%%%%%%%%%%%%%%%%%%%%%%%%%%%%%%%%%%%%%%%%%%%%%%%%%%%%%%%%%%%%%%%%

%%%%%%%%%%%%%%%%%%%%%%%%%%%
%  Initiate New Counters  %
%%%%%%%%%%%%%%%%%%%%%%%%%%%

\newcounter{lecturecounter}

%%%%%%%%%%%%%%%%%%%%%%%%%%
%  Helpful New Commands  %
%%%%%%%%%%%%%%%%%%%%%%%%%%

\makeatletter

\newcommand\resetcounters{
  % Reset the counters for subsection, subsubsection and the definition
  % all the custom environments.
  \setcounter{subsection}{0}
  \setcounter{subsubsection}{0}
  \setcounter{paragraph}{0}
  \setcounter{subparagraph}{0}
  \setcounter{theorem}{0}
  \setcounter{claim}{0}
  \setcounter{corollary}{0}
  \setcounter{lemma}{0}
  \setcounter{exercise}{0}

  \@ifclasswith\class{nocolor}{
    \setcounter{definition}{0}
  }{}
}

%%%%%%%%%%%%%%%%%%%%%
%  Lecture Command  %
%%%%%%%%%%%%%%%%%%%%%

\usepackage{xifthen}

% EXAMPLE:
% 1. \lesson{Oct 17 2022 Mon (08:46:48)}{Lecture Title}
% 2. \lesson[4]{Oct 17 2022 Mon (08:46:48)}{Lecture Title}
% 3. \lesson{Oct 17 2022 Mon (08:46:48)}{}
% 4. \lesson[4]{Oct 17 2022 Mon (08:46:48)}{}
% Parameters:
% 1. (Optional) Lesson number.
% 2. Time and date of lecture.
% 3. Lecture Title.
\def\@lesson{}
\newcommand\lesson[3][\arabic{lecturecounter}]{
  % Add 1 to the lecture counter.
  \addtocounter{lecturecounter}{1}

  % Set the section number to the lecture counter.
  \setcounter{section}{#1}
  \renewcommand\thesubsection{#1.\arabic{subsection}}

  % Reset the counters.
  \resetcounters

  % Check if user passed the lecture title or not.
  \ifthenelse{\isempty{#3}}{
    \def\@lesson{Lecture \arabic{lecturecounter}}
  }{
    \def\@lesson{Lecture \arabic{lecturecounter}: #3}
  }

  % Display the information like the following:
  %                                                  Oct 17 2022 Mon (08:49:10)
  % ---------------------------------------------------------------------------
  % Lecture 1: Lecture Title
  \hfill\small{#2}
  \hrule
  \vspace*{-0.3cm}
  \section*{\@lesson}
  \addcontentsline{toc}{section}{\@lesson}
}

%%%%%%%%%%%%%%%%%%%%
%  Import Figures  %
%%%%%%%%%%%%%%%%%%%%

\usepackage{import}
\pdfminorversion=7

% EXAMPLE:
% 1. \incfig{limit-graph}
% 2. \incfig[0.4]{limit-graph}
% Parameters:
% 1. The figure name. It should be located in figures/NAME.tex_pdf.
% 2. (Optional) The width of the figure. Example: 0.5, 0.35.
\newcommand\incfig[2][1]{%
  \def\svgwidth{#1\columnwidth}
  \import{./figures/}{#2.pdf_tex}
}

\begingroup\expandafter\expandafter\expandafter\endgroup
\expandafter\ifx\csname pdfsuppresswarningpagegroup\endcsname\relax
\else
  \pdfsuppresswarningpagegroup=1\relax
\fi

%%%%%%%%%%%%%%%%%
% Fancy Headers %
%%%%%%%%%%%%%%%%%

\usepackage{fancyhdr}

% Force a new page.

\newcommand\forcenewpage{\clearpage\mbox{~}\clearpage\newpage}
\newcommand\createintro{
  \pagestyle{fancy}
  \fancyhead{}
  \fancyhead[C]{23PHY114}
  \fancyfoot[L]{Adithya Nair}
  \fancyfoot[R]{AID23002}
  % Create a new page.
}
  \newpage
\makeatother

%%%%%%%%%%%%%%%%%%%%%%%%%%%%%%%%%%%%%%%%%%%%%%%%%%%%%%%%%%%%%%%%%%%%%%%%%%%%%%%
%                               Custom Commands                               %
%%%%%%%%%%%%%%%%%%%%%%%%%%%%%%%%%%%%%%%%%%%%%%%%%%%%%%%%%%%%%%%%%%%%%%%%%%%%%%%

%%%%%%%%%%%%
%  Circle  %
%%%%%%%%%%%%

\newcommand*\circled[1]{\tikz[baseline=(char.base)]{
  \node[shape=circle,draw,inner sep=1pt] (char) {#1};}
}

%%%%%%%%%%%%%%%%%%%
%  Todo Commands  %
%%%%%%%%%%%%%%%%%%%

\usepackage{xargs}
\usepackage[colorinlistoftodos]{todonotes}

\makeatletter

\@ifclasswith\class{working}{
  \newcommandx\unsure[2][1=]{\todo[linecolor=red,backgroundcolor=red!25,bordercolor=red,#1]{#2}}
  \newcommandx\change[2][1=]{\todo[linecolor=blue,backgroundcolor=blue!25,bordercolor=blue,#1]{#2}}
  \newcommandx\info[2][1=]{\todo[linecolor=OliveGreen,backgroundcolor=OliveGreen!25,bordercolor=OliveGreen,#1]{#2}}
  \newcommandx\improvement[2][1=]{\todo[linecolor=Plum,backgroundcolor=Plum!25,bordercolor=Plum,#1]{#2}}

  \newcommand\listnotes{
    \newpage
    \listoftodos[Notes]
  }
}{
  \newcommandx\unsure[2][1=]{}
  \newcommandx\change[2][1=]{}
  \newcommandx\info[2][1=]{}
  \newcommandx\improvement[2][1=]{}

  \newcommand\listnotes{}
}

\makeatother

%%%%%%%%%%%%%
%  Correct  %
%%%%%%%%%%%%%

% EXAMPLE:
% 1. \correct{INCORRECT}{CORRECT}
% Parameters:
% 1. The incorrect statement.
% 2. The correct statement.
\definecolor{correct}{HTML}{009900}
\newcommand\correct[2]{{\color{red}{#1 }}\ensuremath{\to}{\color{correct}{ #2}}}


%%%%%%%%%%%%%%%%%%%%%%%%%%%%%%%%%%%%%%%%%%%%%%%%%%%%%%%%%%%%%%%%%%%%%%%%%%%%%%%
%                                 Environments                                %
%%%%%%%%%%%%%%%%%%%%%%%%%%%%%%%%%%%%%%%%%%%%%%%%%%%%%%%%%%%%%%%%%%%%%%%%%%%%%%%

\usepackage{varwidth}
\usepackage{thmtools}
\usepackage[most,many,breakable]{tcolorbox}

\tcbuselibrary{theorems,skins,hooks}
\usetikzlibrary{arrows,calc,shadows.blur}

%%%%%%%%%%%%%%%%%%%
%  Define Colors  %
%%%%%%%%%%%%%%%%%%%

\definecolor{myblue}{RGB}{45, 111, 177}
\definecolor{mygreen}{RGB}{56, 140, 70}
\definecolor{myred}{RGB}{199, 68, 64}
\definecolor{mypurple}{RGB}{197, 92, 212}

\definecolor{definition}{HTML}{228b22}
\definecolor{theorem}{HTML}{00007B}
\definecolor{example}{HTML}{2A7F7F}
\definecolor{definition}{HTML}{228b22}
\definecolor{prop}{HTML}{191971}
\definecolor{lemma}{HTML}{983b0f}
\definecolor{exercise}{HTML}{88D6D1}

\colorlet{definition}{mygreen!85!black}
\colorlet{claim}{mygreen!85!black}
\colorlet{corollary}{mypurple!85!black}
\colorlet{proof}{theorem}

%%%%%%%%%%%%%%%%%%%%%%%%%%%%%%%%%%%%%%%%%%%%%%%%%%%%%%%%%
%  Create Environments Styles Based on Given Parameter  %
%%%%%%%%%%%%%%%%%%%%%%%%%%%%%%%%%%%%%%%%%%%%%%%%%%%%%%%%%

\mdfsetup{skipabove=1em,skipbelow=0em}

%%%%%%%%%%%%%%%%%%%%%%
%  Helpful Commands  %
%%%%%%%%%%%%%%%%%%%%%%

% EXAMPLE:
% 1. \createnewtheoremstyle{thmdefinitionbox}{}{}
% 2. \createnewtheoremstyle{thmtheorembox}{}{}
% 3. \createnewtheoremstyle{thmproofbox}{qed=\qedsymbol}{
%       rightline=false, topline=false, bottomline=false
%    }
% Parameters:
% 1. Theorem name.
% 2. Any extra parameters to pass directly to declaretheoremstyle.
% 3. Any extra parameters to pass directly to mdframed.
\newcommand\createnewtheoremstyle[3]{
  \declaretheoremstyle[
  headfont=\bfseries\sffamily, bodyfont=\normalfont, #2,
  mdframed={
    #3,
  },
  ]{#1}
}

% EXAMPLE:
% 1. \createnewcoloredtheoremstyle{thmdefinitionbox}{definition}{}{}
% 2. \createnewcoloredtheoremstyle{thmexamplebox}{example}{}{
%       rightline=true, leftline=true, topline=true, bottomline=true
%     }
% 3. \createnewcoloredtheoremstyle{thmproofbox}{proof}{qed=\qedsymbol}{backgroundcolor=white}
% Parameters:
% 1. Theorem name.
% 2. Color of theorem.
% 3. Any extra parameters to pass directly to declaretheoremstyle.
% 4. Any extra parameters to pass directly to mdframed.
\newcommand\createnewcoloredtheoremstyle[4]{
  \declaretheoremstyle[
  headfont=\bfseries\sffamily\color{#2}, bodyfont=\normalfont, #3,
  mdframed={
    linewidth=2pt,
    rightline=false, leftline=true, topline=false, bottomline=false,
    linecolor=#2, backgroundcolor=#2!5, #4,
  },
  ]{#1}
}

%%%%%%%%%%%%%%%%%%%%%%%%%%%%%%%%%%%
%  Create the Environment Styles  %
%%%%%%%%%%%%%%%%%%%%%%%%%%%%%%%%%%%

\makeatletter
\@ifclasswith\class{nocolor}{
  % Environments without color.

  \createnewtheoremstyle{thmdefinitionbox}{}{}
  \createnewtheoremstyle{thmtheorembox}{}{}
  \createnewtheoremstyle{thmexamplebox}{}{}
  \createnewtheoremstyle{thmclaimbox}{}{}
  \createnewtheoremstyle{thmcorollarybox}{}{}
  \createnewtheoremstyle{thmpropbox}{}{}
  \createnewtheoremstyle{thmlemmabox}{}{}
  \createnewtheoremstyle{thmexercisebox}{}{}
  \createnewtheoremstyle{thmdefinitionbox}{}{}
  \createnewtheoremstyle{thmquestionbox}{}{}
  \createnewtheoremstyle{thmsolutionbox}{}{}

  \createnewtheoremstyle{thmproofbox}{qed=\qedsymbol}{}
  \createnewtheoremstyle{thmexplanationbox}{}{}
}{
  % Environments with color.

  \createnewcoloredtheoremstyle{thmdefinitionbox}{definition}{}{}
  \createnewcoloredtheoremstyle{thmtheorembox}{theorem}{}{}
  \createnewcoloredtheoremstyle{thmexamplebox}{example}{}{
    rightline=true, leftline=true, topline=true, bottomline=true
  }
  \createnewcoloredtheoremstyle{thmclaimbox}{claim}{}{}
  \createnewcoloredtheoremstyle{thmcorollarybox}{corollary}{}{}
  \createnewcoloredtheoremstyle{thmpropbox}{prop}{}{}
  \createnewcoloredtheoremstyle{thmlemmabox}{lemma}{}{}
  \createnewcoloredtheoremstyle{thmexercisebox}{exercise}{}{}

  \createnewcoloredtheoremstyle{thmproofbox}{proof}{qed=\qedsymbol}{backgroundcolor=white}
  \createnewcoloredtheoremstyle{thmexplanationbox}{example}{qed=\qedsymbol}{backgroundcolor=white}
}
\makeatother

%%%%%%%%%%%%%%%%%%%%%%%%%%%%%
%  Create the Environments  %
%%%%%%%%%%%%%%%%%%%%%%%%%%%%%

\declaretheorem[numberwithin=section, style=thmtheorembox,     name=Theorem]{theorem}
\declaretheorem[numbered=no,          style=thmexamplebox,     name=Example]{example}
\declaretheorem[numberwithin=section, style=thmclaimbox,       name=Claim]{claim}
\declaretheorem[numberwithin=section, style=thmcorollarybox,   name=Corollary]{corollary}
\declaretheorem[numberwithin=section, style=thmpropbox,        name=Proposition]{prop}
\declaretheorem[numberwithin=section, style=thmlemmabox,       name=Lemma]{lemma}
\declaretheorem[numberwithin=section, style=thmexercisebox,    name=Exercise]{exercise}
\declaretheorem[numbered=no,          style=thmproofbox,       name=Proof]{replacementproof}
\declaretheorem[numbered=no,          style=thmexplanationbox, name=Proof]{expl}

\makeatletter
\@ifclasswith\class{nocolor}{
  % Environments without color.

  \newtheorem*{note}{Note}

  \declaretheorem[numberwithin=section, style=thmdefinitionbox, name=Definition]{definition}
  \declaretheorem[numberwithin=section, style=thmquestionbox,   name=Question]{question}
  \declaretheorem[numberwithin=section, style=thmsolutionbox,   name=Solution]{solution}
}{
  % Environments with color.

  \newtcbtheorem[number within=section]{Definition}{Definition}{
    enhanced,
    before skip=2mm,
    after skip=2mm,
    colback=red!5,
    colframe=red!80!black,
    colbacktitle=red!75!black,
    boxrule=0.5mm,
    attach boxed title to top left={
      xshift=1cm,
      yshift*=1mm-\tcboxedtitleheight
    },
    varwidth boxed title*=-3cm,
    boxed title style={
      interior engine=empty,
      frame code={
        \path[fill=tcbcolback]
        ([yshift=-1mm,xshift=-1mm]frame.north west)
        arc[start angle=0,end angle=180,radius=1mm]
        ([yshift=-1mm,xshift=1mm]frame.north east)
        arc[start angle=180,end angle=0,radius=1mm];
        \path[left color=tcbcolback!60!black,right color=tcbcolback!60!black,
        middle color=tcbcolback!80!black]
        ([xshift=-2mm]frame.north west) -- ([xshift=2mm]frame.north east)
        [rounded corners=1mm]-- ([xshift=1mm,yshift=-1mm]frame.north east)
        -- (frame.south east) -- (frame.south west)
        -- ([xshift=-1mm,yshift=-1mm]frame.north west)
        [sharp corners]-- cycle;
      },
    },
    fonttitle=\bfseries,
    title={#2},
    #1
  }{def}

  \NewDocumentEnvironment{definition}{O{}O{}}
    {\begin{Definition}{#1}{#2}}{\end{Definition}}

  \newtcolorbox{note}[1][]{%
    enhanced jigsaw,
    colback=gray!20!white,%
    colframe=gray!80!black,
    size=small,
    boxrule=1pt,
    title=\textbf{Note:-},
    halign title=flush center,
    coltitle=black,
    breakable,
    drop shadow=black!50!white,
    attach boxed title to top left={xshift=1cm,yshift=-\tcboxedtitleheight/2,yshifttext=-\tcboxedtitleheight/2},
    minipage boxed title=1.5cm,
    boxed title style={%
      colback=white,
      size=fbox,
      boxrule=1pt,
      boxsep=2pt,
      underlay={%
        \coordinate (dotA) at ($(interior.west) + (-0.5pt,0)$);
        \coordinate (dotB) at ($(interior.east) + (0.5pt,0)$);
        \begin{scope}
          \clip (interior.north west) rectangle ([xshift=3ex]interior.east);
          \filldraw [white, blur shadow={shadow opacity=60, shadow yshift=-.75ex}, rounded corners=2pt] (interior.north west) rectangle (interior.south east);
        \end{scope}
        \begin{scope}[gray!80!black]
          \fill (dotA) circle (2pt);
          \fill (dotB) circle (2pt);
        \end{scope}
      },
    },
    #1,
  }

  \newtcbtheorem{Question}{Question}{enhanced,
    breakable,
    colback=white,
    colframe=myblue!80!black,
    attach boxed title to top left={yshift*=-\tcboxedtitleheight},
    fonttitle=\bfseries,
    title=\textbf{Question:-},
    boxed title size=title,
    boxed title style={%
      sharp corners,
      rounded corners=northwest,
      colback=tcbcolframe,
      boxrule=0pt,
    },
    underlay boxed title={%
      \path[fill=tcbcolframe] (title.south west)--(title.south east)
      to[out=0, in=180] ([xshift=5mm]title.east)--
      (title.center-|frame.east)
      [rounded corners=\kvtcb@arc] |-
      (frame.north) -| cycle;
    },
    #1
  }{def}

  \NewDocumentEnvironment{question}{O{}O{}}
  {\begin{Question}{#1}{#2}}{\end{Question}}

  \newtcolorbox{Solution}{enhanced,
    breakable,
    colback=white,
    colframe=mygreen!80!black,
    attach boxed title to top left={yshift*=-\tcboxedtitleheight},
    title=\textbf{Solution:-},
    boxed title size=title,
    boxed title style={%
      sharp corners,
      rounded corners=northwest,
      colback=tcbcolframe,
      boxrule=0pt,
    },
    underlay boxed title={%
      \path[fill=tcbcolframe] (title.south west)--(title.south east)
      to[out=0, in=180] ([xshift=5mm]title.east)--
      (title.center-|frame.east)
      [rounded corners=\kvtcb@arc] |-
      (frame.north) -| cycle;
    },
  }

  \NewDocumentEnvironment{solution}{O{}O{}}
  {\vspace{-10pt}\begin{Solution}{#1}{#2}}{\end{Solution}}
}
\makeatother

%%%%%%%%%%%%%%%%%%%%%%%%%%%%
%  Edit Proof Environment  %
%%%%%%%%%%%%%%%%%%%%%%%%%%%%

\renewenvironment{proof}[1][\proofname]{\vspace{-10pt}\begin{replacementproof}}{\end{replacementproof}}
\newenvironment{explanation}[1][\proofname]{\vspace{-10pt}\begin{expl}}{\end{expl}}

\theoremstyle{definition}

\newtheorem*{notation}{Notation}
\newtheorem*{previouslyseen}{As previously seen}
\newtheorem*{problem}{Problem}
\newtheorem*{observe}{Observe}
\newtheorem*{property}{Property}
\newtheorem*{intuition}{Intuition}

%%%%%%%%%%%%%%%%%%%%%%%%%%%%%%%%%%%%%%%%%%%%%%%%%%%%%%%%%%%%%%%
%                 Code Highlighting                           %
%%%%%%%%%%%%%%%%%%%%%%%%%%%%%%%%%%%%%%%%%%%%%%%%%%%%%%%%%%%%%%%
\usepackage{listings}
\lstset{
language=Octave,
backgroundcolor=\color{white},   % choose the background color; you must add \usepackage{color} or \usepackage{xcolor}
basicstyle=\footnotesize\ttfamily,        % the size of the fonts that are used for the code
breakatwhitespace=false,         % sets if automatic breaks should only happen at whitespace
breaklines=true,                 % sets automatic line breaking
captionpos=b,                    % sets the caption-position to bottom
commentstyle=\color{gray},    % comment style
%escapeinside={\%*}{*)},          % if you want to add LaTeX within your code
extendedchars=true,            % lets you use non-ASCII characters; for 8-bits encodings only, does not work with UTF-8
frame=single,                    % adds a frame around the code
% frameround=fttt,
keepspaces=true,                 % keeps spaces in text, useful for keeping indentation of code (possibly needs columns=flexible)
columns=flexible,
classoffset=0,
keywordstyle=\color{RoyalBlue},       % keyword style
deletekeywords={function,endfunction, if,endif},
classoffset=1,
morekeywords={function,endfunction, if,endif},
keywordstyle=\bf\color{Red},       % keyword style
classoffset=2,
morekeywords={persistent},            % if you want to add more keywords to the set
keywordstyle=\bf\color{ForestGreen},       % keyword style
classoffset=0,
literate=
{/}{{{\color{Mahogany}/}}}1
{*}{{{\color{Mahogany}*}}}1
{.*}{{{\color{Mahogany}.*}}}2
{+}{{{\color{Mahogany}+{}}}}1
{=}{{{\bf\color{Mahogany}=}}}1
{-}{{{\color{Mahogany}-}}}1
{[}{{{\bf\color{RedOrange}[}}}1
{]}{{{\bf\color{RedOrange}]}}}1
{ç}{{\c{c}}}1 % Cedilha
{á}{{\'{a}}}1 % Acentos agudos
{é}{{\'{e}}}1
{í}{{\'{i}}}1
{ó}{{\'{o}}}1
{ú}{{\'{u}}}1
{â}{{\^{a}}}1 % Acentos circunflexos
{ê}{{\^{e}}}1
{î}{{\^{i}}}1
{ô}{{\^{o}}}1
{û}{{\^{u}}}1
{à}{{\`{a}}}1 % Acentos graves
{è}{{\`{e}}}1
{ì}{{\`{i}}}1
{ò}{{\`{o}}}1
{ù}{{\`{u}}}1
{ã}{{\~{a}}}1 % Tils
{ẽ}{{\~{e}}}1
{ĩ}{{\~{i}}}1
{õ}{{\~{o}}}1
{ũ}{{\~{u}}}1,
numbers=left,                    % where to put the line-numbers; possible values are (none, left, right)
numbersep=6pt,                   % how far the line-numbers are from the code
numberstyle=\tiny\color{gray}, % the style that is used for the line-numbers
rulecolor=\color{black},         % if not set, the frame-color may be changed on line-breaks within not-black text (e.g. comments (green here))
showspaces=false,                % show spaces everywhere adding particular underscores; it overrides 'showstringspaces'
showstringspaces=false,          % underline spaces within strings only
showtabs=false,                  % show tabs within strings adding particular underscores
stepnumber=1,                    % the step between two line-numbers. If it's 1, each line will be numbered
stringstyle=\color{purple},     % string literal style
tabsize=2,                       % sets default tabsize to 2 spaces
}


\author{Adithya Nair}
\date{\today}
\title{Introduction To Robotics}
\hypersetup{
 pdfauthor={Adithya Nair},
 pdftitle={Introduction To Robotics},
 pdfkeywords={},
 pdfsubject={},
 pdfcreator={Emacs 29.4 (Org mode 9.7.11)}, 
 pdflang={English}}
\begin{document}

\maketitle
\tableofcontents

\part{Robotics}
\label{sec:org97aac8e}
\href{file:///home/adithya/university-notes/Introduction To Robotics/textbooks/Introduction-to-Robotics-Craig.pdf}{Textbook}
\chapter{Unit 1}
\label{sec:orgf437c89}
\section{Introduction}
\label{sec:org15d2aa9}
This course is mainly going to focus on \textbf{Manipulators}. These machines are used to manipulate positions and the state of the objects in an environment. We're going to break down their movements into Dynamics Analysis akin to the work done in Computational Mechanics.
\section{Syllabus}
\label{sec:org0348181}
\begin{itemize}
\item Overview Of Robotics
\item Kinematics Of Simple Robotic Systems
\item Dynamics And Control Of Simple Robotic Systems
\end{itemize}
\section{Glossary}
\label{sec:org48c43ea}
\begin{enumerate}
\item Actuator - Does work upon receiving voltage
\item Encoder - Sensor that measures raw angle data.
\end{enumerate}

Software for robotics - \textbf{v-rep}, MATLAB
This software involves making a CAD model, and apply a mathematical model. Unity can also be used in making such models.
\section{Degree Of Freedom}
\label{sec:orgce5a525}
The degree of  freedom of a mechanical system is defined as the no. of independent paramets need to completely define its position in space at a given time..

The degree of freedom is defined with respect to a reference frame. If the object is free to rotate and move, it means it has 6 degrees of freedom.
Localization - Finding the position and orientation of an object in 3-dimensional space.

We call a system 'fully actuated' when there are as many actuators as there are degrees of freedom.
$$\text{No. of controlling inputs} < \text{No. of degrees of freedom}$$
Underactuated systems contain lesser actuators than the number of degrees of freedom
$$\text{No. of controlling inputs} = \text{No. of degrees of freedom}$$

Redundant systems contain more actuators than the number of degrees of freedom
$$\text{No. of controlling inputs} >\text{ No. of degrees of freedom}$$
\section{Kinematic Pair}
\label{sec:org8d79d36}
Linkages are the basic elements of all mechanisms and robots. Links are rigid body member with nodes, and joints are connection between links at nodes. Allows relative motion between links.
\section{Robotic Manipulator}
\label{sec:org4761a46}
\begin{itemize}
\item Why study kinematics and dynamics of robotic manipulator
\begin{itemize}
\item To manipulate an object in space
\item Understand the workspace and limitations of a robotic manipulator
\item Understand and estimate contact force between end-effector and object being manipulated.
\end{itemize}
\end{itemize}
\section{Pose of a rigid body}
\label{sec:orgcec442a}
A rigid body is completely defined in space by its position and orientation with respect to a reference.

We use the terminology `Inertial reference frame' to mean an observer where Newton's laws of physics apply. We use the terminology 'Inertial reference frame' to mean an observer where Newton's laws of physics apply and the frame itself does not accelerate. Generally the base of the robotic manipulator is treated as the inertial reference frame

We use unit vectors \(\hat{x},\hat{y},\hat{z}\) to describe the basis vectors. For the orientation of the rigid body, since they lie in 3d space, we must define new basis vectors to define the orientation, \(\hat{x'}, \hat{y'}, \hat{z'}\)

$$
\hat{x'} = x'_x \hat{x} + x'_y\hat{y} + x'_z\hat{z}
$$

$$
\hat{y'} = y'_x \hat{x} + y'_y\hat{y} + y'_z\hat{z}
$$

$$
\hat{z'} = z'_x \hat{x} + z'_y\hat{y} + z'_z\hat{z}
$$
\begin{enumerate}
\item When the frame is translationally different from the original frame.
\label{sec:orgf110128}
$$
\vec{^AP} = \vec{^BP} + \vec{^AP}_{B org}
$$
Where \(^A\vec{P}\) is the position vector of P with respect to A, \(\vec{^BP}\) is the position vector of P with respect to B, and \(\vec{^AP}_{B org}\) is the vector of A to B.
\item When the frame is oriented differently from the original frame.
\label{sec:orgfeb3d47}
Generating a rotation matrix \(^B_AR\) to rotate vectors from a frame B to A. Where A and B are reference frames, with B being oriented differently than A.

$$ \begin{bmatrix} \hat{X_b} \cdot \hat{X_a} & \hat{Y_b} \cdot \hat{X_a} & \hat{Z_b} \cdot \hat{X_a} \\ \hat{X_b} \cdot \hat{Y_a} & \hat{Y_b} \cdot \hat{Y_a} & \hat{Z_b} \cdot \hat{Y_a} \\ \hat{X_b} \cdot \hat{Z_a} & \hat{Y_b} \cdot \hat{Z_a} & \hat{Z_b} \cdot \hat{Z_a} \\ \end{bmatrix}$$

We can simply the matrix into 3 column vectors, with the notation \(^AX_B\) which means B with respect to A

$$\begin{bmatrix}\hat{^AX_B} & \hat{^AY_b} & \hat{^AZ_B}\end{bmatrix}$$
\item When the frame is both translationally and oriented different from the original frame
\label{sec:org7c3de0c}
$$ ^A\vec{P} = ^A_B R ^B\vec{P} + ^A\vec{P}_{B org}$$
To simplify the equations, we write.

$$^A\vec{P} = _B^AT ^BP$$
Where T becomes
$$^A_BT = \begin{bmatrix}^A_BR_{3 \times 3} & ^AP_{B org} \\ 0_{1 \times 3} & 1_{1 \times 1}\end{bmatrix}$$

This T is the homogeneous transformation matrix.

The Rotation Matrix belongs to a category of matrices called \(SO(3)\)(Special Orthogonal Matrices)
\end{enumerate}
\section{Denavit Hartenberg Parameters}
\label{sec:org21b16e0}
These parameters are used to assign the same co-ordinate frames while dealing with robotic manipulators to ensure that the entire scientific community are working under the same conventions.
\begin{itemize}
\item Number the link sequentially from \(0\) to \(n\). Giving us \(n+1\) links.
\item Number the joints sequentially from \(1\) to \(n\), Giving us \(n\) joints.
\item Number of co-ordinate frames: \(n+1\)
\item The \(Z_i\) axis is aligned with the \((i+1)^th\) joint axis.
\item \(X_i\) is defined along the common normal between the \(Z_i\) and \(Z_{i-1}\) axis.(Common normal is the line which is perpendicular to both of these axes.)
\item \(Y_i\) is obtained via cross product between \(Z_i\) and \(X_i\) axis.
\item The origin is placed at this point of intersection of \(x_i\), \(y_i\) and \(z_i\)
\item We define \(H_i\)(which is the point of intersection of the common normal of the next axis with the current axis.), and 4 new terms:
\begin{itemize}
\item \(a_i\) - Offset distance between two adjacent joint axes(The distance between \(O_i\) and \(H_{i-1}\))
\item \(d_i\) - Distance between \(H_{i-1}\) and \(O_{i-1}\)
\item \(\alpha_i\) Angle between \(Z_i\) and \(Z_{i-1}\) when viewed from \(X_i\), this is also known as the twist angle.
\item \(\theta_i\) Angle between \(X_i\) and \(X_{i-1}\) when viewed from \(Z_i\),
\end{itemize}
\end{itemize}

For a 3R Planar serial chain manipulator:

\begin{center}
\begin{tabular}{llrrl}
\hline
 & a\textsubscript{i} & d\textsubscript{i} & \(\alpha_{i}\) & \(\theta\)\textsubscript{i}\\
\hline
Joint i = 1 & L\textsubscript{1} & 0 & 0 & \(\theta\)\textsubscript{1}\\
Joint i = 2 & L\textsubscript{2} & 0 & 0 & \(\theta\)\textsubscript{2}\\
Joint i = 3 & L\textsubscript{3} & 0 & 0 & \(\theta\)\textsubscript{3}\\
\hline
\end{tabular}
\end{center}
\section{Rotation Matrix}
\label{sec:org8a53922}
We have three reference frames, with a common origin. With the notation we have setup we can say, \(\{0\},\{1\},\{2\}\) can be defined for a point \(P\)

\begin{align*}
^0P = ^0_1R ^1P \\
^0P = ^0_2R ^2P \\
^1P = ^1_2R ^2P \\
^0_2R = ^0_1R _2^1R
\end{align*}
\section{Derivation Using DH Parameters}
\label{sec:org88e1d0a}
These 4 parameters can be expressed by,

$$^{i-1}A_i = T(z,d)T(z,\Theta),T(x, \alpha), T(x, a)$$
\section{Skew Symmetric Matrix}
\label{sec:org1a6b910}

A matrix such that \(A = -A^T\)

So this means that,

$$S + S^{T}=0$$

$$\vec{a} = a_x \hat{i} + \vec{a_y} \hat{j} \vec{a_z} \hat{k}$$

We define a matrix function,

$$S(\vec{a}) = \begin{bmatrix}
0 & -a_z & a_y \\
a_z & 0 & - a_x \\
-a_y & a_x 0
\end{bmatrix}
$$

So we can say for the basis vectors

$$S(\hat{i}) = \begin{bmatrix}
0 & 0 & 0 \\
0 & 0 & -1 \\
0 & 1 & 0
\end{bmatrix}
$$
$$S(\hat{j}) = \begin{bmatrix}
0 & 0 & 1 \\
0 & 0 & 0 \\
-1 & 0 & 0dt&i0
\end{bmatrix}
$$
$$S(\hat{k}) = \begin{bmatrix}
0 & -1 & 0 \\
1 & 0 & 0  \\
0 & 0 & 0
\end{bmatrix}
$$

Multiplying the matrix function's output with a vector is the same as doing a cross product with those two vectors.
$$S (\vec{a}) P = \vec{a} \times \vec{p} $$
This matrix operation also has linearity in operations.
$$S (\alpha\vec{a} + \beta \vec{b}) P = \alpha S(\vec{a}) +  \beta S(\vec{b}) $$
When multiplied with a rotation matrix.
$$^a_bR S(\vec{a})^a_bR^T = S(^a_bR \vec{a})$$

We know that,
\begin{align*}
R R^T &= I \\
R(\theta) R^t(\theta) &= I \\
\text{Taking the derivative} \\
\frac{d R(\theta)}{d \theta} R^T(\theta) + R(\theta) \frac{d R^{T}(\theta)}{d \theta} &= 0 \\
\frac{d R(\theta)}{d \theta} R^T(\theta) + R(\theta) \left( \frac{d R^{T}(\theta)}{d \theta} \right)^{T}&= 0 \\
\end{align*}

What this means is that the term \(\frac{d R(\theta)}{d \theta} R^T (\theta)\) is a skew-symmetric matrix.

This means that the the term can be written as \(S\),

$$S = \frac{d R(\theta)}{d \theta} R^T (\theta)$$

In other words, S becomes an operator on \(R\) to give us the derivative.

$$S(\hat{k})R(\theta) = \frac{d R(\theta)}{d\theta}$$

Taking both sides with respect to time,

$$ \frac{d R(\theta)}{d \theta} \frac{d \theta}{dt} = \dot{\theta} S(\hat{k})R(\theta) $$
In other words,
$$\dot{R} = S(\dot{\theta}\hat{k})R(\theta)$$

So we have,
$$\dot{R} = S(\vec{\omega})R(\theta)$$

$$S(\vec{\omega}) = \begin{bmatrix}
0 & -\omega_z & \omega_y \\
\omega_z & 0 & -\omega_x \\
\omega_y & \omega_{x} & 0
\end{bmatrix}$$

$$\Omega = \begin{bmatrix}
0 & -\omega_z & \omega_y \\
\omega_z & 0 & -\omega_x \\
\omega_y & \omega_{x} & 0
\end{bmatrix}$$

\begin{align*}
^0P = ^0_1R ^1P \\
^0P = R ^1P \\
^0 \dot{P} = \dot{R} ^1P + R^1 \dot{P} \\
^0 \dot{P} = S(\vec{\omega}) R ^1P \\
^0 \dot{P} = \omega \times R ^1 \vec{p} \\
^0 \dot{P} = \omega \times R ^1 \vec{p} \\
\vec{v} = \vec{\omega} \times \vec{r}
\end{align*}
\chapter{Unit 2 - Kinematics Of Robotic Manipulators}
\label{sec:org1b37152}
\section{Forward And Inverse Kinematics}
\label{sec:orgec32e2c}
Forward kinematics is the use of the joint space to get to the task space.
Inverse Kinematics is the use of the task space to get to the joint space.

Workspace - The space of all points that the end effector can reach.
\section{Inverse Kinematics Of 2R Planar Structures}
\label{sec:orgbe26c8d}

We solve for \(q_2\) and eliminate \(q_1\)

\begin{align*}
q_2 &= \cos^{-1} \kappa \\
\kappa &= \frac{^0X_{2}^2 + ^0Y_{2}^2 - L_1^2 - L_2^2}}{2L_1L_2} \\
\end{align*}

We have two cases,

Case 1: \(-1<\kappa<1\) We have two distinct and real solutions of \(q_2\)

Case 2: \(|\kappa| = 1\) We have one solution

Case 3: \(|\kappa| < 1\) No solution exists
\section{Forward Kinematics And Inverse Kinematics Of 3R Planar Structures}
\label{sec:org418a560}
$$X_e = L_1 \cos {q_1} + L_2 \cos(q_1+ q_2) + L_3 cos(q_1 + q_2 + q_3)$$

For a 3R Planar structure, the way to find the forward kinematics is straightforward.

What we get for the inverse kinematics is with 3 knowns, \(^0X_e, ^0Y_e, \phi\) where \((\phi = q_1 + q_2 + q_3)\) we reduce the problem down to \(^0X_{2}, ^0Y_2\) by finding,

\begin{align*}
^0X_e - L_3 \cos{\phi} &= L_{1} \cos q_1 + L_2 \cos q_{12}\\
^0Y_e - L_3 \sin{\phi} &= L_{1}\sin q_1 + L_2 \sin q_{12}
\end{align*}

Now we know that those terms are nothing but

\begin{align*}
^0X_{2}&= L_{1} \cos q_1 + L_2 \cos q_{12}\\
^0Y_{2} &= L_{1}\sin q_1 + L_2 \sin q_{12}
\end{align*}

Which we can solve the same as the 2R Planar Structure
\section{Jacobian Forward Kinematics}
\label{sec:org8eefc3d}
Given \(\dot{q_{1}}, \dot{q_{2}}\), how do we get \(^0X_{e}, ^0Y_e\)? and similarly how do we get the angular velocities of a machine from the end effector's velocity.

We know that for 2R Planar SCMs,
\begin{align*}
^0X_e &= L_1 c q_1  + L_2 c(q_1 + q_2) \\
^0Y_e &= L_1 s q_1 + L_2 s(q_1+q_2) \\
^0\dot{X_e} &= -\dot{q_1}L_1 s q_1 - L_2s(q_{1}+q_2)\times(\dot{q_1} + \dot{q_2}) \\
^0\dot{Y_e} &= +\dot{q_1}L_1 c q_1 + L_2c(q_{1}+q_2)\times(\dot{q_1} + \dot{q_2})
\end{align*}

We have,
$$\dot{X}_{2\times1} = J_{2 \times2} \times \dot{q}_{2 \times1}$$


We now have this for a specific case, when we generalize,
$$ \dot{X} =  \begin{bmatrix} ^0_nv \\ ^0_n\omega \\ \end{bmatrix} = J \dot{q}$$

So now we can write the term

$$^0_{n}\omega = \sum_{i=1}^n \dot{\theta}_i $$
\section{Singularity}
\label{sec:org0436413}
So now we have,

\begin{align*}
\dot{X} &= J \dot{q} \\
\dot{J} &= q^{-1} X \\
J^{-1} &= \frac{adj(J)}{det(J)}
\end{align*}

We term singularity to be the case where the determinant of \(J\) is 0


$$det(J) = L_1L_2\sin{\theta_2}$$


This means that we're only depending on \(\theta_2\)

Which cannot be \(n\pi\)

What this means is that when \(\theta_2\)  should never be zero.

This means that when the arm is fully extended or retracted then it is not a point that the robot can recover from.
\section{Finding The Jacobian quickly}
\label{sec:org4d96d8a}
\begin{align*}
\dot{X}_{m \times 1} &= J \dot{q} \\
J &= [ J_1_{n \times 1}, J_2_{n \times 1} , \cdots, J_n] \\

J_i = \begin{bmatrix}^0_{i-1}z \times ^{i-1}_{n}p^{*} \\ ^0_{i-1}z\end{bmatrix}
\end{align*}

Where \(^{i-1}_n P^*\) is the position vector of \(\{n\}th\) origin relative to \(\{i-1\}^{th}\) frame but expressed in \(\{0\}^{th}\) frame.


\begin{align*}
J_i = \begin{bmatrix} ^0_{i-1}z \\ 0  \end{bmatrix} \\

\end{align*}
\section{Lab Questions}
\label{sec:org8134b85}
\begin{enumerate}
\item Question 1
\label{sec:org06446b4}
Consider the 3R Planar SCM with \(L_1 = 2m\), \(L_2 = 3m, L_3 = 4m\)
\begin{verbatim}
L = [2;3;4];

% Workspace Points
WS_points = zeros(0,0);
\end{verbatim}
\begin{enumerate}
\item Forward Kinematics,
\label{sec:orgcb34c91}
Find the co-ordinates of the end-effector in the base frame:
\begin{enumerate}
\item For \(q = [q_1 q_2 q_3]^T= [30\deg 45 \deg 60 \deg]^{T}\)
\item For \(q = [q_1 q_2 q_3]^T= [270\deg 75 \deg 10 \deg]^{T}\)
\end{enumerate}
\item Inverse kinematics,
\label{sec:org8ce9ed7}
Find the joint space variables,
\begin{enumerate}
\item For \(\phi = q_1 + q_2 + q_3 = 0\) and \([^0X_e, ^0Y_e] = [2,4]\)
\item For \(\phi = q_1 + q_2 + q_3 = 45\) and \([^0X_e, ^0Y_e] = [5,5]\)
\end{enumerate}
\item Workspace Analysis
\label{sec:org443bff5}
Draw the workspace of the 3R planar serial chain manipulator using forward and inverse kinematics.
\begin{enumerate}
\item For \(\phi = 0\) and
\item For \(\phi = 45\deg\)
\end{enumerate}
\begin{enumerate}
\item Using Forward Kinematics
\label{sec:orgf730c03}
\item Using Inverse Kinematics
\label{sec:org35699d0}

\begin{align*}
\begin{bmatrix} ^0X_e \\ ^0Y_e \end{bmatrix} &= \begin{bmatrix}L_1 cos q_1 + L_2cos(q_{12}) + L_3 cos(\phi) \\ L_1 sin q_1 + L_2sin(q_{12}) + L_3 sin(\phi) \end{bmatrix}\\
\begin{bmatrix} ^0X_e - L_3 cos(\phi) \\ ^0Y_e - L_3 sin(\phi) \end{bmatrix} &= \begin{bmatrix}L_1 cos q_1 + L_2cos(q_{12}) \\ L_1 sin q_1 + L_2sin(q_{12}) \end{bmatrix} \\
\begin{bmatrix} ^0X_2 \\ ^0Y_2 \end{bmatrix} &= \begin{bmatrix}L_1 cos q_1 + L_2cos(q_{12}) \\ L_1 sin q_1 + L_2sin(q_{12}) \end{bmatrix} \\
^0X_2^2 + ^0Y_2^2 &= (L_1^2) + (L_2)^2 + 2L_1L_2\cos(q_2)\\
q_2 = \cos^{-1}\kappa
\end{align*}

We plot the workspace by checking the point \((L_{1} + L_2 + L_3,0)\) and \((-(L_1 + L_2 + L_3),0)\) as they are the points at which the arm is fully extended. This means that the radius of the circle we need to check is just a circle where \(r = (L_1 + L_2 + L_{3})\)

We then look at points where \(\kappa > -1\) and \(\kappa < 1\) since those are the points which are reachable in the workspace.

\begin{verbatim}
P = zeros(3,1);
P(3) = 0;
range = linspace(-(L(1) + L(2) + L(3)),L(1) + L(2) + L(3));
[X,Y] = ndgrid(range);

% Position vector of {2} from ground frame
for i = 1:length(X)
  for j = 1:length(Y)
    P(1) = X(i,j);
    P(2) = Y(i,j);
    X2 = [P(1) - L(3)*cos(P(3));P(2) - L(2)*sin(P(3))];
    kappa = (X2(1)^2 + X2(2)^2 - L(1)^2 - L(2)^2)/2*L(1)*L(2);
        if(kappa >= -1 && kappa <= 1)
        % Appending to the workspace
        WS_points = cat(1,WS_points,[P(1) P(2)])
        end
  end
end
\end{verbatim}

Now we plot,
\begin{verbatim}
figure('1, "visible", "off", 'units', 'normalized','outerposition',[0 0 0.5 1])

grid on; grid minor
ylim([-((L(1) + L(2) + L(3))-1 L(1) + L(2) + L(3))+1 ])
xlim([-((L(1) + L(2) + L(3))-1 L(1) + L(2) + L(3))+1 ])
axis equal
xlabel('X-Axis(m)')
ylabel('Y-Axis(m)')

ans = "../images/octave-chart.png";
\end{verbatim}
\end{enumerate}
\end{enumerate}
\end{enumerate}
\chapter{Unit 3- Statics And Dynamics}
\label{sec:org331f141}
Statics is the study of the forces and torques acting on bodies that are at rest or in a state of equilibrium.

How does this play into robotics? This comes in handy in situations where the robot must stand still. It needs to apply forces to counteract all the forces on it to stay stationary.

We know from inverse kinematics that,

$$\dot{X} = J \dot{q}$$

The final answer here is,
$$\tau = J^TF$$
Where \(\tau\) is the joint torques and \(F\) is the End-Effector's wrench vector, which containeds both forces and torques.

We invert both sides

$$\dot{q} = J^{-1}\dot{X}$$ and $$F = (J^T)^{-1}\tau$$

With singularity, we're unable to map the end effector's veloticy to joint space velocities. It also means we're unable to map joint torques(\(\tau\)) to the Wrench vector(\(F\)) In the neighbourhood of  sinularity, samll velocity in task space will cause very high velocity in joint space, and small joint torques will cause very high forces and torques.
\section{Proof}
\label{sec:orgfe790b7}
\textbf{Using work method},

Work done by external agent at the Ende Effector = Work done by actuators at the joint.

$$\vec{F} \cdot \partial {\vec{X}} = \tau \cdot \partial \theta $$
$${F}^{T} \cdot \partial {\vec{X}} = \tau^{T} \cdot \partial \theta $$

From Jacobian relation:

$$F^T = J \partial \theta => \partial X = J \partial \theta$$

So,
$$F^T J d \theta = \tau^T d \theta$$
$$F^T J = \tau^T d$$

Taking the transpose on both sides
$$J^{T} F  = \tau$$

Using Newton's second law,
Define the wrench vector for the end effector
$$F_{6 \times 1} = \begin{bmatrix}f_{3 \times 1} \\ n_{3 \times 1}\end{bmatrix}$$


For a given link \(i\)

$$\Sum \vec{f} = 0$$

\begin{align*}
f_{i,i-1} - f_{i+1,i} + m_ig &= 0 \\
f_{i,i-1} = f_{i+1,i} + m_ig  \\
\end{align*}

\begin{align*}
n_{i,i-1} = n_{i+1,1} + (r_i \times f_{i,i+1}) - (*r_{com(i),i} \times m_i g) \\
\end{align*}

We iterate from i = n to 1, and we calculate the forces first.

Lung wen tsai 266-268
\chapter{Velocity and force ellipse}
\label{sec:orgc58d4e9}
We take the force vectors and the end effector velocity, \(\dot{X}\)

We put the constraint that \(\dot{X}^T\dot{X} = 1\)

For a 2R planar, this gives us the equation of a unit circle in task space
For a 3R planar, this gives us the equation of a unit sphere in task space

\begin{align*}
\dot{X}^T\dot{X} &= 1 \\
\dot{q}^{T}J^{T} J \dot{q} &= 1 \\
\end{align*}

The Jacobian is an expression of the configuration, the robot's shape and orientation.

A property of the Jacobian is that it is symmetric positive semi-definite, this implies the following
\begin{itemize}
\item The eigenvalues of \(J^{T} J\) are non-negative
\item Eigenvalues tell the length of major axis and minor axis value of the ellipse in joint space.
\item Eigenvector tells the direction of the major and minor axis of the ellipse in joint space
\end{itemize}

The ellipse constantly changes as the jacobian changes.

But there are points in the robot's workspace where the ellipse becomes a perfect circle. These are known as \textbf{'isotropic points'}

And at points of singularity, the ellipse flattens.

Similarly,
We can do this for
$$\tau = J^T F$$
\section{Finding the locus of isotropic points}
\label{sec:org43a405f}
At isotropic points, the we find the eigenvalues of the jacobian. then we equate the eigenvalues such that $$\frac{1}{\lambda_{1}} = \frac{1}{\lambda_2}=1$$ so that the major and minor axis are equal to form a circle.

Similarly the eigenvectors define the orientation.
\chapter{Dynamics}
\label{sec:org0796bd0}
It's the study of systems in motion under the influence of external forces and torques.

What forces and torques are required to achieve a given (desired) motion of a robot.

For a general second order system

\[ m\ddot{x} + b \dot{x} c x = F(t) \]

This is true for a general robotic manipulator,

$$M \ddot{q} + C \dot{q} + G = \tau$$

We get this in matrix form.

$$M(q) \ddot{q} + C(q,\dot{q})\dot{q} + G(q) = \tau$$

\begin{itemize}
\item Inertia matrix \(M(q)\)
\begin{itemize}
\item tells about the mass distribution of the robotic manipulator
\item positive semi-definite
\item always iverttible
\end{itemize}
\item Velocity Coupling Vector \(C(q,\dot{q})\)
\begin{itemize}
\item Velocity squared - centripetal acceleration, \(q_{1}^2\)
\item cross velocity multiplication, \(\dot{q_1}\dot{q_2}\)
\end{itemize}
\item Gravitational Force Vector \(G(q)\)
\item \(tau\) is nothing but \(\tau + J^TF\)
\end{itemize}


$$D(q) \ddot{q} + H(q,\dot{q}) + G(q) = \tau$$

You start with task space, apply inverse kinematics to find the joint space and find \(\ddot{q}\) to fully understand the joint space. Then apply forward dynamics to find the torque space.

Finding the inertia matrix is done like so,

$$D = \sum_{i=1}^n(J^T_{vi} m_i J_{vi} + J^T I_i J_{wi})$$

Finding the velocity coupling vector,

$$H_{ijk}\sum_{j=1}^n\sum_{k=1}^n(\frac{\partial D_{ij}}{\partial q_k} - \frac{1}{2}\frac{\partial D_{jk}}{\partial q_i})$$
\part{Practice Sheets}
\label{sec:orgec109a1}
The notations used in our lectures are similar with this book.

\begin{itemize}
\item For Rotation Matrices, Pose of a rigid body, Euler Angles, Homogeneous Transformation ---> read Chapter 2
\item For DH Parameters and Coordinate Convention ---> read Chapter 3, Sections 3.1 to 3.4
\end{itemize}
\part{Assignment}
\label{sec:org57db669}
Free marks

Kinetic energy can be found by $$\frac{1}{2} v^T m v$$
\part{End-Sem Project}
\label{sec:org91f29aa}
\begin{enumerate}
\item Select a robotic system
\begin{enumerate}
\item select only serial chain robotic manipulator
\item select standard SCM
\item No more than 5 or 6 joints
\item Robotic SCM can be as simple as planar 2R manipulator or complex as Stanford Manipulator
\item The simpler the robotic manipulator, problem should be detailed
\item No more than 2 teams should select the same robotic system, but they should work on different problems.
\end{enumerate}
\item Problems
\begin{center}
\begin{tabular}{lll}
\hline
Problems & Type & Examples\\
\hline
Forward Kinematics & Compulsory & Completely define the robotic system using DH Parameters.\\
 &  & Find homogeneous transformation and jacobian matrix\\
 &  & Give arbitrary end-effector position by user and robotic SCM should reach it\\
\hline
Inverse Kinematics & Compulsory & Find the workspace and show it graphically on simulation\\
 &  & Take up some inverse kinematics problems on your will.\\
 &  & Examples-\\
 &  & Trajectory planning, stylus writing manpulator, dance moves performed by the end effector\\
\hline
Statics & Optional & \\
Dynamics & Optional & \\
Controls & Optional & \\
\hline
\end{tabular}
\end{center}
\item Software
\begin{enumerate}
\item Matlab - scripts/functions, simscabe, rigidbody, multibody tool
\item CoppeliaSim
\end{enumerate}
\end{enumerate}
\part{Dynamics}
\label{sec:org81a6a4c}
\part{Syllabus}
\label{sec:org7388c6a}
\href{file:///home/adithya/university-notes/Introduction To Robotics/textbooks/Introduction-to-Robotics-Craig.pdf}{Chapter 2
}- Pose of a rigid body
\begin{itemize}
\item Position + Orientation
\end{itemize}
\begin{itemize}
\item Rotation Matrices
\item Homogeneous Transformation Matrix
\end{itemize}

\href{file:///home/adithya/university-notes/Introduction To Robotics/textbooks/Introduction-to-Robotics-Craig.pdf}{Chapter 3 - 3.1 to 3.4
}- DH Parameters+Co-ordinate Convention

\href{file:///home/adithya/university-notes/Introduction To Robotics/textbooks/(Tj211.T75) Lung-Wen Tsai - Robot Analysis_ The mechanics of serial and parallel manipulators (1999).pdf}{(Tj211.T75) Lung-Wen Tsai Segment on Direct And Inverse Kinematics}
\href{file:///home/adithya/university-notes/Introduction To Robotics/textbooks/Introduction-to-Robotics-Craig.pdf}{John J Craig On Forward And Inverse Kinematics}
\begin{itemize}
\item Forward Kinematics
\item Inverse Kinematics
\end{itemize}

\href{file:///home/adithya/university-notes/Introduction To Robotics/textbooks/(Tj211.T75) Lung-Wen Tsai - Robot Analysis_ The mechanics of serial and parallel manipulators (1999).pdf}{4.5-4.6.1}
\begin{itemize}
\item Jacobian
\item Singularity
\end{itemize}
\part{Assignment}
\label{sec:org117816d}
\chapter{Quetion 1}
\label{sec:org6033d27}
\part{Project}
\label{sec:org741c28b}
\chapter{Robotics Project}
\label{sec:orgdab67c9}
\section{Forward Kinematics}
\label{sec:org5b8c9aa}
Find the homogeneous transformation matrices, find jacobian(pen and paper). Simulate inputting joint space variables to give the end effector's point. The simulation should show the end effector moving to that point (code)
\section{Inverse Kinematics}
\label{sec:org7d4286b}
Select two points within the workspace and trace the trajectory of the robot between those two points or find the workspace and plot it.
\section{Statics}
\label{sec:org1427f62}
Simulate a robot under some forces and the robot maintains its equilibrium. Come up with some creative problems, such as the robot is holding a heavy object and it applies a force to hold the object against gravity.
\section{Dynamics}
\label{sec:org973838c}
\begin{enumerate}
\item Forward
\label{sec:org6ca9748}
Knowing the joint torques and the external forces/torques, find the manipulator's trajectory \(q(t),\dot{q(t)}, \ddot{q(t)}\)
\item Inverse
\label{sec:org5e7c779}
Find the joint torques required to move a robot in a specific trajectory. Find the trajectory of a robot with a specific joint torque vector.
\end{enumerate}
\section{Report Is Required}
\label{sec:org73cba2b}
You can use a MATLAB live script or \LaTeX{}
\chapter{Modelling Project}
\label{sec:org5c4aff1}
\section{Modelling}
\label{sec:orgc67ddac}
Make FBD and model it with analytical method.
\section{Dynamical Response To Standard Input}
\label{sec:org53e3530}
Compare the robot's response to step, ramp and impulse input.
\section{Frequency Response}
\label{sec:orgd1aa60e}
Apply a sinusoidal input, and bode plot, interpret the results.
\end{document}
