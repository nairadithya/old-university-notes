% Created 2024-09-25 Wed 12:56
% Intended LaTeX compiler: pdflatex
\documentclass[11pt]{report}
\usepackage[utf8]{inputenc}
\usepackage[T1]{fontenc}
\usepackage{graphicx}
\usepackage{longtable}
\usepackage{wrapfig}
\usepackage{rotating}
\usepackage[normalem]{ulem}
\usepackage{amsmath}
\usepackage{amssymb}
\usepackage{capt-of}
\usepackage{hyperref}
\input{preamble}
\author{Adithya Nair}
\date{\today}
\title{Software Defined Communication System}
\hypersetup{
 pdfauthor={Adithya Nair},
 pdftitle={Software Defined Communication System},
 pdfkeywords={},
 pdfsubject={},
 pdfcreator={Emacs 29.4 (Org mode 9.7.11)}, 
 pdflang={English}}
\begin{document}

\maketitle
\tableofcontents

\part{Notes}
\label{sec:orgc34de2d}
\chapter{Unit 1}
\label{sec:org7d2c385}
\section{Classification Of Systems}
\label{sec:org7e67950}
\begin{itemize}
\item No. of inputs - SISO, MIMO
\item Type of signal - Continuous time-system, discrete
\item Dimension - One-dimensional, multi-dimensional
\end{itemize}
\section{Even And Odd Signals}
\label{sec:org93dcdf3}
\begin{enumerate}
\item Even Signal
\label{sec:org69eba5d}
$$x(t) = x(-t)$$
\item Odd Signal
\label{sec:orgde50520}
$$x(t) = -x(-t)$$
\end{enumerate}
\section{Signal Energy And Power}
\label{sec:org29adfdf}
$$E = \int_{-\infty}^{\infty}|x(t)|^2dt$$
$$P \lim_{T \rightarrow \infty} \frac{1}{T} \int_{-T/2}^{T/2} |x(t)|^2dt$$
\section{Linear Time Invariance}
\label{sec:org972c01e}
\begin{enumerate}
\item Why LTI?
\label{sec:org924b53d}
\begin{itemize}
\item The effect of a system on the spectrum can be analyzed easily if and only if the system is LTI.
\end{itemize}
\item How To Check Time Invariance
\label{sec:orgc7bb8da}
\begin{itemize}
\item No time scaling
\item Coefficient should be not be time dependent.
\item Any addition or subtraction term to the system, should be constant or zero.
\end{itemize}
\end{enumerate}
\section{Linearity}
\label{sec:org286224a}
\begin{itemize}
\item Any linear  system is independent of time scaling
\item If the order of system is greater than 1, then it is not linear.
\end{itemize}
\section{Discrete Convolution}
\label{sec:orgacc2079}
$$y[n] = \Sigma_{k = -\infty}^{\infty}x[k]h[n-k]$$
\section{Sinusoids}
\label{sec:orgf6f79c6}
\begin{enumerate}
\item Why Do We Need Sinusoids?
\label{sec:org6801b64}
\begin{itemize}
\item Used in modulation schemes
\item Used in communication systems
\item Any signal can be represented as a sum of sinusoids.
\item Filters and equalizers can be characterized by their response
\end{itemize}
\end{enumerate}
\section{Complex Numbers}
\label{sec:org983677c}
\begin{itemize}
\item Inphase, \(V_I = |V| \cos(V)\)
\item Quadrature \(V_Q = |V| \sin(V)\)
\end{itemize}
\section{Complex Sinusoids}
\label{sec:orgd01b7c6}
\begin{itemize}
\item V rotating anticlockwise at a constant rate
\item Made up of two real sinusoids.
\item Complex sinusoids with frequency F is composed of,
\begin{align*}
V_I = \cos 2 \pi Ft \\
V_Q = \sin 2 \pi Ft \\
\end{align*}
\end{itemize}
\section{Filters}
\label{sec:orgcd435f2}
\begin{enumerate}
\item Primary Functions
\label{sec:org250d6c8}
\begin{itemize}
\item To confine a signal into a prescribed frequency band.
\item To decompose a signal into two or more sub-band signals for sub-band signal processing, in music
\item To modify frequency spectrum
\item To model input output relation of a system.
\end{itemize}
\item Types Of Filters
\label{sec:orge1ab688}
\begin{itemize}
\item Low pass
Capacitor in parallel and inductor in series
\item High pass
\item Band pass
\item Band stop
\item Band reject
\item Notch
\end{itemize}
\item Pass Band
\label{sec:org05f39e5}
Range of frequences in which the filter allows the signal to pass
\item Stop Band
\label{sec:org00be9e7}
Range of frequences which will reject passage
\end{enumerate}
\chapter{Unit 2}
\label{sec:org5bdaffd}
\section{Modulation}
\label{sec:org9d60fdd}
Why do modulation?
\begin{itemize}
\item Reduces the size of antenna $$L \propto \lambda \text{ and } \lambda = \frac{c}{f}$$

\item To reduce interference
\item To improve SNR(\(\frac{\text{Power Of Signal}}{{Power Of Noise}}\))
\item To allow multiplexing of the signals
\item Optimizes bandwidth utilization
\end{itemize}
Value of the modulating signal,

$$v_1 = V_c + v_m = V_c + V_m \sin{2 \pi f_m t}$$
$$v_{2} = v_{1} sin 2\pi f_ct$$

$$\text{Modulation Index}, m = \frac{V_m}{V_c}$$
\section{Over-Modulation}
\label{sec:org1b53222}
When \(V_m > V_{c}\), over-modulation. Automatic circuits called compression circuits solve this problem by amplifying the lower signals and suppressing or compressing the higher level signals
\section{Side-Band Calculation}
\label{sec:org68cdac4}
$$V_{AM} = V_C sin 2 \pi f_c t + (V_m sin 2 \pi f_c t)(sin 2 \pi f_c t)$$
$$V_{AM} = V_C sin 2 \pi f_c t + \frac{V_m}{2} cos 2 \pi t (f_c - f_m) - \frac{V_{m}}{2} \cos 2 \pi t (f_c + f_{m}) $$

\begin{align*}
f_{USB} = f_c + f_m \\
f_{LSB} = f_c - f_m \\
\end{align*}
\begin{enumerate}
\item Types Of Modulation
\label{sec:org34d1b7e}
\begin{enumerate}
\item Analog Modulation
\begin{itemize}
\item Amplitude Modulation
\item Frequency Modulation
\item Phase Modulation
\end{itemize}
\item Digital Modulation
\end{enumerate}
\end{enumerate}
\section{Analog Modulation}
\label{sec:org7672938}
\begin{enumerate}
\item Total Power
\label{sec:orgd0ef6a1}
For power calculation, rms value is used.
$$P_T = P_C + P_{USB} + P_{LSB}$$
\textbf{**}
\item Power In Terms Of Modulation Index
\label{sec:org117e346}
$$P_T = P_C(1 + \frac{m^2}{2})$$
\end{enumerate}
\section{Types Of Amplitude Modulation}
\label{sec:org8b78d20}
\begin{enumerate}
\item Double SideBand - Suppressed Carrier
\label{sec:org948e0a2}

$$s_{am-dsb-sc}(t) = \frac{A_iA_c}{2}(cos(\omega_c- \omega_i)t + cos((\omega_c + \omega_i)t)$$

\begin{itemize}
\item In AM, 2/3rds of the transmitted power is in the carrier, which conveys no information
\item No power is wasted on the carrier
\item It's given an efficiency of 50\%
\item Used in transmission of color info in a TV signal.
\end{itemize}
\begin{enumerate}
\item Transmission Efficiency
\label{sec:org381a7b0}
$$\eta = \frac{P_{USB} + P_{LSB}}{P_{T}} = \frac{\mu^2}{\mu^2 + 2}$$
\end{enumerate}
\item Single SideBand
\label{sec:orgbbede92}
\begin{itemize}
\item The primary benefit is that it occupies only half of AM-DSB
\item SSB signals occupy narrower bandwidth so less signal noise
\item Less selective fading of SSB signal over long distances.
\end{itemize}

$$s_{am-susb}(t) = \frac{A_iA_c}{2}cos((\omega_c + \omega_i)t)$$
$$s_{am-slsb}(t) = \frac{A_iA_c}{2}cos((\omega_c - \omega_i)t)$$
\begin{enumerate}
\item Advantages
\label{sec:org8e8114e}
\begin{itemize}
\item Half the bandwidth is required compared to DSB
\item Due to suppression of carrier and one sideband power is saved
\item Reduced noise interference due to to reduced bandwidth
\end{itemize}
\item Disadvantages
\label{sec:org170013c}
\begin{itemize}
\item Generation and reception of SSB signals are complex
\item SSB transmitter and receiver need to have an excellent frequency stability
\item SSB modulation is expensive and complex to implement.
\end{itemize}
\item Applications
\label{sec:org75daf04}
\begin{itemize}
\item Used at HF segment of the spectrum(for frequency below 10MHz-LSB, above 10MHz - USB)
\item Used where power saving is required
\item Also used in low bandwidth requirements
\end{itemize}
\end{enumerate}
\item Vestigial SideBand
\label{sec:org4c0cc86}
\begin{itemize}
\item Used to reduce the spectral requirements of analog TV
\item Apply BPF to AM-DSB-TC signals to suppress most of the sidebands.
\item In VSB, one sideband and part of the other sideband is transmitted
\item Bandwidth is slightly higher than SSB
\item Easier to implement than SSB.
\end{itemize}
\end{enumerate}
\section{Amplitude Demodulation}
\label{sec:org9c1f102}
\begin{enumerate}
\item Coherent AM Demodulation
\label{sec:org18ba5b4}
To demodulate an AM signal, a recover must multiply the received signal with a sine wave, that has exactly the same frequency and phase as the carrier embedded within it.

The mixing operation shifts the modulated information from being centered around carrier frequency back to baseband.
\begin{enumerate}
\item Demodulation of AM-DSB-SC signals
\label{sec:org0facd58}
Given that the sine wave synthesized with the same frequency and phase
$$s_d = \frac{A_iA_c}{2}\cos{\omega_i t}$$
\end{enumerate}
\item Non-coherent AM Demodulation
\label{sec:org5e34a7b}
\begin{itemize}
\item During the positive half cycle diode conducts and the capacitor charges to the peak voltage.
\item During -ve half cycle capacitor discharges
\item Because the capacitor charges and discharges, the recovered signal has a small amount of ripple
\item Distortion occurs when the time constant of resistor and capacitor is too long or too short.
\item If the time constant is too long, the discharge is too slow to follow the faster changes in the modulated signal.
\end{itemize}
\end{enumerate}
\section{Frequency Modulation}
\label{sec:org5d3a382}
\begin{itemize}
\item The carrier amplitude remains constant, the carrier frequency is changed by the modulating signal.
\item If amplitude of modulating signal increases, frequency increases, if amplitude decreases, frequency decreases
\item Change in carrier frequency produced by modulating frequency is \(f_d\).
\item FM is used for commercial radio.
\end{itemize}
\begin{enumerate}
\item Voltage Controlled Oscillator
\label{sec:orgb8a9040}
$$\hat{\theta}(t) = k_o \int_{-\infty}^t v(t) dt$$
\begin{itemize}
\item \(k_o\) - voltage to frequency gain ratio
\item integrated(change phase by 90)

$$s_{fm} = A_c \cos{\omega_c t + 2 \pi K_{fm} \times \int_{-\infty}^ts_i(t) dt}$$
\end{itemize}
\item Modulating a sine wave
\label{sec:org0d6346e}
$$s_i = A_i \cos{\omegas_i t}$$
$$s_{fm} = A_c \cos{(\omega_{c} t) + \beta_{fm} \cos{\omegas_i t}$$

Where,
\begin{itemize}
\item \(\beta_{fm}\) - modulating index(\(\frac{\Delta f}{f_i}\))
\end{itemize}

The instantaneous frequency is

$$f_{fm} = f_c + K_{fm} s_i(t)$$
\item Sidebands
\label{sec:orgecb9d61}
In FM, a large number of sidebands are generated. These sidebands are of the form, \(f_c \pm kf_m\)
\end{enumerate}
\section{Types Of FM}
\label{sec:org65d600b}
\begin{itemize}
\item Narrowband FM
$$\beta_{fm} << \frac{\pi}{2} or 1$$

$$s_{fm-nfm} = A_c [\cos{\omega_c t} + \frac{\beta_{fm}}{2}\cos(\omega_c+ \omega_{i} t)+ \frac{\beta_{fm}}{2}\cos(\omega_c - \omega_{i} t)]$$
\item Wideband FM
$$\beta_{fm} >> \frac{\pi}{2} or 1$$
\end{itemize}
\begin{enumerate}
\item Advantages Of FM Over AM
\label{sec:org012a1a2}
\begin{itemize}
\item Improved Signal-To-Noise Ratio
\item Better Sound Quality
\item Reduced Distortion
\end{itemize}
\end{enumerate}
\section{Phase Alternating Line}
\label{sec:orgf927207}
PAL is a composite video because luminance(luma, monochrome image) and chrominance(chroma, color applied to monochrome) are transmitted as one signal.
\end{document}
