\documentclass{report}

\input{preamble}
\input{macros}
\input{letterfonts}

\title{\Huge{23CHY115-Introduction To Material Informatics}\\Lecture Notes}
\author{\huge{Adithya Nair}}
\date{}

\begin{document}

\maketitle
\newpage% or \cleardoublepage
% \pdfbookmark[<level>]{<title>}{<dest>}
\pdfbookmark[section]{\contentsname}{toc}
\chapter{Lecture 1 - The Key Difference Between Quantum and Classical Mechanics} % (fold) \label{chap:Lec1}
In a traditional engineering curriculum, this course goes by the name of Materials Science And Engineering. The addition to the course is Computational Materials Science and AI/ML algorithms. The objective of the program is to have domain knowledge. The primary emphasis is on Materials Science And Engineering.
\begin{Reference}{Resources to look at for this course}
 Callister Jr. Rethwich \\
\textit{Materials Science And Engineering.}  \\
An Introduction (2018,10th Edition) \\
MIT 3.091 SC
3.091
\end{Reference}
The typical way this works is that engineers are given an objective, and materials engineering involves the usage of knowledge in choosing the right material to accomplish that objective in the most efficient way possible.

If you choose a particular material, you're choosing it because it has certain properties that will assist in the performance in the final product.


Structure is due to quantum mechanics(the realm of atoms)

In traditional engineering, we only deal with performance due to properties. The quantum mechanics involved is studied by physicists.

The reason traditional engineers do not need to know about the quantum mechanics underlying in the materials they use, is because they assume materials to be of a continuum. It is governed by calculus. It is deterministic.

Quantum mechanics is governed by probability. It is stochastic.

Thanks to advancements in computing and AI/ML, algorithms can be designed to create new materials and even predict the properties for these new materials.

The analysis of the structural differences in materials will be covered by Abhinav Sir.

Example text

% chapter  (end)

\end{document}
