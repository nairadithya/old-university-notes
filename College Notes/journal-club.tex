% Created 2024-08-13 Tue 08:30
% Intended LaTeX compiler: pdflatex
\documentclass[11pt]{report}
\usepackage[utf8]{inputenc}
\usepackage[T1]{fontenc}
\usepackage{graphicx}
\usepackage{longtable}
\usepackage{wrapfig}
\usepackage{rotating}
\usepackage[normalem]{ulem}
\usepackage{amsmath}
\usepackage{amssymb}
\usepackage{capt-of}
\usepackage{hyperref}
\input{preamble}
\author{Adithya Nair}
\date{\today}
\title{Journal Club}
\hypersetup{
 pdfauthor={Adithya Nair},
 pdftitle={Journal Club},
 pdfkeywords={},
 pdfsubject={},
 pdfcreator={Emacs 29.4 (Org mode 9.8)}, 
 pdflang={English}}
\begin{document}

\maketitle
\tableofcontents

\part{Real Analysis}
\label{sec:org6912abd}
TEXTBOOK: \href{file:///home/adithya/University-Latex-Notes/Journal Club/Analysis I - Tao.pdf}{Analysis 1 by Terence Tao}
\chapter{Natural Numbers}
\label{sec:orge00d2a5}
Numbers were built to count. A system for counting was made, and that system is the number system.
\begin{definition}
A natural number is an element of the set $\mathbb{N}$ of the set
\[
\mathbb{N} = \{0,1,2,3\cdots \}
\]
is obtained from 0 and counting forward indefinitely.
\end{definition}
\section{Peano Axioms}
\label{sec:org5b819e2}
We start with axioms to help clarify this.
\begin{itemize}
\item Axiom 1 : \(0 \in \mathbb{N}\)
\item Axiom 2: If \(n \in \mathbb{N}\),, then \(n++ \in \mathbb{N}\)
\item Axiom 3: 0 is not an increment of any other natural number \(n \in \mathbb{N}\)
\item Axiom 4: If \(n \neq m\), \(n++ \neq m++\)
\item Axiom 5: (Principle Of Mathematical Induction) Let \(P(n)\) be any property pertaining to a natural number \(n\). Suppose that \(P(0)\) is true, and suppose that whenever \(P(n)\) is true, \(P(n++)\) is also true. Then \(P(n)\) is true for every natural number.
\end{itemize}

We then make an assumption: That the set \(\mathbb{N}\) which satisfies these five axioms is called the set of natural numbers.
With these 5 axioms, we can construct sequences
\section{Recursive Definitions}
\label{sec:orge61edad}
\begin{prop}[Recursive Definitions]
Suppose for each natural number \(n\), we have some function \(f_n:\mathbb{N} \rightarrow \mathbb{N}\) from the natural numbers to the natural numbers. Then we can assign a unique natural number \(a_n\) to each natural number \(n\), such that \(a_0 = c\) and \(a_{n++} = f_n(a_n)\) for each natural number \(n\).
\end{prop}
\section{Addition}
\label{sec:orgfddcb7a}
\begin{definition}[Addition Of Natural Numbers]
Let n be a natural number. $(n \in N)$. To add zero to m, we define $0+m:=m$ Now suppose inductively that we have defined how to add $n$ to $m$. Then we can add $n++$ to $m$ by defining($n++$) + m := (n+m)++
\end{definition}

\begin{lemma}
For any natural number $n + 0=n$
\end{lemma}
\begin{proof}
We use induction,

The base case, n = 0,
\begin{align*}
n &= 0, 0 + 0 = 0 \\
n+0 &= n \\
(n++) + 0 &= (n+0)++ = (n++)
\end{align*}

Suppose inductively, that $n+0=n$,

For $n=n++$,
\begin{align*}
(n++) + 0 &= (n+0)++ \\
\text{We know that $n+0=n$} \\
(n++) + 0 &= (n++)
\end{align*}
\end{proof}

\begin{lemma}
For any natural numbers $n$ and $m$,
$$n + (m++) = (n+m)++$$
\end{lemma}
\begin{proof}
Inducting on $n$ while keeping $m$ fixed,
\begin{align*}
n &= 0, \\
0 + (m++) &= (0+m)++ \\
0 + (m++) &= (m++)
\end{align*}
This we know is true from the definition of addition $(0+m:=m)$

Suppose inductively, that $n+(m++) = (n+m)++$ is true.
For $n=(n++)$,
\begin{align*}
(n++) + (m++) &= ((n++)+m)++ &\text{From the definition of addition} \\
&=(n+(m++))++ \\
&=((n+m)++))++
\end{align*}
\end{proof}

Putting m = 0, we get \(n+1 = n++\)

\begin{prop}[Addition is commutative]
For any natural numbers $n$ and $m$, $n+m=m+n$
\end{prop}
\begin{proof}
We induct over $n$,
For the base case, $n=0$,

We must show that $m+0 = 0+m$
From the definition of addition, we have
$$0+m = m$$

As shown earlier, we have

$$m+0 = m$$

This is clearly true for $n=0$.

Now suppose inductively that $m+n = n+m$

For $n=n++$, we must show that $m+(n++) = (n++) + m$

We know from the definition of addition that,

$$(n++) + m := (m+n)++$$

And we proved earlier that,

$$m+(n++) = (m+n)++$$

Therefore,

$$m+(n++) = (n++)+m$$
\end{proof}
\begin{prop}[Addition is associative]
For any natural numbers, $a,b$ and $c$, we have $(a+b)+c = a+(b+c)$
\end{prop}
\begin{proof}
We take $(a+b)+n = a + (b+n)$

Inducting over n,

For $n=0$,

We have in the LHS,
\begin{align*}
&=(a+b)+0 &\text{Since $n+0 = n$}\\
&=a+b
\end{align*}

On the RHS,
\begin{align*}
&=a + (b+0) &\text{Since $n+0 = n$}\\
&=a + b
\end{align*}

Suppose inductively that $(a+b)+n = a+(b+n)$,

For $n=n++$,
We have to show that $(a+b)+(n++) = a+(b+(n++))$

On the LHS we have,

\begin{align*}
&=(a+b)+(n++) \\
&=(a+b+n)++ &\text{(From the lemma $m+(n++) = (m+n)++$)} \\
\end{align*}

On the RHS we have,

\begin{align*}
&=a+(b+(n++)) \\
&=a+(b+n)++ &\text{(From the lemma $m+(n++) = (m+n)++$)} \\
&=(a+b+n)++
\end{align*}

LHS = RHS
\end{proof}

\begin{prop}[Cancellation Law]
Let $a,b,c$ be natural numbers such that $a+b=a+c$. Then we have $b=c$.
\end{prop}
\begin{proof}
We have,
$$n+b=n+c$$

Inducting over n,
For the base case, $n=0$
\begin{align*}
0 + b &= 0 + c \\
b &= c
\end{align*}

Suppose inductively that $n+b=n+c$
For $n=n++$,
$$(n++)+b=(n++)+c$$
On the LHS
\begin{align*}
&=(n++) + b \\
&=(n+b)++
\end{align*}

On the RHS
\begin{align*}
&=(n++) + c \\
&=(n+c)++
\end{align*}

We know from the inductive hypothesis that,
$$\text{If} n+b = n+c, \text{then} b = c$$

Thus we have,
$$b++ = c++$$
\end{proof}

\begin{definition}[Positive natural number]
All numbers where,
\[
n \neq 0, n \in \mathbb{N}
\]
\end{definition}

\begin{lemma}
For every $a$, there exists a $b$ such that $b++ = a$
\end{lemma}

\begin{definition}[Order]
Let n and m be natural numbers we say that $n$ is greater than or equal to m, and write $n \geq m$ iff we have $n = m + a$ for some natural number $a$. We say that $n > m$ when $n \geq m$ and $n \neq m$
\end{definition}
\section{Strong Induction}
\label{sec:orgbbdddf9}
\begin{theorem}
Let \(m_0\) be a natural number, and let \(P(m)\) be a property pertaining to an arbitrary natural number \(m\). Suppose that for each \(m \geq m_0\), we have the following implication: if \(P(m')\) is true for all natural numbers \(m_0 \leq m' < m\), then \(P(m)\) is also true.(In particular this means that \(P(m_0\) is true, since in this case the hypothesis is vacuous.) Then we can conclude that \(P(m)\) is true for all natural numbers \(m \geq m_0\).
\end{theorem}
\begin{proof}
For a property \(Q(n)\), which is the property that \(P(m')\) is true for \(m_0 \leq m' < n\), then \(P(n)\) is true.

For \(Q(0)\),
\(0\) is either lesser than or equal to \(m_0\).

When \(0\) is lesser than \(m_0\),

This is vacuously true.

When \(0 = m_0\),
\end{proof}
\section{Induction Starting From The Base Case n}
\label{sec:org113550b}
\begin{proposition}
Let n be a natural number, and let P(m) be a property pertaining to the natural numbers such that whenever P(m) is true, P(m++) is true. Show that if P(n) is true, then P(m) is true for all m ≥ n. (This principle is sometimes referred to as the principle of induction starting from the base case n.)
\end{proposition}
\begin{proof}
Take a property \(P(n)\), \(m \geq n\)

Inducting over \(n\),
\end{proof}
\section{Multiplication}
\label{sec:org296d646}
\begin{definition}
Let \(m\) be a natural number. To multiply zero to \(m\), we define \(0 \times m := 0\). Now suppose inductively that we have defined how to multiply \(n\) to \(m\). Then we can multiply \(n++\) to \(m\) by defining \((n++) \times m := (n \times m) + m\)
\end{definition}
\begin{lemma}
Prove that multiplication is commutative
\end{lemma}
\section{Exercise}
\label{sec:orgb8c9171}
\begin{enumerate}
\item Prove the identity \((a+b)^2 = a^2 + 2ab + b^2\)
\label{sec:org6fd7acd}
\item (Euclid's division lemma)
\label{sec:orgcf77075}
\item Backward Induction
\label{sec:org1728b90}
\(m \in \mathbb{N}, P(m), P(m++) \implies P(m)\), Suppose \(P(n)\) is true, then \(P(m) \forall m \le n\)
For the base case, n = 0,
P(0) \implies P(0), so it's true.

For the inductive step, supposing \(Q(n)\) is true,
\item Strong induction
\label{sec:org14d1129}
\item Distributive Law
\label{sec:org5f048db}
\item Multiplication
\label{sec:org1bb9e17}
\begin{enumerate}
\item Cancellation Law
\label{sec:org918e70d}
\item Associativity
\label{sec:org624b130}
\item If a<b, and c is positive then ac<bc
\label{sec:org8dd571a}


\begin{center}
\begin{tabular}{rrr}
Exam 1 & Exam 2 & Mean\\
12 & 19 & 15.5\\
14 & 13 & 13.5\\
19 & 19 & 19\\
\end{tabular}
\end{center}
\end{enumerate}
\end{enumerate}
\end{document}
