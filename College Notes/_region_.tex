\message{ !name(real-analysis.tex)}% Created 2024-08-24 Sat 17:21
% Intended LaTeX compiler: pdflatex
\documentclass[11pt]{report}
\usepackage[utf8]{inputenc}
\usepackage[T1]{fontenc}
\usepackage{graphicx}
\usepackage{longtable}
\usepackage{wrapfig}
\usepackage{rotating}
\usepackage[normalem]{ulem}
\usepackage{amsmath}
\usepackage{amssymb}
\usepackage{capt-of}
\usepackage{hyperref}
%%%%%%%%%%%%%%%%%%%%%%%%%%%%%%%%%%%%%%%%%%%%%%%%%%%%%%%%%%%%%%%%%%%%%%%%%%%%%%%
%                                Basic Packages                               %
%%%%%%%%%%%%%%%%%%%%%%%%%%%%%%%%%%%%%%%%%%%%%%%%%%%%%%%%%%%%%%%%%%%%%%%%%%%%%%%

% Gives us multiple colors.
\usepackage[dvipsnames,pdftex]{xcolor}
\usepackage[tmargin=2cm,rmargin=1in,lmargin=1in,margin=0.85in,bmargin=2cm,footskip=.2in]{geometry}

% Lets us style link colors.
\usepackage{hyperref}
% Lets us import images and graphics.
\usepackage{graphicx}
% Lets us use figures in floating environments.
\usepackage{float}
% Lets us create multiple columns.
\usepackage{multicol}
% Gives us better math syntax.
\usepackage{amsmath,amsfonts,mathtools,amsthm,amssymb}
% Lets us strikethrough text.
\usepackage{cancel}
% Lets us edit the caption of a figure.
\usepackage{caption}
% Lets us import pdf directly in our tex code.
\usepackage{pdfpages}
% Lets us do algorithm stuff.
\usepackage[ruled,vlined,linesnumbered]{algorithm2e}
% Use a smiley face for our qed symbol.
\usepackage{tikzsymbols}
\renewcommand\qedsymbol{$\Laughey$}

\def\class{article}


%%%%%%%%%%%%%%%%%%%%%%%%%%%%%%%%%%%%%%%%%%%%%%%%%%%%%%%%%%%%%%%%%%%%%%%%%%%%%%%
%                                Basic Settings                               %
%%%%%%%%%%%%%%%%%%%%%%%%%%%%%%%%%%%%%%%%%%%%%%%%%%%%%%%%%%%%%%%%%%%%%%%%%%%%%%%

%%%%%%%%%%%%%
%  Symbols  %
%%%%%%%%%%%%%

\let\implies\Rightarrow
\let\impliedby\Leftarrow
\let\iff\Leftrightarrow
\let\epsilon\varepsilon

%%%%%%%%%%%%
%  Tables  %
%%%%%%%%%%%%

\setlength{\tabcolsep}{5pt}
\renewcommand\arraystretch{1.5}

%%%%%%%%%%%%%%
%  SI Unitx  %
%%%%%%%%%%%%%%

\usepackage{siunitx}
\sisetup{locale = FR}

%%%%%%%%%%
%  TikZ  %
%%%%%%%%%%

\usepackage[framemethod=TikZ]{mdframed}
\usepackage{tikz}
\usepackage{tikz-cd}
\usepackage{tikzsymbols}

\usetikzlibrary{intersections, angles, quotes, calc, positioning}
\usetikzlibrary{arrows.meta}

\tikzset{
  force/.style={thick, {Circle[length=2pt]}-stealth, shorten <=-1pt}
}

%%%%%%%%%%%%%%%
%  PGF Plots  %
%%%%%%%%%%%%%%%

\usepackage{pgfplots}
\pgfplotsset{compat=1.13}

%%%%%%%%%%%%%%%%%%%%%%%
%  Center Title Page  %
%%%%%%%%%%%%%%%%%%%%%%%

\usepackage{titling}
\renewcommand\maketitlehooka{\null\mbox{}\vfill}
\renewcommand\maketitlehookd{\vfill\null}

%%%%%%%%%%%%%%%%%%%%%%%%%%%%%%%%%%%%%%%%%%%%%%%%%%%%%%%
%  Create a grey background in the middle of the PDF  %
%%%%%%%%%%%%%%%%%%%%%%%%%%%%%%%%%%%%%%%%%%%%%%%%%%%%%%%

\usepackage{eso-pic}
\newcommand\definegraybackground{
  \definecolor{reallylightgray}{HTML}{FAFAFA}
  \AddToShipoutPicture{
    \ifthenelse{\isodd{\thepage}}{
      \AtPageLowerLeft{
        \put(\LenToUnit{\dimexpr\paperwidth-222pt},0){
          \color{reallylightgray}\rule{222pt}{297mm}
        }
      }
    }
    {
      \AtPageLowerLeft{
        \color{reallylightgray}\rule{222pt}{297mm}
      }
    }
  }
}

%%%%%%%%%%%%%%%%%%%%%%%%
%  Modify Links Color  %
%%%%%%%%%%%%%%%%%%%%%%%%

\hypersetup{
  % Enable highlighting links.
  colorlinks,
  % Change the color of links to blue.
  linkcolor=blue,
  % Change the color of citations to black.
  citecolor={black},
  % Change the color of url's to blue with some black.
  urlcolor={blue!80!black}
}

%%%%%%%%%%%%%%%%%%
% Fix WrapFigure %
%%%%%%%%%%%%%%%%%%

\newcommand{\wrapfill}{\par\ifnum\value{WF@wrappedlines}>0
    \parskip=0pt
    \addtocounter{WF@wrappedlines}{-1}%
    \null\vspace{\arabic{WF@wrappedlines}\baselineskip}%
    \WFclear
\fi}

%%%%%%%%%%%%%%%%%
% Multi Columns %
%%%%%%%%%%%%%%%%%

\let\multicolmulticols\multicols
\let\endmulticolmulticols\endmulticols

\RenewDocumentEnvironment{multicols}{mO{}}
{%
  \ifnum#1=1
    #2%
  \else % More than 1 column
    \multicolmulticols{#1}[#2]
  \fi
}
{%
  \ifnum#1=1
\else % More than 1 column
  \endmulticolmulticols
\fi
}

\newlength{\thickarrayrulewidth}
\setlength{\thickarrayrulewidth}{5\arrayrulewidth}


%%%%%%%%%%%%%%%%%%%%%%%%%%%%%%%%%%%%%%%%%%%%%%%%%%%%%%%%%%%%%%%%%%%%%%%%%%%%%%%
%                           School Specific Commands                          %
%%%%%%%%%%%%%%%%%%%%%%%%%%%%%%%%%%%%%%%%%%%%%%%%%%%%%%%%%%%%%%%%%%%%%%%%%%%%%%%

%%%%%%%%%%%%%%%%%%%%%%%%%%%
%  Initiate New Counters  %
%%%%%%%%%%%%%%%%%%%%%%%%%%%

\newcounter{lecturecounter}

%%%%%%%%%%%%%%%%%%%%%%%%%%
%  Helpful New Commands  %
%%%%%%%%%%%%%%%%%%%%%%%%%%

\makeatletter

\newcommand\resetcounters{
  % Reset the counters for subsection, subsubsection and the definition
  % all the custom environments.
  \setcounter{subsection}{0}
  \setcounter{subsubsection}{0}
  \setcounter{paragraph}{0}
  \setcounter{subparagraph}{0}
  \setcounter{theorem}{0}
  \setcounter{claim}{0}
  \setcounter{corollary}{0}
  \setcounter{lemma}{0}
  \setcounter{exercise}{0}

  \@ifclasswith\class{nocolor}{
    \setcounter{definition}{0}
  }{}
}

%%%%%%%%%%%%%%%%%%%%%
%  Lecture Command  %
%%%%%%%%%%%%%%%%%%%%%

\usepackage{xifthen}

% EXAMPLE:
% 1. \lesson{Oct 17 2022 Mon (08:46:48)}{Lecture Title}
% 2. \lesson[4]{Oct 17 2022 Mon (08:46:48)}{Lecture Title}
% 3. \lesson{Oct 17 2022 Mon (08:46:48)}{}
% 4. \lesson[4]{Oct 17 2022 Mon (08:46:48)}{}
% Parameters:
% 1. (Optional) Lesson number.
% 2. Time and date of lecture.
% 3. Lecture Title.
\def\@lesson{}
\newcommand\lesson[3][\arabic{lecturecounter}]{
  % Add 1 to the lecture counter.
  \addtocounter{lecturecounter}{1}

  % Set the section number to the lecture counter.
  \setcounter{section}{#1}
  \renewcommand\thesubsection{#1.\arabic{subsection}}

  % Reset the counters.
  \resetcounters

  % Check if user passed the lecture title or not.
  \ifthenelse{\isempty{#3}}{
    \def\@lesson{Lecture \arabic{lecturecounter}}
  }{
    \def\@lesson{Lecture \arabic{lecturecounter}: #3}
  }

  % Display the information like the following:
  %                                                  Oct 17 2022 Mon (08:49:10)
  % ---------------------------------------------------------------------------
  % Lecture 1: Lecture Title
  \hfill\small{#2}
  \hrule
  \vspace*{-0.3cm}
  \section*{\@lesson}
  \addcontentsline{toc}{section}{\@lesson}
}

%%%%%%%%%%%%%%%%%%%%
%  Import Figures  %
%%%%%%%%%%%%%%%%%%%%

\usepackage{import}
\pdfminorversion=7

% EXAMPLE:
% 1. \incfig{limit-graph}
% 2. \incfig[0.4]{limit-graph}
% Parameters:
% 1. The figure name. It should be located in figures/NAME.tex_pdf.
% 2. (Optional) The width of the figure. Example: 0.5, 0.35.
\newcommand\incfig[2][1]{%
  \def\svgwidth{#1\columnwidth}
  \import{./figures/}{#2.pdf_tex}
}

\begingroup\expandafter\expandafter\expandafter\endgroup
\expandafter\ifx\csname pdfsuppresswarningpagegroup\endcsname\relax
\else
  \pdfsuppresswarningpagegroup=1\relax
\fi

%%%%%%%%%%%%%%%%%
% Fancy Headers %
%%%%%%%%%%%%%%%%%

\usepackage{fancyhdr}

% Force a new page.

\newcommand\forcenewpage{\clearpage\mbox{~}\clearpage\newpage}
\newcommand\createintro{
  \pagestyle{fancy}
  \fancyhead{}
  \fancyhead[C]{23PHY114}
  \fancyfoot[L]{Adithya Nair}
  \fancyfoot[R]{AID23002}
  % Create a new page.
}
  \newpage
\makeatother

%%%%%%%%%%%%%%%%%%%%%%%%%%%%%%%%%%%%%%%%%%%%%%%%%%%%%%%%%%%%%%%%%%%%%%%%%%%%%%%
%                               Custom Commands                               %
%%%%%%%%%%%%%%%%%%%%%%%%%%%%%%%%%%%%%%%%%%%%%%%%%%%%%%%%%%%%%%%%%%%%%%%%%%%%%%%

%%%%%%%%%%%%
%  Circle  %
%%%%%%%%%%%%

\newcommand*\circled[1]{\tikz[baseline=(char.base)]{
  \node[shape=circle,draw,inner sep=1pt] (char) {#1};}
}

%%%%%%%%%%%%%%%%%%%
%  Todo Commands  %
%%%%%%%%%%%%%%%%%%%

\usepackage{xargs}
\usepackage[colorinlistoftodos]{todonotes}

\makeatletter

\@ifclasswith\class{working}{
  \newcommandx\unsure[2][1=]{\todo[linecolor=red,backgroundcolor=red!25,bordercolor=red,#1]{#2}}
  \newcommandx\change[2][1=]{\todo[linecolor=blue,backgroundcolor=blue!25,bordercolor=blue,#1]{#2}}
  \newcommandx\info[2][1=]{\todo[linecolor=OliveGreen,backgroundcolor=OliveGreen!25,bordercolor=OliveGreen,#1]{#2}}
  \newcommandx\improvement[2][1=]{\todo[linecolor=Plum,backgroundcolor=Plum!25,bordercolor=Plum,#1]{#2}}

  \newcommand\listnotes{
    \newpage
    \listoftodos[Notes]
  }
}{
  \newcommandx\unsure[2][1=]{}
  \newcommandx\change[2][1=]{}
  \newcommandx\info[2][1=]{}
  \newcommandx\improvement[2][1=]{}

  \newcommand\listnotes{}
}

\makeatother

%%%%%%%%%%%%%
%  Correct  %
%%%%%%%%%%%%%

% EXAMPLE:
% 1. \correct{INCORRECT}{CORRECT}
% Parameters:
% 1. The incorrect statement.
% 2. The correct statement.
\definecolor{correct}{HTML}{009900}
\newcommand\correct[2]{{\color{red}{#1 }}\ensuremath{\to}{\color{correct}{ #2}}}


%%%%%%%%%%%%%%%%%%%%%%%%%%%%%%%%%%%%%%%%%%%%%%%%%%%%%%%%%%%%%%%%%%%%%%%%%%%%%%%
%                                 Environments                                %
%%%%%%%%%%%%%%%%%%%%%%%%%%%%%%%%%%%%%%%%%%%%%%%%%%%%%%%%%%%%%%%%%%%%%%%%%%%%%%%

\usepackage{varwidth}
\usepackage{thmtools}
\usepackage[most,many,breakable]{tcolorbox}

\tcbuselibrary{theorems,skins,hooks}
\usetikzlibrary{arrows,calc,shadows.blur}

%%%%%%%%%%%%%%%%%%%
%  Define Colors  %
%%%%%%%%%%%%%%%%%%%

\definecolor{myblue}{RGB}{45, 111, 177}
\definecolor{mygreen}{RGB}{56, 140, 70}
\definecolor{myred}{RGB}{199, 68, 64}
\definecolor{mypurple}{RGB}{197, 92, 212}

\definecolor{definition}{HTML}{228b22}
\definecolor{theorem}{HTML}{00007B}
\definecolor{example}{HTML}{2A7F7F}
\definecolor{definition}{HTML}{228b22}
\definecolor{prop}{HTML}{191971}
\definecolor{lemma}{HTML}{983b0f}
\definecolor{exercise}{HTML}{88D6D1}

\colorlet{definition}{mygreen!85!black}
\colorlet{claim}{mygreen!85!black}
\colorlet{corollary}{mypurple!85!black}
\colorlet{proof}{theorem}

%%%%%%%%%%%%%%%%%%%%%%%%%%%%%%%%%%%%%%%%%%%%%%%%%%%%%%%%%
%  Create Environments Styles Based on Given Parameter  %
%%%%%%%%%%%%%%%%%%%%%%%%%%%%%%%%%%%%%%%%%%%%%%%%%%%%%%%%%

\mdfsetup{skipabove=1em,skipbelow=0em}

%%%%%%%%%%%%%%%%%%%%%%
%  Helpful Commands  %
%%%%%%%%%%%%%%%%%%%%%%

% EXAMPLE:
% 1. \createnewtheoremstyle{thmdefinitionbox}{}{}
% 2. \createnewtheoremstyle{thmtheorembox}{}{}
% 3. \createnewtheoremstyle{thmproofbox}{qed=\qedsymbol}{
%       rightline=false, topline=false, bottomline=false
%    }
% Parameters:
% 1. Theorem name.
% 2. Any extra parameters to pass directly to declaretheoremstyle.
% 3. Any extra parameters to pass directly to mdframed.
\newcommand\createnewtheoremstyle[3]{
  \declaretheoremstyle[
  headfont=\bfseries\sffamily, bodyfont=\normalfont, #2,
  mdframed={
    #3,
  },
  ]{#1}
}

% EXAMPLE:
% 1. \createnewcoloredtheoremstyle{thmdefinitionbox}{definition}{}{}
% 2. \createnewcoloredtheoremstyle{thmexamplebox}{example}{}{
%       rightline=true, leftline=true, topline=true, bottomline=true
%     }
% 3. \createnewcoloredtheoremstyle{thmproofbox}{proof}{qed=\qedsymbol}{backgroundcolor=white}
% Parameters:
% 1. Theorem name.
% 2. Color of theorem.
% 3. Any extra parameters to pass directly to declaretheoremstyle.
% 4. Any extra parameters to pass directly to mdframed.
\newcommand\createnewcoloredtheoremstyle[4]{
  \declaretheoremstyle[
  headfont=\bfseries\sffamily\color{#2}, bodyfont=\normalfont, #3,
  mdframed={
    linewidth=2pt,
    rightline=false, leftline=true, topline=false, bottomline=false,
    linecolor=#2, backgroundcolor=#2!5, #4,
  },
  ]{#1}
}

%%%%%%%%%%%%%%%%%%%%%%%%%%%%%%%%%%%
%  Create the Environment Styles  %
%%%%%%%%%%%%%%%%%%%%%%%%%%%%%%%%%%%

\makeatletter
\@ifclasswith\class{nocolor}{
  % Environments without color.

  \createnewtheoremstyle{thmdefinitionbox}{}{}
  \createnewtheoremstyle{thmtheorembox}{}{}
  \createnewtheoremstyle{thmexamplebox}{}{}
  \createnewtheoremstyle{thmclaimbox}{}{}
  \createnewtheoremstyle{thmcorollarybox}{}{}
  \createnewtheoremstyle{thmpropbox}{}{}
  \createnewtheoremstyle{thmlemmabox}{}{}
  \createnewtheoremstyle{thmexercisebox}{}{}
  \createnewtheoremstyle{thmdefinitionbox}{}{}
  \createnewtheoremstyle{thmquestionbox}{}{}
  \createnewtheoremstyle{thmsolutionbox}{}{}

  \createnewtheoremstyle{thmproofbox}{qed=\qedsymbol}{}
  \createnewtheoremstyle{thmexplanationbox}{}{}
}{
  % Environments with color.

  \createnewcoloredtheoremstyle{thmdefinitionbox}{definition}{}{}
  \createnewcoloredtheoremstyle{thmtheorembox}{theorem}{}{}
  \createnewcoloredtheoremstyle{thmexamplebox}{example}{}{
    rightline=true, leftline=true, topline=true, bottomline=true
  }
  \createnewcoloredtheoremstyle{thmclaimbox}{claim}{}{}
  \createnewcoloredtheoremstyle{thmcorollarybox}{corollary}{}{}
  \createnewcoloredtheoremstyle{thmpropbox}{prop}{}{}
  \createnewcoloredtheoremstyle{thmlemmabox}{lemma}{}{}
  \createnewcoloredtheoremstyle{thmexercisebox}{exercise}{}{}

  \createnewcoloredtheoremstyle{thmproofbox}{proof}{qed=\qedsymbol}{backgroundcolor=white}
  \createnewcoloredtheoremstyle{thmexplanationbox}{example}{qed=\qedsymbol}{backgroundcolor=white}
}
\makeatother

%%%%%%%%%%%%%%%%%%%%%%%%%%%%%
%  Create the Environments  %
%%%%%%%%%%%%%%%%%%%%%%%%%%%%%

\declaretheorem[numberwithin=section, style=thmtheorembox,     name=Theorem]{theorem}
\declaretheorem[numbered=no,          style=thmexamplebox,     name=Example]{example}
\declaretheorem[numberwithin=section, style=thmclaimbox,       name=Claim]{claim}
\declaretheorem[numberwithin=section, style=thmcorollarybox,   name=Corollary]{corollary}
\declaretheorem[numberwithin=section, style=thmpropbox,        name=Proposition]{prop}
\declaretheorem[numberwithin=section, style=thmlemmabox,       name=Lemma]{lemma}
\declaretheorem[numberwithin=section, style=thmexercisebox,    name=Exercise]{exercise}
\declaretheorem[numbered=no,          style=thmproofbox,       name=Proof]{replacementproof}
\declaretheorem[numbered=no,          style=thmexplanationbox, name=Proof]{expl}

\makeatletter
\@ifclasswith\class{nocolor}{
  % Environments without color.

  \newtheorem*{note}{Note}

  \declaretheorem[numberwithin=section, style=thmdefinitionbox, name=Definition]{definition}
  \declaretheorem[numberwithin=section, style=thmquestionbox,   name=Question]{question}
  \declaretheorem[numberwithin=section, style=thmsolutionbox,   name=Solution]{solution}
}{
  % Environments with color.

  \newtcbtheorem[number within=section]{Definition}{Definition}{
    enhanced,
    before skip=2mm,
    after skip=2mm,
    colback=red!5,
    colframe=red!80!black,
    colbacktitle=red!75!black,
    boxrule=0.5mm,
    attach boxed title to top left={
      xshift=1cm,
      yshift*=1mm-\tcboxedtitleheight
    },
    varwidth boxed title*=-3cm,
    boxed title style={
      interior engine=empty,
      frame code={
        \path[fill=tcbcolback]
        ([yshift=-1mm,xshift=-1mm]frame.north west)
        arc[start angle=0,end angle=180,radius=1mm]
        ([yshift=-1mm,xshift=1mm]frame.north east)
        arc[start angle=180,end angle=0,radius=1mm];
        \path[left color=tcbcolback!60!black,right color=tcbcolback!60!black,
        middle color=tcbcolback!80!black]
        ([xshift=-2mm]frame.north west) -- ([xshift=2mm]frame.north east)
        [rounded corners=1mm]-- ([xshift=1mm,yshift=-1mm]frame.north east)
        -- (frame.south east) -- (frame.south west)
        -- ([xshift=-1mm,yshift=-1mm]frame.north west)
        [sharp corners]-- cycle;
      },
    },
    fonttitle=\bfseries,
    title={#2},
    #1
  }{def}

  \NewDocumentEnvironment{definition}{O{}O{}}
    {\begin{Definition}{#1}{#2}}{\end{Definition}}

  \newtcolorbox{note}[1][]{%
    enhanced jigsaw,
    colback=gray!20!white,%
    colframe=gray!80!black,
    size=small,
    boxrule=1pt,
    title=\textbf{Note:-},
    halign title=flush center,
    coltitle=black,
    breakable,
    drop shadow=black!50!white,
    attach boxed title to top left={xshift=1cm,yshift=-\tcboxedtitleheight/2,yshifttext=-\tcboxedtitleheight/2},
    minipage boxed title=1.5cm,
    boxed title style={%
      colback=white,
      size=fbox,
      boxrule=1pt,
      boxsep=2pt,
      underlay={%
        \coordinate (dotA) at ($(interior.west) + (-0.5pt,0)$);
        \coordinate (dotB) at ($(interior.east) + (0.5pt,0)$);
        \begin{scope}
          \clip (interior.north west) rectangle ([xshift=3ex]interior.east);
          \filldraw [white, blur shadow={shadow opacity=60, shadow yshift=-.75ex}, rounded corners=2pt] (interior.north west) rectangle (interior.south east);
        \end{scope}
        \begin{scope}[gray!80!black]
          \fill (dotA) circle (2pt);
          \fill (dotB) circle (2pt);
        \end{scope}
      },
    },
    #1,
  }

  \newtcbtheorem{Question}{Question}{enhanced,
    breakable,
    colback=white,
    colframe=myblue!80!black,
    attach boxed title to top left={yshift*=-\tcboxedtitleheight},
    fonttitle=\bfseries,
    title=\textbf{Question:-},
    boxed title size=title,
    boxed title style={%
      sharp corners,
      rounded corners=northwest,
      colback=tcbcolframe,
      boxrule=0pt,
    },
    underlay boxed title={%
      \path[fill=tcbcolframe] (title.south west)--(title.south east)
      to[out=0, in=180] ([xshift=5mm]title.east)--
      (title.center-|frame.east)
      [rounded corners=\kvtcb@arc] |-
      (frame.north) -| cycle;
    },
    #1
  }{def}

  \NewDocumentEnvironment{question}{O{}O{}}
  {\begin{Question}{#1}{#2}}{\end{Question}}

  \newtcolorbox{Solution}{enhanced,
    breakable,
    colback=white,
    colframe=mygreen!80!black,
    attach boxed title to top left={yshift*=-\tcboxedtitleheight},
    title=\textbf{Solution:-},
    boxed title size=title,
    boxed title style={%
      sharp corners,
      rounded corners=northwest,
      colback=tcbcolframe,
      boxrule=0pt,
    },
    underlay boxed title={%
      \path[fill=tcbcolframe] (title.south west)--(title.south east)
      to[out=0, in=180] ([xshift=5mm]title.east)--
      (title.center-|frame.east)
      [rounded corners=\kvtcb@arc] |-
      (frame.north) -| cycle;
    },
  }

  \NewDocumentEnvironment{solution}{O{}O{}}
  {\vspace{-10pt}\begin{Solution}{#1}{#2}}{\end{Solution}}
}
\makeatother

%%%%%%%%%%%%%%%%%%%%%%%%%%%%
%  Edit Proof Environment  %
%%%%%%%%%%%%%%%%%%%%%%%%%%%%

\renewenvironment{proof}[1][\proofname]{\vspace{-10pt}\begin{replacementproof}}{\end{replacementproof}}
\newenvironment{explanation}[1][\proofname]{\vspace{-10pt}\begin{expl}}{\end{expl}}

\theoremstyle{definition}

\newtheorem*{notation}{Notation}
\newtheorem*{previouslyseen}{As previously seen}
\newtheorem*{problem}{Problem}
\newtheorem*{observe}{Observe}
\newtheorem*{property}{Property}
\newtheorem*{intuition}{Intuition}

%%%%%%%%%%%%%%%%%%%%%%%%%%%%%%%%%%%%%%%%%%%%%%%%%%%%%%%%%%%%%%%
%                 Code Highlighting                           %
%%%%%%%%%%%%%%%%%%%%%%%%%%%%%%%%%%%%%%%%%%%%%%%%%%%%%%%%%%%%%%%
\usepackage{listings}
\lstset{
language=Octave,
backgroundcolor=\color{white},   % choose the background color; you must add \usepackage{color} or \usepackage{xcolor}
basicstyle=\footnotesize\ttfamily,        % the size of the fonts that are used for the code
breakatwhitespace=false,         % sets if automatic breaks should only happen at whitespace
breaklines=true,                 % sets automatic line breaking
captionpos=b,                    % sets the caption-position to bottom
commentstyle=\color{gray},    % comment style
%escapeinside={\%*}{*)},          % if you want to add LaTeX within your code
extendedchars=true,            % lets you use non-ASCII characters; for 8-bits encodings only, does not work with UTF-8
frame=single,                    % adds a frame around the code
% frameround=fttt,
keepspaces=true,                 % keeps spaces in text, useful for keeping indentation of code (possibly needs columns=flexible)
columns=flexible,
classoffset=0,
keywordstyle=\color{RoyalBlue},       % keyword style
deletekeywords={function,endfunction, if,endif},
classoffset=1,
morekeywords={function,endfunction, if,endif},
keywordstyle=\bf\color{Red},       % keyword style
classoffset=2,
morekeywords={persistent},            % if you want to add more keywords to the set
keywordstyle=\bf\color{ForestGreen},       % keyword style
classoffset=0,
literate=
{/}{{{\color{Mahogany}/}}}1
{*}{{{\color{Mahogany}*}}}1
{.*}{{{\color{Mahogany}.*}}}2
{+}{{{\color{Mahogany}+{}}}}1
{=}{{{\bf\color{Mahogany}=}}}1
{-}{{{\color{Mahogany}-}}}1
{[}{{{\bf\color{RedOrange}[}}}1
{]}{{{\bf\color{RedOrange}]}}}1
{ç}{{\c{c}}}1 % Cedilha
{á}{{\'{a}}}1 % Acentos agudos
{é}{{\'{e}}}1
{í}{{\'{i}}}1
{ó}{{\'{o}}}1
{ú}{{\'{u}}}1
{â}{{\^{a}}}1 % Acentos circunflexos
{ê}{{\^{e}}}1
{î}{{\^{i}}}1
{ô}{{\^{o}}}1
{û}{{\^{u}}}1
{à}{{\`{a}}}1 % Acentos graves
{è}{{\`{e}}}1
{ì}{{\`{i}}}1
{ò}{{\`{o}}}1
{ù}{{\`{u}}}1
{ã}{{\~{a}}}1 % Tils
{ẽ}{{\~{e}}}1
{ĩ}{{\~{i}}}1
{õ}{{\~{o}}}1
{ũ}{{\~{u}}}1,
numbers=left,                    % where to put the line-numbers; possible values are (none, left, right)
numbersep=6pt,                   % how far the line-numbers are from the code
numberstyle=\tiny\color{gray}, % the style that is used for the line-numbers
rulecolor=\color{black},         % if not set, the frame-color may be changed on line-breaks within not-black text (e.g. comments (green here))
showspaces=false,                % show spaces everywhere adding particular underscores; it overrides 'showstringspaces'
showstringspaces=false,          % underline spaces within strings only
showtabs=false,                  % show tabs within strings adding particular underscores
stepnumber=1,                    % the step between two line-numbers. If it's 1, each line will be numbered
stringstyle=\color{purple},     % string literal style
tabsize=2,                       % sets default tabsize to 2 spaces
}


\author{Adithya Nair}
\date{\today}
\title{Journal Club}
\hypersetup{
	pdfauthor={Adithya Nair},
	pdftitle={Journal Club},
	pdfsubject={},
	pdflang={English}}
\begin{document}

\message{ !name(real-analysis.tex) !offset(-3) }


\maketitle
\tableofcontents

\part{Real Analysis}
\label{sec:org654bad8}
TEXTBOOK: \href{file:///home/adithya/University-Latex-Notes/Journal Club/Analysis I - Tao.pdf}{Analysis 1 by Terence Tao}
\chapter{Natural Numbers}
\label{sec:org904821b}
Numbers were built to count. A system for counting was made, and that system is the number system.
\begin{definition}
	A natural number is an element of the set $\mathbb{N}$ of the set
	\[
		\mathbb{N} = \{0,1,2,3\cdots \}
	\]
	is obtained from 0 and counting forward indefinitely.
\end{definition}
\section{Peano Axioms}
\label{sec:orgcc5f6c9}
We start with axioms to help clarify this.
\begin{itemize}
	\item Axiom 1 : \(0 \in \mathbb{N}\)
	\item Axiom 2: If \(n \in \mathbb{N}\),, then \(n++ \in \mathbb{N}\)
	\item Axiom 3: 0 is not an increment of any other natural number \(n \in \mathbb{N}\)
	\item Axiom 4: If \(n \neq m\), \(n++ \neq m++\)
	\item Axiom 5: (Principle Of Mathematical Induction) Let \(P(n)\) be any property pertaining to a natural number \(n\). Suppose that \(P(0)\) is true, and suppose that whenever \(P(n)\) is true, \(P(n++)\) is also true. Then \(P(n)\) is true for every natural number.
\end{itemize}

We then make an assumption: That the set \(\mathbb{N}\) which satisfies these five axioms is called the set of natural numbers.
With these 5 axioms, we can construct sequences
\section{Recursive Definitions}
\label{sec:org1dae040}
\begin{prop}[Recursive Definitions]
	Suppose for each natural number $n$, we have some function $f_n:\mathbb{N} \rightarrow \mathbb{N}$ from the natural numbers to the natural numbers. Then we can assign a unique natural number $a_n$ to each natural number $n$, such that $a_0 = c$ and $a_{n++} = f_n(a_n)$ for each natural number $n$.
\end{prop}
\section{Addition}
\label{sec:orgabde90c}
\begin{definition}[Addition Of Natural Numbers]
	Let n be a natural number. $(n \in N)$. To add zero to m, we define $0+m:=m$ Now suppose inductively that we have defined how to add $n$ to $m$. Then we can add $n++$ to $m$ by defining($n++$) + m := (n+m)++
\end{definition}

\begin{lemma}
	For any natural number $n + 0=n$
\end{lemma}
\begin{proof}
	We use induction,

	The base case, n = 0,
	\begin{align*}
		n         & = 0, 0 + 0 = 0    \\
		n+0       & = n               \\
		(n++) + 0 & = (n+0)++ = (n++)
	\end{align*}

	Suppose inductively, that $n+0=n$,

	For $n=n++$,
	\begin{align*}
		(n++) + 0 & = (n+0)++       \\
		\text{We know that $n+0=n$} \\
		(n++) + 0 & = (n++)
	\end{align*}
\end{proof}

\begin{lemma}
	For any natural numbers $n$ and $m$,
	$$n + (m++) = (n+m)++$$
\end{lemma}
\begin{proof}
	Inducting on $n$ while keeping $m$ fixed,
	\begin{align*}
		n         & = 0,      \\
		0 + (m++) & = (0+m)++ \\
		0 + (m++) & = (m++)
	\end{align*}
	This we know is true from the definition of addition $(0+m:=m)$

	Suppose inductively, that $n+(m++) = (n+m)++$ is true.
	For $n=(n++)$,
	\begin{align*}
		(n++) + (m++) & = ((n++)+m)++ & \text{From the definition of addition} \\
		              & =(n+(m++))++                                           \\
		              & =((n+m)++))++
	\end{align*}
\end{proof}

Putting m = 0, we get \(n+1 = n++\)

\begin{prop}[Addition is commutative]
	For any natural numbers $n$ and $m$, $n+m=m+n$
\end{prop}
\begin{proof}
	We induct over $n$,
	For the base case, $n=0$,

	We must show that $m+0 = 0+m$
	From the definition of addition, we have
	$$0+m = m$$

	As shown earlier, we have

	$$m+0 = m$$

	This is clearly true for $n=0$.

	Now suppose inductively that $m+n = n+m$

	For $n=n++$, we must show that $m+(n++) = (n++) + m$

	We know from the definition of addition that,

	$$(n++) + m := (m+n)++$$

	And we proved earlier that,

	$$m+(n++) = (m+n)++$$

	Therefore,

	$$m+(n++) = (n++)+m$$
\end{proof}
\begin{prop}[Addition is associative]
	For any natural numbers, $a,b$ and $c$, we have $(a+b)+c = a+(b+c)$
\end{prop}
\begin{proof}
	We take $(a+b)+n = a + (b+n)$

	Inducting over n,

	For $n=0$,

	We have in the LHS,
	\begin{align*}
		 & =(a+b)+0 & \text{Since $n+0 = n$} \\
		 & =a+b
	\end{align*}

	On the RHS,
	\begin{align*}
		 & =a + (b+0) & \text{Since $n+0 = n$} \\
		 & =a + b
	\end{align*}

	Suppose inductively that $(a+b)+n = a+(b+n)$,

	For $n=n++$,
	We have to show that $(a+b)+(n++) = a+(b+(n++))$

	On the LHS we have,

	\begin{align*}
		 & =(a+b)+(n++)                                               \\
		 & =(a+b+n)++   & \text{(From the lemma $m+(n++) = (m+n)++$)} \\
	\end{align*}

	On the RHS we have,

	\begin{align*}
		 & =a+(b+(n++))                                               \\
		 & =a+(b+n)++   & \text{(From the lemma $m+(n++) = (m+n)++$)} \\
		 & =(a+b+n)++
	\end{align*}

	LHS = RHS
\end{proof}

\begin{prop}[Cancellation Law]
	Let $a,b,c$ be natural numbers such that $a+b=a+c$. Then we have $b=c$.
\end{prop}
\begin{proof}
	We have,
	$$n+b=n+c$$

	Inducting over n,
	For the base case, $n=0$
	\begin{align*}
		0 + b & = 0 + c \\
		b     & = c
	\end{align*}

	Suppose inductively that $n+b=n+c$
	For $n=n++$,
	$$(n++)+b=(n++)+c$$
	On the LHS
	\begin{align*}
		 & =(n++) + b \\
		 & =(n+b)++
	\end{align*}

	On the RHS
	\begin{align*}
		 & =(n++) + c \\
		 & =(n+c)++
	\end{align*}

	We know from the inductive hypothesis that,
	$$\text{If} n+b = n+c, \text{then} b = c$$

	Thus we have,
	$$b++ = c++$$
\end{proof}

\begin{definition}[Positive natural number]
	All numbers where,
	\[
		n \neq 0, n \in \mathbb{N}
	\]
\end{definition}
\begin{prop}
	If $a$ is a positive natural number and $b$ is a natural number, then $a+b$ is positive.
\end{prop}
\begin{proof}
	Inducting over b,

	For $b$ = 0,
	\begin{align*}
		a+0 = a
	\end{align*}
	This proves the base case, since we know a is positive.

	Now, suppose inductively, that $(a+b)$ is positive.

	For $(a+(n++))$,
	\begin{align*}
		a+(n++) = (a+n)++
	\end{align*}
	We know from Axiom 3 that $n++ \neq 0$. Thus we close the inductive loop.
\end{proof}
\begin{lemma}
	For every $a$, there exists a unique $b$ such that $b++ = a$
\end{lemma}
\begin{proof}
	Proof by contradiction,
	Suppose that there are two different increments, $m++$, $n++$ that equal to $a$,

	We have,
	\begin{align*}
		m++ & = a \\
		n++ & = a
	\end{align*}

	Then we can say,
	\begin{align*}
		m++   & = n++                                \\
		m + 1 & = n+1                                \\
		m     & = n   & \text{(By Cancellation Law)}
	\end{align*}

	But we said that m and n are different numbers which increment to $a$.

	Therefore, we can conclude that there is only one number $b$ which increments to $a$
\end{proof}
\section{Order}
\label{sec:org9ec10b6}
\begin{definition}[Order]
	Let n and m be natural numbers we say that $n$ is greater than or equal to m, and write $n \geq m$ iff we have $n = m + a$ for some natural number $a$. We say that $n > m$ when $n \geq m$ and $n \neq m$
\end{definition}
\begin{prop}[Basic properties of order for natural numbers]
	Let $a,b,c$ be natural numbers then
	\begin{enumerate}
		\item (Order is reflexive) $a \geq a$
		\item (Order is transitive) If $a \geq b$ and $b \geq c$, then $a \geq c$
		\item (Order is antisymmetric) If $a \geq b$ and $b \geq a$ then $a=b$
		\item (Addition preserves order) $a \geq b$ if and only if $a+c \geq b+c$
		\item $a<b$ if and only if $a++ \leq b$
		\item $a<b$ if and only if $b= a+d$ for some positive number d.
	\end{enumerate}
\end{prop}
\begin{proof}
	\begin{enumerate}
		\item Proving order is reflexive, $a \geq a$

		      We know that,

		      $a = a + 0$

		      From the definition of order,
		      We can write that $a \geq b$ when $a = b + d$ where $d \in \mathbb{N}$

		      Thus $a \geq a$.

		\item Proving order is transitive, $a \geq b$ and $b \geq c$ then $a \geq c$

		      We write,

		      \begin{align*}
			      a & = b + d     \\
			      b & = c + e     \\
			      a & = c + e + d
		      \end{align*}
		      We can say that since $(e+d) \in \mathbb{N}$

		      We define $f := (e+d)$
		      Where $f \in \mathbb{N}$
		      \begin{align*}
			      a & = c + (f)
		      \end{align*}

		      Thus we can say,
		      $$\text{If } a \geq b, b \geq c \text{ then } a \geq c$$

		\item Proving order is antisymmetric, If $a \geq b$ and $b \geq a$ then $a=b$
		      We can say,
		      \begin{align*}
			      a = b + d \\
			      b = a + e \\
		      \end{align*}
		      Where $d,e \in \mathbb{N}$

		      \begin{align*}
			      a = (a + e) + d \\
			      b = (b + d) + e \\
		      \end{align*}

		      Then we can write,
		      \begin{align*}
			      a = a + (e + d) \\
			      b = b + (d + e) \\
		      \end{align*}

		      Then we can say that $(e+d)$ and $(d+e)$ are 0.

		      We know that if $a + b = 0$ then $a,b = 0$

		      Thus $d$ and $e$ are 0.
		      \begin{align*}
			      a = b + d \\
			      a = b
		      \end{align*}
		\item Proving $a < b$ if and only if $b =a+d$ for some positive number d
		      If $b = a+d$ where $d$ is a positive natural number, $d \neq 0$

		      Which means that $b \neq a + 0$ or $b \neq a$

		      This means that b is strictly greater than a

		      If $a<b$ then $a \geq b$ and $a \neq b$

		      So if $a \geq b$
		      Then,
		      \begin{align*}
			      a = b + d \\
		      \end{align*}
		      But,
		      \begin{align*}
			      a \neq b     \\
			      a \neq b + 0 \\
			      d \neq 0
		      \end{align*}
		      Thus d cannot be 0. $d$ can only be a positive natural number.
		\item Proving addition preserves order, $a \geq b$ if and only if $a + c \geq b + c$
		      Proving $a \geq b$ if $a + c \geq b + C$

		      Where $d \in \mathbb{N}$
		      \begin{align*}
			      a + c & = b + c + d &   & \text{By definition}       \\
			      a + c & = (b+d) + c &                                \\
			      a     & = (b+d)     &   & \text{By cancellation law} \\
			      a     & \geq b
		      \end{align*}
		      Proving $a + c \geq b +c$ if $a \geq b$

		      We know,
		      \begin{align*}
			      a = b + d \\
		      \end{align*}
		      Where $d \in \mathbb{N}$

		      We write a+c using what we know from above,
		      \begin{align*}
			      a + c   & = b + d + c   \\
			      a + c   & = b + c + d   \\
			      (a + c) & = (b + c) + d \\
			      a + c   & \geq b + c
		      \end{align*}

		\item Proving $a < b$ if and only if $a++ \leq b$
		      Proving $a < b$ if $a++ \leq b$

		      We can write,
		      \begin{align*}
			      a++       & = b + d & \text{Where $d \in \mathbb{N}$} \\
			      a++ + d   & = b                                       \\
			      a + (d++) & = b                                       \\
		      \end{align*}
		      Since from Axiom 3, we know that 0 is not an increment of any natural number, $(d++ \neq 0)$
		      Therefore,
		      \begin{align*}
			      a & < b
		      \end{align*}

	\end{enumerate}
\end{proof}
\begin{prop}[Trichotomy of order for natural numbers]
	Let $a$ and $b$ be natural numbers. Then exactly one of the following statements is true: $a<b, a=b or a>b$
\end{prop}
\begin{proof}
	First we show that no more than one of the statements is true.
	If $a<b$ then $a \neq b$ by definition. If $a>b$ then $a \neq b$ by definition. If $a>b$ and $a<b$ then $a=b$, which we proved earlier.

	Now to show that exactly one of these statements are true.
	We induct on a,

	When a = 0,
	We know that,
	\begin{align*}
		 & b & = 0 + b & (\forall b \in \mathbb{N}) \\
		 & b & \geq 0
	\end{align*}

	Suppose inductively that exactly one of the above statements are true for a and b.
	For a++,
	We take each statement. First for $a>b$
	\begin{align*}
		a     & > b                                                                  \\
		a     & = b + d                                                              \\
		(a++) & = (b + d)++                                                          \\
		(a++) & = b + d++                                                            \\
		(a++) & > b         & \text{If $d \in \mathbb{N}$ then $d++ \in \mathbb{N}$}
	\end{align*}
	For $a=b$
	\begin{align*}
		a     & = b     \\
		(a++) & = (b)++ \\
		(a++) & = b + 1 \\
		a     & > b     \\
	\end{align*}
	For $a<b$
	\begin{align*}
		a & <b            \\
		a + d = b         \\
		(a + d)++ = b++   \\
		(a++) + d = b++   \\
		(a++) + d = b + 1 \\
	\end{align*}
	We have two cases,
	If $d = 1$,
	Then by cancellation law
	$$ a++ = b $$
	If $d \neq 1$
	Then
	$$a++ < b$$
	But never both, which concludes the inductive loop.
\end{proof}
\section{Special Forms Of Induction}
\label{sec:orgd06d9c6}
\begin{prop}[Strong Principle Of Induction]
	Let $m_0$ be a natural number, and let $P(m)$ be a property pertaining to an arbitrary natural number $m$. Suppose that for each $m \geq m_0$, we have the following implication: if $P(m')$ is true for all natural numbers $m_0 \leq m' < m$, then $P(m)$ is also true.(In particular this means that $P(m_0$ is true, since in this case the hypothesis is vacuous.) Then we can conclude that $P(m)$ is true for all natural numbers $m \geq m_0$.
\end{prop}
\begin{proof}
	For a property $Q(n)$, which is the property that $P(m')$ is true for $m_0 \leq m < n$, then $P(n)$ is true... Then it is true $\forall m \geq m_0$

	For $Q(0)$, we can say that the statement is vacuous since the conditions are not satisfied for both when $m_{0} = 0$ and when $m_{0} <0$

	Suppose inductively that $Q(n)$ is true.
	Which means that

	Then for Q(n++)
\end{proof}
\begin{prop}[Backward Induction]
	Let $n$ be a natural number, and let $P(m)$ be a property pertaining to the natural numbers such that whenever $P(m++)$ is true, then $P(m)$. Suppose that $P(n)$ is also true. Prove that $P(m)$ is true for all natural numbers $m \leq n$.
\end{prop}
\begin{prop}[Induction starting from the base case $n$]
	Let n be a natural number, and let $P(m)$ be a property pertaining to the natural numbers such that whenever P(m) is true, P(m++) is true. Show that if P(n) is true, then P(m) is true for all m ≥ n. (This principle is sometimes referred to as the principle of induction starting from the base case n.)
\end{prop}
\begin{proof}
	Take a property $P(n)$, $m \geq n$

	Inducting over $n$,
\end{proof}
\section{Multiplication}
\label{sec:org572658a}
\begin{definition}[Multiplication]
	Let $m$ be a natural number. To multiply zero to $m$, we define $0 \times m := 0$. Now suppose inductively that we have defined how to multiply $n$ to $m$. Then we can multiply $n++$ to $m$ by defining $(n++) \times m := (n \times m) + m$
\end{definition}
We can say \(0 \times m = 0\), \(1 \times m = 0 + m\), \(2 \times m= 0 + m + m\) and so on.
\begin{lemma}
	Prove that multiplication is commutative
\end{lemma}
\begin{proof}
	We use the way we proved that addition is commutative as a blueprint.
	There are two things we need to prove first.
	\begin{enumerate}
		\item For any natural number, $n$, $n \times 0 = 0$
		\item For any natural numbers, $n$ and $m$, $n \times (m++) = (n \times m) + m$
	\end{enumerate}

	First we prove,
	For any natural number, $n$, $n \times 0 = n$
	We induct over $n$,
	For $n = 0$,
	$$0 \times 0 = 0$$

	Which is true from the definition

	Now suppose inductively, that $n \times 0 = 0$,
	For $(n++) \times 0$,
	From the definition we can write this as,
	\begin{align*}
		(n++) \times 0 & = (n \times 0) + 0 \\
		\text{We know that $n \times 0 = 0$}
		(n++) \times 0 & = 0 + 0            \\
		(n++) \times 0 & = 0
	\end{align*}
	Therefore, $$n \times 0 = n$$

	Now we prove,
	For any natural numbers, $n$ and $m$, $n \times (m++) = (n \times m) + m$
	We induct over $n$, (keeping $m$ fixed)

	For $n = 0$,
	We know from the definition for multiplication with zero that,
	\begin{align*}
		0 \times (m++) = 0                                    \\
		\text{We also know that}                              \\
		(m++) \times 0                  & = (m \times 0 ) + 0 \\
		(m++) \times 0                  & = 0                 \\
		(m++) \times 0 = 0 \times (m++) & = (0 \times m) + m
	\end{align*}

	Suppose inductively that $n \times (m++) = (n \times m) + m$
	For $n = (n++)$
	To prove $(n++) \times (m++) = ((n++) \times m) + m$,

	\begin{align*}
		(n++) \times (m++) & = (n \times (m++)) + m++            \\
		\text{We can rewrite RHS using the inductive hypothesis} \\
		(n++) \times (m++) & = ((n \times m) + m) + m++          \\
	\end{align*}
\end{proof}

For simplicity we now write \(a \times b\) as \(ab\)
\chapter{Set Theory}
\label{sec:org7b352c4}
We define first what a set is:

\begin{definition}[Sets]
	We define set A to be any unordered collection of objects. If $x$ is an object, we say that x is an element of A or $x \in A$ if x lies in the collection. Otherwise $x \in A$
\end{definition}

We start with some axioms:
\begin{enumerate}
	\item (Sets are objects) If $A$ is a set, then $A$ is also an object.
	      A side track about "Pure Set Theory" - This theory states that everything in the mathematical universe is a set. We can write 0 as {} or an empty set, 1 can be written as {0} and 2 as {0,1} and so on. Terence Tao argues that they are the 'cardinalities of the set.'
	\item (Equality of sets) Two sets A and B are equal, A = B, iff every element of A is an element of B. A = B, if and only if every element of $x$ of A also belongs to B, and every element $y$ of B belongs to A.
	\item (Empty set) There exists a set $\emptyset$ known as the empty set, which contains no elements. $x \notin \emptyset$
	      \begin{prop}[Partial Order]
		      If A \subseteq B, B \subseteq C \Rightarrow A \subseteq C
	      \end{prop}
	      \begin{proof}
		      If $x \in A$, then $x \in B$,
		      If $x \in B$, then $x \in C$,
		      Then $x \in A$, then $x \in C$

		      Thus, $A \subseteq C$
	      \end{proof}
	      \begin{lemma}[Single choice]
		      Let $A$ be a non-empty set. Then there exists an object $x$ such that $x \in A$
	      \end{lemma}
	      \begin{proof}
		      Proving by contradiction,
		      Suppose there is no object $x$ that belongs to A. For all $x$, we have $x \notin A$. We know from Axiom 3, that $x \notin \emptyset$

		      For the statement,
		      $$ x \in A \leftrightarrow x \in \emptyset $$

		      Is false both ways, which gives us the result true, which is a contradiction.

		      Thus we also prove that $\emptyset$ is unique.
	      \end{proof}
	\item (Pairwise Union) $A \cup B = \{x : x \in A \text{or} x \in B\}$
	      \begin{lemma}
		      $A \cup (B \cup C) = (A \cup B) \cup C$
	      \end{lemma}
	      \begin{proof}
		      Taking the left hand side,
		      We have $x \in A$ or $x \in (B \cup C)$. If we look to the right hand side, we have $x \in (A \cup B)$ or $ x \in C$
		      If we break the statement down further.
		      We have $x \in A$ or $x \in B$ or $x \in C$, and  on the right $x \in A$ or $x \in B$ or $x\in C$

		      The two statements are equivalent.
	      \end{proof}
	\item (Axiom Of Specification) A, $x \in A$, let P(x) be a property pertaining to $x$. Then there exists a set called $\{x \in A, P(x) \text{is true}\}$ whose elements are precisely the elements $x$ in A for which $P(x)$ is true.
	\item (Replacement) Let A be a set, for any object $x \in A$, and any object $y$, suppose we have a statement $P(x,y)$ pertaining to $x$ and $y$, such that for each $x \in A$ there is at most one $y$ for which $P(x,y)$ is true. Then there exists a set $\{y : P(x,y)$ is true for some $x \in A\}$
	\item (Infinity) There exists a set $\mathbb{N}$, whose elements are called natural numbers, as well as an object $0$ in $\mathbb{N}$, and an object $n++$ assigned to every natural number $n \in \mathbb{N}$ such that the Peano axioms hold.

	\item \textbf{Russel's Paradox} (Axiom Of Universal Specification) Suppose for every $x$ we have a property $P(x)$ pertaining to $x$, Then there exists a set $\{x : P(x)$ is true$\}$ such that for every object $y$:

	      $$ y \in \{ x : P(x) is true \} \Leftrightarrow P(y) \text{is true.}$$

	      There is an issue, let's say we have a set, where the property of the objects is that they themselves are sets.

	      Let's say we look at one of these sets,

	      \begin{align*}
		      \Omega & = \{ x: P(x) is true\} \\
		             & = \{x : x \notin x \}
	      \end{align*}

	\item (Regularity) If A is a  non-empty set, then there is at least one element $x$ of A which is either not a set or disjoint from A.
\end{enumerate}
\begin{definition}[Intersection Of Sets]
	$$A \cap B = \{x: x \in A \text{ and } x \in B \}$$
\end{definition}
\end{document}

\message{ !name(real-analysis.tex) !offset(-634) }
