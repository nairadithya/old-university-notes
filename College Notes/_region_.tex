\message{ !name(real-analysis.tex)}% Created 2024-08-24 Sat 17:21
% Intended LaTeX compiler: pdflatex
\documentclass[11pt]{report}
\usepackage[utf8]{inputenc}
\usepackage[T1]{fontenc}
\usepackage{graphicx}
\usepackage{longtable}
\usepackage{wrapfig}
\usepackage{rotating}
\usepackage[normalem]{ulem}
\usepackage{amsmath}
\usepackage{amssymb}
\usepackage{capt-of}
\usepackage{hyperref}
\input{preamble}
\author{Adithya Nair}
\date{\today}
\title{Journal Club}
\hypersetup{
	pdfauthor={Adithya Nair},
	pdftitle={Journal Club},
	pdfsubject={},
	pdflang={English}}
\begin{document}

\message{ !name(real-analysis.tex) !offset(-3) }


\maketitle
\tableofcontents

\part{Real Analysis}
\label{sec:org654bad8}
TEXTBOOK: \href{file:///home/adithya/University-Latex-Notes/Journal Club/Analysis I - Tao.pdf}{Analysis 1 by Terence Tao}
\chapter{Natural Numbers}
\label{sec:org904821b}
Numbers were built to count. A system for counting was made, and that system is the number system.
\begin{definition}
	A natural number is an element of the set $\mathbb{N}$ of the set
	\[
		\mathbb{N} = \{0,1,2,3\cdots \}
	\]
	is obtained from 0 and counting forward indefinitely.
\end{definition}
\section{Peano Axioms}
\label{sec:orgcc5f6c9}
We start with axioms to help clarify this.
\begin{itemize}
	\item Axiom 1 : \(0 \in \mathbb{N}\)
	\item Axiom 2: If \(n \in \mathbb{N}\),, then \(n++ \in \mathbb{N}\)
	\item Axiom 3: 0 is not an increment of any other natural number \(n \in \mathbb{N}\)
	\item Axiom 4: If \(n \neq m\), \(n++ \neq m++\)
	\item Axiom 5: (Principle Of Mathematical Induction) Let \(P(n)\) be any property pertaining to a natural number \(n\). Suppose that \(P(0)\) is true, and suppose that whenever \(P(n)\) is true, \(P(n++)\) is also true. Then \(P(n)\) is true for every natural number.
\end{itemize}

We then make an assumption: That the set \(\mathbb{N}\) which satisfies these five axioms is called the set of natural numbers.
With these 5 axioms, we can construct sequences
\section{Recursive Definitions}
\label{sec:org1dae040}
\begin{prop}[Recursive Definitions]
	Suppose for each natural number $n$, we have some function $f_n:\mathbb{N} \rightarrow \mathbb{N}$ from the natural numbers to the natural numbers. Then we can assign a unique natural number $a_n$ to each natural number $n$, such that $a_0 = c$ and $a_{n++} = f_n(a_n)$ for each natural number $n$.
\end{prop}
\section{Addition}
\label{sec:orgabde90c}
\begin{definition}[Addition Of Natural Numbers]
	Let n be a natural number. $(n \in N)$. To add zero to m, we define $0+m:=m$ Now suppose inductively that we have defined how to add $n$ to $m$. Then we can add $n++$ to $m$ by defining($n++$) + m := (n+m)++
\end{definition}

\begin{lemma}
	For any natural number $n + 0=n$
\end{lemma}
\begin{proof}
	We use induction,

	The base case, n = 0,
	\begin{align*}
		n         & = 0, 0 + 0 = 0    \\
		n+0       & = n               \\
		(n++) + 0 & = (n+0)++ = (n++)
	\end{align*}

	Suppose inductively, that $n+0=n$,

	For $n=n++$,
	\begin{align*}
		(n++) + 0 & = (n+0)++       \\
		\text{We know that $n+0=n$} \\
		(n++) + 0 & = (n++)
	\end{align*}
\end{proof}

\begin{lemma}
	For any natural numbers $n$ and $m$,
	$$n + (m++) = (n+m)++$$
\end{lemma}
\begin{proof}
	Inducting on $n$ while keeping $m$ fixed,
	\begin{align*}
		n         & = 0,      \\
		0 + (m++) & = (0+m)++ \\
		0 + (m++) & = (m++)
	\end{align*}
	This we know is true from the definition of addition $(0+m:=m)$

	Suppose inductively, that $n+(m++) = (n+m)++$ is true.
	For $n=(n++)$,
	\begin{align*}
		(n++) + (m++) & = ((n++)+m)++ & \text{From the definition of addition} \\
		              & =(n+(m++))++                                           \\
		              & =((n+m)++))++
	\end{align*}
\end{proof}

Putting m = 0, we get \(n+1 = n++\)

\begin{prop}[Addition is commutative]
	For any natural numbers $n$ and $m$, $n+m=m+n$
\end{prop}
\begin{proof}
	We induct over $n$,
	For the base case, $n=0$,

	We must show that $m+0 = 0+m$
	From the definition of addition, we have
	$$0+m = m$$

	As shown earlier, we have

	$$m+0 = m$$

	This is clearly true for $n=0$.

	Now suppose inductively that $m+n = n+m$

	For $n=n++$, we must show that $m+(n++) = (n++) + m$

	We know from the definition of addition that,

	$$(n++) + m := (m+n)++$$

	And we proved earlier that,

	$$m+(n++) = (m+n)++$$

	Therefore,

	$$m+(n++) = (n++)+m$$
\end{proof}
\begin{prop}[Addition is associative]
	For any natural numbers, $a,b$ and $c$, we have $(a+b)+c = a+(b+c)$
\end{prop}
\begin{proof}
	We take $(a+b)+n = a + (b+n)$

	Inducting over n,

	For $n=0$,

	We have in the LHS,
	\begin{align*}
		 & =(a+b)+0 & \text{Since $n+0 = n$} \\
		 & =a+b
	\end{align*}

	On the RHS,
	\begin{align*}
		 & =a + (b+0) & \text{Since $n+0 = n$} \\
		 & =a + b
	\end{align*}

	Suppose inductively that $(a+b)+n = a+(b+n)$,

	For $n=n++$,
	We have to show that $(a+b)+(n++) = a+(b+(n++))$

	On the LHS we have,

	\begin{align*}
		 & =(a+b)+(n++)                                               \\
		 & =(a+b+n)++   & \text{(From the lemma $m+(n++) = (m+n)++$)} \\
	\end{align*}

	On the RHS we have,

	\begin{align*}
		 & =a+(b+(n++))                                               \\
		 & =a+(b+n)++   & \text{(From the lemma $m+(n++) = (m+n)++$)} \\
		 & =(a+b+n)++
	\end{align*}

	LHS = RHS
\end{proof}

\begin{prop}[Cancellation Law]
	Let $a,b,c$ be natural numbers such that $a+b=a+c$. Then we have $b=c$.
\end{prop}
\begin{proof}
	We have,
	$$n+b=n+c$$

	Inducting over n,
	For the base case, $n=0$
	\begin{align*}
		0 + b & = 0 + c \\
		b     & = c
	\end{align*}

	Suppose inductively that $n+b=n+c$
	For $n=n++$,
	$$(n++)+b=(n++)+c$$
	On the LHS
	\begin{align*}
		 & =(n++) + b \\
		 & =(n+b)++
	\end{align*}

	On the RHS
	\begin{align*}
		 & =(n++) + c \\
		 & =(n+c)++
	\end{align*}

	We know from the inductive hypothesis that,
	$$\text{If} n+b = n+c, \text{then} b = c$$

	Thus we have,
	$$b++ = c++$$
\end{proof}

\begin{definition}[Positive natural number]
	All numbers where,
	\[
		n \neq 0, n \in \mathbb{N}
	\]
\end{definition}
\begin{prop}
	If $a$ is a positive natural number and $b$ is a natural number, then $a+b$ is positive.
\end{prop}
\begin{proof}
	Inducting over b,

	For $b$ = 0,
	\begin{align*}
		a+0 = a
	\end{align*}
	This proves the base case, since we know a is positive.

	Now, suppose inductively, that $(a+b)$ is positive.

	For $(a+(n++))$,
	\begin{align*}
		a+(n++) = (a+n)++
	\end{align*}
	We know from Axiom 3 that $n++ \neq 0$. Thus we close the inductive loop.
\end{proof}
\begin{lemma}
	For every $a$, there exists a unique $b$ such that $b++ = a$
\end{lemma}
\begin{proof}
	Proof by contradiction,
	Suppose that there are two different increments, $m++$, $n++$ that equal to $a$,

	We have,
	\begin{align*}
		m++ & = a \\
		n++ & = a
	\end{align*}

	Then we can say,
	\begin{align*}
		m++   & = n++                                \\
		m + 1 & = n+1                                \\
		m     & = n   & \text{(By Cancellation Law)}
	\end{align*}

	But we said that m and n are different numbers which increment to $a$.

	Therefore, we can conclude that there is only one number $b$ which increments to $a$
\end{proof}
\section{Order}
\label{sec:org9ec10b6}
\begin{definition}[Order]
	Let n and m be natural numbers we say that $n$ is greater than or equal to m, and write $n \geq m$ iff we have $n = m + a$ for some natural number $a$. We say that $n > m$ when $n \geq m$ and $n \neq m$
\end{definition}
\begin{prop}[Basic properties of order for natural numbers]
	Let $a,b,c$ be natural numbers then
	\begin{enumerate}
		\item (Order is reflexive) $a \geq a$
		\item (Order is transitive) If $a \geq b$ and $b \geq c$, then $a \geq c$
		\item (Order is antisymmetric) If $a \geq b$ and $b \geq a$ then $a=b$
		\item (Addition preserves order) $a \geq b$ if and only if $a+c \geq b+c$
		\item $a<b$ if and only if $a++ \leq b$
		\item $a<b$ if and only if $b= a+d$ for some positive number d.
	\end{enumerate}
\end{prop}
\begin{proof}
	\begin{enumerate}
		\item Proving order is reflexive, $a \geq a$

		      We know that,

		      $a = a + 0$

		      From the definition of order,
		      We can write that $a \geq b$ when $a = b + d$ where $d \in \mathbb{N}$

		      Thus $a \geq a$.

		\item Proving order is transitive, $a \geq b$ and $b \geq c$ then $a \geq c$

		      We write,

		      \begin{align*}
			      a & = b + d     \\
			      b & = c + e     \\
			      a & = c + e + d
		      \end{align*}
		      We can say that since $(e+d) \in \mathbb{N}$

		      We define $f := (e+d)$
		      Where $f \in \mathbb{N}$
		      \begin{align*}
			      a & = c + (f)
		      \end{align*}

		      Thus we can say,
		      $$\text{If } a \geq b, b \geq c \text{ then } a \geq c$$

		\item Proving order is antisymmetric, If $a \geq b$ and $b \geq a$ then $a=b$
		      We can say,
		      \begin{align*}
			      a = b + d \\
			      b = a + e \\
		      \end{align*}
		      Where $d,e \in \mathbb{N}$

		      \begin{align*}
			      a = (a + e) + d \\
			      b = (b + d) + e \\
		      \end{align*}

		      Then we can write,
		      \begin{align*}
			      a = a + (e + d) \\
			      b = b + (d + e) \\
		      \end{align*}

		      Then we can say that $(e+d)$ and $(d+e)$ are 0.

		      We know that if $a + b = 0$ then $a,b = 0$

		      Thus $d$ and $e$ are 0.
		      \begin{align*}
			      a = b + d \\
			      a = b
		      \end{align*}
		\item Proving $a < b$ if and only if $b =a+d$ for some positive number d
		      If $b = a+d$ where $d$ is a positive natural number, $d \neq 0$

		      Which means that $b \neq a + 0$ or $b \neq a$

		      This means that b is strictly greater than a

		      If $a<b$ then $a \geq b$ and $a \neq b$

		      So if $a \geq b$
		      Then,
		      \begin{align*}
			      a = b + d \\
		      \end{align*}
		      But,
		      \begin{align*}
			      a \neq b     \\
			      a \neq b + 0 \\
			      d \neq 0
		      \end{align*}
		      Thus d cannot be 0. $d$ can only be a positive natural number.
		\item Proving addition preserves order, $a \geq b$ if and only if $a + c \geq b + c$
		      Proving $a \geq b$ if $a + c \geq b + C$

		      Where $d \in \mathbb{N}$
		      \begin{align*}
			      a + c & = b + c + d &   & \text{By definition}       \\
			      a + c & = (b+d) + c &                                \\
			      a     & = (b+d)     &   & \text{By cancellation law} \\
			      a     & \geq b
		      \end{align*}
		      Proving $a + c \geq b +c$ if $a \geq b$

		      We know,
		      \begin{align*}
			      a = b + d \\
		      \end{align*}
		      Where $d \in \mathbb{N}$

		      We write a+c using what we know from above,
		      \begin{align*}
			      a + c   & = b + d + c   \\
			      a + c   & = b + c + d   \\
			      (a + c) & = (b + c) + d \\
			      a + c   & \geq b + c
		      \end{align*}

		\item Proving $a < b$ if and only if $a++ \leq b$
		      Proving $a < b$ if $a++ \leq b$

		      We can write,
		      \begin{align*}
			      a++       & = b + d & \text{Where $d \in \mathbb{N}$} \\
			      a++ + d   & = b                                       \\
			      a + (d++) & = b                                       \\
		      \end{align*}
		      Since from Axiom 3, we know that 0 is not an increment of any natural number, $(d++ \neq 0)$
		      Therefore,
		      \begin{align*}
			      a & < b
		      \end{align*}

	\end{enumerate}
\end{proof}
\begin{prop}[Trichotomy of order for natural numbers]
	Let $a$ and $b$ be natural numbers. Then exactly one of the following statements is true: $a<b, a=b or a>b$
\end{prop}
\begin{proof}
	First we show that no more than one of the statements is true.
	If $a<b$ then $a \neq b$ by definition. If $a>b$ then $a \neq b$ by definition. If $a>b$ and $a<b$ then $a=b$, which we proved earlier.

	Now to show that exactly one of these statements are true.
	We induct on a,

	When a = 0,
	We know that,
	\begin{align*}
		 & b & = 0 + b & (\forall b \in \mathbb{N}) \\
		 & b & \geq 0
	\end{align*}

	Suppose inductively that exactly one of the above statements are true for a and b.
	For a++,
	We take each statement. First for $a>b$
	\begin{align*}
		a     & > b                                                                  \\
		a     & = b + d                                                              \\
		(a++) & = (b + d)++                                                          \\
		(a++) & = b + d++                                                            \\
		(a++) & > b         & \text{If $d \in \mathbb{N}$ then $d++ \in \mathbb{N}$}
	\end{align*}
	For $a=b$
	\begin{align*}
		a     & = b     \\
		(a++) & = (b)++ \\
		(a++) & = b + 1 \\
		a     & > b     \\
	\end{align*}
	For $a<b$
	\begin{align*}
		a & <b            \\
		a + d = b         \\
		(a + d)++ = b++   \\
		(a++) + d = b++   \\
		(a++) + d = b + 1 \\
	\end{align*}
	We have two cases,
	If $d = 1$,
	Then by cancellation law
	$$ a++ = b $$
	If $d \neq 1$
	Then
	$$a++ < b$$
	But never both, which concludes the inductive loop.
\end{proof}
\section{Special Forms Of Induction}
\label{sec:orgd06d9c6}
\begin{prop}[Strong Principle Of Induction]
	Let $m_0$ be a natural number, and let $P(m)$ be a property pertaining to an arbitrary natural number $m$. Suppose that for each $m \geq m_0$, we have the following implication: if $P(m')$ is true for all natural numbers $m_0 \leq m' < m$, then $P(m)$ is also true.(In particular this means that $P(m_0$ is true, since in this case the hypothesis is vacuous.) Then we can conclude that $P(m)$ is true for all natural numbers $m \geq m_0$.
\end{prop}
\begin{proof}
	For a property $Q(n)$, which is the property that $P(m')$ is true for $m_0 \leq m < n$, then $P(n)$ is true... Then it is true $\forall m \geq m_0$

	For $Q(0)$, we can say that the statement is vacuous since the conditions are not satisfied for both when $m_{0} = 0$ and when $m_{0} <0$

	Suppose inductively that $Q(n)$ is true.
	Which means that

	Then for Q(n++)
\end{proof}
\begin{prop}[Backward Induction]
	Let $n$ be a natural number, and let $P(m)$ be a property pertaining to the natural numbers such that whenever $P(m++)$ is true, then $P(m)$. Suppose that $P(n)$ is also true. Prove that $P(m)$ is true for all natural numbers $m \leq n$.
\end{prop}
\begin{prop}[Induction starting from the base case $n$]
	Let n be a natural number, and let $P(m)$ be a property pertaining to the natural numbers such that whenever P(m) is true, P(m++) is true. Show that if P(n) is true, then P(m) is true for all m ≥ n. (This principle is sometimes referred to as the principle of induction starting from the base case n.)
\end{prop}
\begin{proof}
	Take a property $P(n)$, $m \geq n$

	Inducting over $n$,
\end{proof}
\section{Multiplication}
\label{sec:org572658a}
\begin{definition}[Multiplication]
	Let $m$ be a natural number. To multiply zero to $m$, we define $0 \times m := 0$. Now suppose inductively that we have defined how to multiply $n$ to $m$. Then we can multiply $n++$ to $m$ by defining $(n++) \times m := (n \times m) + m$
\end{definition}
We can say \(0 \times m = 0\), \(1 \times m = 0 + m\), \(2 \times m= 0 + m + m\) and so on.
\begin{lemma}
	Prove that multiplication is commutative
\end{lemma}
\begin{proof}
	We use the way we proved that addition is commutative as a blueprint.
	There are two things we need to prove first.
	\begin{enumerate}
		\item For any natural number, $n$, $n \times 0 = 0$
		\item For any natural numbers, $n$ and $m$, $n \times (m++) = (n \times m) + m$
	\end{enumerate}

	First we prove,
	For any natural number, $n$, $n \times 0 = n$
	We induct over $n$,
	For $n = 0$,
	$$0 \times 0 = 0$$

	Which is true from the definition

	Now suppose inductively, that $n \times 0 = 0$,
	For $(n++) \times 0$,
	From the definition we can write this as,
	\begin{align*}
		(n++) \times 0 & = (n \times 0) + 0 \\
		\text{We know that $n \times 0 = 0$}
		(n++) \times 0 & = 0 + 0            \\
		(n++) \times 0 & = 0
	\end{align*}
	Therefore, $$n \times 0 = n$$

	Now we prove,
	For any natural numbers, $n$ and $m$, $n \times (m++) = (n \times m) + m$
	We induct over $n$, (keeping $m$ fixed)

	For $n = 0$,
	We know from the definition for multiplication with zero that,
	\begin{align*}
		0 \times (m++) = 0                                    \\
		\text{We also know that}                              \\
		(m++) \times 0                  & = (m \times 0 ) + 0 \\
		(m++) \times 0                  & = 0                 \\
		(m++) \times 0 = 0 \times (m++) & = (0 \times m) + m
	\end{align*}

	Suppose inductively that $n \times (m++) = (n \times m) + m$
	For $n = (n++)$
	To prove $(n++) \times (m++) = ((n++) \times m) + m$,

	\begin{align*}
		(n++) \times (m++) & = (n \times (m++)) + m++            \\
		\text{We can rewrite RHS using the inductive hypothesis} \\
		(n++) \times (m++) & = ((n \times m) + m) + m++          \\
	\end{align*}
\end{proof}

For simplicity we now write \(a \times b\) as \(ab\)
\chapter{Set Theory}
\label{sec:org7b352c4}
We define first what a set is:

\begin{definition}[Sets]
	We define set A to be any unordered collection of objects. If $x$ is an object, we say that x is an element of A or $x \in A$ if x lies in the collection. Otherwise $x \in A$
\end{definition}

We start with some axioms:
\begin{enumerate}
	\item (Sets are objects) If $A$ is a set, then $A$ is also an object.
	      A side track about "Pure Set Theory" - This theory states that everything in the mathematical universe is a set. We can write 0 as {} or an empty set, 1 can be written as {0} and 2 as {0,1} and so on. Terence Tao argues that they are the 'cardinalities of the set.'
	\item (Equality of sets) Two sets A and B are equal, A = B, iff every element of A is an element of B. A = B, if and only if every element of $x$ of A also belongs to B, and every element $y$ of B belongs to A.
	\item (Empty set) There exists a set $\emptyset$ known as the empty set, which contains no elements. $x \notin \emptyset$
	      \begin{prop}[Partial Order]
		      If A \subseteq B, B \subseteq C \Rightarrow A \subseteq C
	      \end{prop}
	      \begin{proof}
		      If $x \in A$, then $x \in B$,
		      If $x \in B$, then $x \in C$,
		      Then $x \in A$, then $x \in C$

		      Thus, $A \subseteq C$
	      \end{proof}
	      \begin{lemma}[Single choice]
		      Let $A$ be a non-empty set. Then there exists an object $x$ such that $x \in A$
	      \end{lemma}
	      \begin{proof}
		      Proving by contradiction,
		      Suppose there is no object $x$ that belongs to A. For all $x$, we have $x \notin A$. We know from Axiom 3, that $x \notin \emptyset$

		      For the statement,
		      $$ x \in A \leftrightarrow x \in \emptyset $$

		      Is false both ways, which gives us the result true, which is a contradiction.

		      Thus we also prove that $\emptyset$ is unique.
	      \end{proof}
	\item (Pairwise Union) $A \cup B = \{x : x \in A \text{or} x \in B\}$
	      \begin{lemma}
		      $A \cup (B \cup C) = (A \cup B) \cup C$
	      \end{lemma}
	      \begin{proof}
		      Taking the left hand side,
		      We have $x \in A$ or $x \in (B \cup C)$. If we look to the right hand side, we have $x \in (A \cup B)$ or $ x \in C$
		      If we break the statement down further.
		      We have $x \in A$ or $x \in B$ or $x \in C$, and  on the right $x \in A$ or $x \in B$ or $x\in C$

		      The two statements are equivalent.
	      \end{proof}
	\item (Axiom Of Specification) A, $x \in A$, let P(x) be a property pertaining to $x$. Then there exists a set called $\{x \in A, P(x) \text{is true}\}$ whose elements are precisely the elements $x$ in A for which $P(x)$ is true.
	\item (Replacement) Let A be a set, for any object $x \in A$, and any object $y$, suppose we have a statement $P(x,y)$ pertaining to $x$ and $y$, such that for each $x \in A$ there is at most one $y$ for which $P(x,y)$ is true. Then there exists a set $\{y : P(x,y)$ is true for some $x \in A\}$
	\item (Infinity) There exists a set $\mathbb{N}$, whose elements are called natural numbers, as well as an object $0$ in $\mathbb{N}$, and an object $n++$ assigned to every natural number $n \in \mathbb{N}$ such that the Peano axioms hold.

	\item \textbf{Russel's Paradox} (Axiom Of Universal Specification) Suppose for every $x$ we have a property $P(x)$ pertaining to $x$, Then there exists a set $\{x : P(x)$ is true$\}$ such that for every object $y$:

	      $$ y \in \{ x : P(x) is true \} \Leftrightarrow P(y) \text{is true.}$$

	      There is an issue, let's say we have a set, where the property of the objects is that they themselves are sets.

	      Let's say we look at one of these sets,

	      \begin{align*}
		      \Omega & = \{ x: P(x) is true\} \\
		             & = \{x : x \notin x \}
	      \end{align*}

	\item (Regularity) If A is a  non-empty set, then there is at least one element $x$ of A which is either not a set or disjoint from A.
\end{enumerate}
\begin{definition}[Intersection Of Sets]
	$$A \cap B = \{x: x \in A \text{ and } x \in B \}$$
\end{definition}
\end{document}

\message{ !name(real-analysis.tex) !offset(-634) }
