\documentclass{report}

\input{preamble}
\title{\Huge{23MAT112}\\ Class Notes}
\author{\huge{Adithya Nair}}
\date{}

\begin{document}

\maketitle
\newpage% or \cleardoublepage
% \pdfbookmark[<level>]{<title>}{<dest>}
\tableofcontents

\pagebreak
\chapter{Sets, Vector Spaces And Subspaces}
\section{Sets}
\begin{definition}[Sets]
   A set is a collection of objects. e.g. - $\mathbb{N}, \mathbb{Z}, \mathbb{Q}, \mathbb{R}, \mathbb{C}$
\end{definition}

Notations - 
\begin{itemize}
	 \item $A,B \dots Z$ will denote sets
	 \item $a,b \dots z$ will denote elements 
	 \item $a \in A$, a is an element of A
	 \item $a \notin A$, a is not an element of A
\end{itemize}

Roster Notation
\begin{itemize}
      \item $\mathbb{N} = {1,2,\dots}$
      \item $\mathbb{A} = {2,4,6,8}$ 
      \item $B = {x \in Z^+ | x < 10}$
\end{itemize}
\subsection{Subsets}
\begin{definition}[Subsets]
A and B are two sets. A is a subset of b, and we write $A \subset B$, if every element of A is also an element of B.	
\end{definition}
\begin{theorem}[Equivalence Of Sets]
   Two sets A and B are equal, and we write A = B, $\iff A \subset B$ and $B \subset A$
\end{theorem}
\subsection{Unions}
\begin{definition}[Unions]
The union of two sets A and B denoted by $A \cup B$, is 
\[
   A \cup B = {x \ | \ x \in A \ or \ x \in B}
\]
\end{definition}
\end{document}


