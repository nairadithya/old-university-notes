\documentclass{report}

\input{preamble}
\title{\Huge{23MAT102}\\ Class Notes}
\author{\huge{Adithya Nair}}
\date{}

\begin{document}

\maketitle
\newpage% or \cleardoublepage
% \pdfbookmark[<level>]{<title>}{<dest>}
\tableofcontents

\pagebreak
\section{A Basic Order Of Importance}
\begin{itemize}
    \item \textbf{Axiom} - Statements taken as fact
    \item \textbf{Theorem} - Statements that are proven using axioms
    \item \textbf{Lemma} - Statements proven using theorems
    \item \textbf{Proposition} - Statements, regardless of whether it is true or false, is assumed to be true
    \item \textbf{Corollary} - A theorem that is proven using another theorem.*
\end{itemize}
\chapter{A Revision Of Sets And Functions}
Sets are assumed to be sets on the basis of a theory known as \textbf{Naive Set Theory}. according to this theory, A set is defined as,

\begin{definition}[Sets]
A set is a collection of objects
\end{definition}

e.g. - \[\mathbb{N}, \mathbb{Z}, \mathbb{Q}, \mathbb{R}, \mathbb{C}\]

\section{Notations}
A,B...Z will denote sets\\
a,b...z will denote elements\\
a $\in$ A, a is an element of A\\
a $\notin$  A, a is not an element of A\\

\section{Roster Notation}
\[\mathbb{N} = \{1,2...\}\]
\[A = \{2,4,6,8\]
\[B = \{ x \in Z+| x < 10 \} \]
 B is written in set builder form

\section{Basic Concepts Of Sets}
\begin{definition}[Subsets]
A and B are two sets. A is a subset of B, and we write A $\subset$ B, if every element of A is also an element of B
\end{definition}

\begin{theorem}
    Two sets A and B are equal and we write A=B if and only if A $\subset$ B and B $\subset$ A
\end{theorem}

\begin{definition}[Unions]
    The union of two sets A and B, denoted by A $\cup$ B, is
    \[ A \cup B = \{x|\ x \in A \ and \ x \in B\}\]
\end{definition}
\begin{definition}[Intersections]
    The intersection of two sets denoted by A $\cap$ B, is
    \[A \cap B = \{x|\ x \in A \ or \ x \in B\}\]
\end{definition}
\begin{definition}[Set Difference]
    The difference of two sets denoted by A$\setminus$ B is
    \[A\setminus B = {x|\ x \in A \ and \ x \notin B}\]
\end{definition}

\begin{definition}[Set Complement]
    The complement of a set A, denoted by A $^{C}$ is,
    \[A^{C} = \{x \in X |\ x \notin A \}\]
\end{definition}
\begin{itemize}
    \item $(B \cup C)^{C} = B^{C}  \cap C^{C}$
    \item $(B \cap C)^{C} = B^{C} \cap C^{C}$
    \item $A\setminus (B\cup C) = (A\setminus B) \cap (A \setminus C)$
    \item $A\setminus (B\cap C) = (A\setminus B) \cup (A \setminus C)$
\end{itemize}
\section{Logical Notation}
$\forall$ - for all\\
$\exists$ - there exists\\
$\exists!$ - there exists a unique\\

\section{Functions}
f: A $\rightarrow$ B\\
f(a) = b, a $\in$ A, b $\in$ B\\
A is the \textbf{domain} of the function, B is the \textbf{codomain} of the function and, \\
\{b$\in$ B \textbar f(a) = b \} - Range\\
\section{Cartesian Product}
\[A \times B = \{(a,b) | a \in A, b \in B\}\]



\section{Composition Of Functions}
(g$\circ$f)(x) = (g(f(x))

A function is the same as a mapping, which is the same as a transformation 
\section{Types Of Functions}
\begin{enumerate}
    \item f is injective(one-one) if,
    \[f(a) = f(a')\ then\ a\ =\ a'\]
    \item f is surjective(onto) if, 
    \[\forall b \in B, \exists \ a \in A, \  f(a) = b\]
    \item f is bijective if f is injective and surjective
    \end{enumerate}   
\section*{Reference}
Knowles - Linear Vector Spaces and Cartesian Tensors\\
Halmos - Finite Dimensional Linear Spaces\\
Gelfand - Linear Algebra
\chapter{Vector Spaces}
A vector space over a field F = $\mathbb{R}$ or $\mathbb{C}$ is a set V with two operations:
\begin{enumerate}
\item +:V$\times$ V $\rightarrow$ V i.e. "+" is closed under addition.

\item $\cdot$:F $\times$ V $\rightarrow$ V, i.e. $"\cdot"$ is closed under multiplication
\end{enumerate}
having the following properties

\begin{enumerate}
    \item \textbf{Associativity}\\
    \[\forall \ v_1, v_2,v_3 \in V, (v_1 + v_2) + v_3 = v_1 + (v_2 + v_3)\]
    \item \textbf{Existence of identity element}\\
    \[\exists! \ 0 \in V, \forall v\ in V, such that 0 + v = v\]
    \item \textbf{Existence of additive inverse}
    \[\forall \ v \in V \exists (-v) \in V, v + (-v) = 0\]
    \item \textbf{Commutativity}
    \[\forall \ u, v \in V, u+v = v+u \]
        Properties 1 to 4 constitute a group known as the "abelian group" or "commutative group"
    \item \textbf{Existence of multiplactive identity}
    \[\exists! \ 1 \in V, \ such \ that \ \forall \ v \in V, 1\cdot v = v\]
    \item \textbf{Associativity}
    \[\mu, \lambda \in F, v \in V, \lambda(\mu \cdot v) = (\lambda\mu)\cdot v\]
    \item \textbf{Distribution of + over $\cdot$}
    \[(\lambda + \mu) \cdot v = \lambda \cdot v + \mu \cdot v, \forall \ \mu, \lambda \in F\]
    \item \textbf{Distribution of $\cdot$ over +}
    \[\lambda \cdot(u + v)\ = \lambda \cdot u + \lambda \cdot v, \forall \lambda \in \ F, u,v \in V\]    
\end{enumerate}
\section{Examples Of Vector Spaces}
\begin{enumerate}
    \item V = {0}
    \item $\mathbb{R}$
    \item All polynomials of order \textbf{at most} n
\end{enumerate}
\section*{Reference}
    \begin{itemize}
        \item Donald Knuth
        \item Marvin Mirsky, MIT
        \item Web Of Stories, Youtube Channel
        \item Axler, Chapter 1
        \item Olver, Shakiban, Chapter 2
        \item Terrence Tao Notes - AMS Open Math
    \end{itemize}
    \section{Some Theorems And Proofs Regarding Vector Spaces}
\begin{theorem}
    Additive identity is unique
\end{theorem}
    \begin{proof}
        Suppose $\exists \ additive \ identities \ 0_1$, $0_2$ such that
        \[\forall u \in V, 0_1 + u \ \& \ 0_2 + u = u\]
        \[0_1 + 0_2 = 0_2\]
        \[0_2 + 0_1 = 0_1\]
        \[\therefore 0_1 = 0_2\]
    \end{proof}
\begin{theorem}
    Additive inverse is unique
\end{theorem}
    \begin{proof}
        Suppose additive inverses of u are $v_1, v_2$
        \[u + v_1 = 0, u + v_2 = 0\]
        \[v_2 + (u+v_1) = v_2 + 0\]
        \[(v_2 + u) + v_1 = v_2\]
        \[0 + v_1 = v_2\]
        \[v_1 = v_2\]
        \end{proof}
\begin{theorem}
    $0\cdot u = 0$
\end{theorem}
\begin{proof}

   Let 0 . u = 0 
   Consider,
   \[
   v + v = 0.u + 0.u = (0+0).u
   \]
\[
	= 0.u = v
\]	
\[
	\implies v+v = v \\
\]
\[
       v+(v + (-v)) = v + -v \\
\]
\[
	\implies v = 0 
\]
\end{proof}
\begin{theorem}[Scalars And Inverses]
	\[
	   (-\lambda)u = -(\lambda.u) = \lambda . (-u)
	\]
\end{theorem}
\begin{proof}
	Let $v = (-\lambda).u$
	Consider,
	\begin{displaymath}
	v + \lambda. u = (-\lambda).u + \lambda u	
	\end{displaymath}
	
\end{proof}

\section{Fields}
A) To every pair $\alpha$ and $\beta$ of scalars, there corresponds a scalar $\alpha + \beta$ called the sum of $\alpha \text{ and } \beta$, in such a way that 
\begin{enumerate}
	\item addition is commutative, $\alpha + \beta = \beta + \alpha$
	\item addition is associative, $(\alpha + \beta)+\gamma = \alpha + (\beta + \gamma)$   
	\item There exists a unique scalar 0, called zero, such that $\alpha + 0 = \alpha$ for every scalar $\alpha$, and 
	\item to every scalar $\alpha$, there corresponds a unique scalar $(-\alpha)$ such that $\alpha + (-\alpha) = 0$ 
\end{enumerate}
\end{document}
