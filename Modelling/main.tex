\documentclass[twoside]{report}
\input{preamble}
\title{\Huge{23MAT112}\\ Class Notes}
\usepackage{svg}
\author{\huge{Adithya Nair}}
\date{}
\begin{document}
\chapter{Preamble}
This is a precursor to the real class.

An example of the use of modelling, in a somewhat humourous approach is through the use of a girl and boy's love story.

Let's take two functions $G(t)$ and $B(t)$. These functions output the `feelings' of the girl and the boy.

Then,
\begin{align*}
	\frac{dG(t)}{dt} = cG(t) + dB(t) \\
	\frac{dB(t)}{dt} = aB(t) \pm bG(t) \\
\end{align*}
These constants represent some influence on the change in the feelings due to the current feelings you and your significant other can have.

Here's another example:

Model chocolate consumption and the way it leads to happiness.
State the variables:
\begin{align*}
	C(t) - \text{Chocolates consumed at a given day}\\
	H(t) - \text{Happiness on a given day}
\end{align*}
The model for these would work like this.
\begin{align*}
	\frac{dC(t)}{dt} = \alpha C(t) + \beta H(t) \\
	\frac{dH(t)}{dt} = \delta C(t) + \gamma H(t) \\
\end{align*}
The amount of chocolate depends on our happiness as well as the chocolates already consumed.

Here's a more serious problem.

Modelling the COVID pandemic.

We can use three main variables:

\begin{align*}
	A(t) &- \text{Affected cases} \\
	R(t) &- \text{Recovered cases} \\
	U(t) &- \text{Non-Affected cases}\\
\end{align*}
\begin{align*}
	\text{Total no. of people} &= A(t) + R(t) + U(t) \\
\end{align*}


\end{document}

