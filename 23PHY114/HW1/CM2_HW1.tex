\documentclass{article}
\title{Computational Mechanics 2 \\ Homework 1 Solutions}
\author{Adithya Nair \\ AID23002}
\usepackage{ragged2e}
\usepackage{minted}
\input{preamble.tex}
\begin{document}
\maketitle
 For all the followin three trusses (Figures 1 to 3), first draw the free body diagram of the structure nd write the equations for LMB and the AMB about point A. Then write a finite element method code in Octave to calculate the reaction forces at the supports. Finally, plot the deformed shape of the truss on top of the original shape. 
\pagebreak
The Procedure For Question 1:
\begin{enumerate}
    \item Input all given values, keep track of order of evaluations using the nodeVector. Loop through the structure using the localStiffnessGenerator function as well as the elementCount.
    \item Apply boundary conditions, and find displacementVector by $K^{-1}f$
    \item Add the displacementVector to the original positionVector, plot over both of them.
\end{enumerate}
\section{Figure 1} 
\begin{minted}{octave}
% Input
E = 11.4e6;
areaVector = 50e-4*ones(5,1);
lengthVector = [0.5;1.0;sqrt(5)/2;1/sqrt(2);0.5];
angleVector = [0;0;pi- atan(1/2);pi/4;pi/2];
positionVector = [-0.5;0;0;0;1;0;0;0.5]; % Found by taking B as origin

nodeVector = [1;2;2;3;3;4;1;4;2;4]; % AB, BC, CD, DA, DB
elementCount = 5;
nodeCount = 4;

displacementVector = zeros(2*nodeCount,1);
forceVector = zeros(2*nodeCount,1);
forceVector(6) = -1500;
forceVector(7) = -1000;


stiffnessGlobal = zeros(2*nodeCount,2*nodeCount);

function stiffnessLocal = localStiffnessGenerator(E,A,l,theta);
    stiffnessConstant = E*A/l;
    R = [cos(theta) -sin(theta); sin(theta) cos(theta)];
    stiffnessMatrix = [stiffnessConstant 0 -stiffnessConstant 0; zeros(1,4);-stiffnessConstant 0 stiffnessConstant 0; zeros(1,4)];
    R4 = [R zeros(2,2); zeros(2,2) R];
    stiffnessLocal = R4*stiffnessMatrix*R4';
end
function stiffnessLocalGlobal = local2Global(stiffnessLocal,node1,node2,nodeCount)
    stiffnessLocalGlobal = zeros(2*nodeCount,2*nodeCount);
    i = 2*node1 - 1;
    j = 2*node2 - 1;
    stiffnessLocalGlobal(i:(i+1),i:(i+1)) = stiffnessLocal(1:2,1:2);
    stiffnessLocalGlobal(i:(i+1),j:(j+1)) = stiffnessLocal(1:2,3:4);
    stiffnessLocalGlobal(j:(j+1),i:(i+1)) = stiffnessLocal(3:4,1:2);
    stiffnessLocalGlobal(j:(j+1),j:(j+1)) = stiffnessLocal(3:4,3:4);
end
function forceAxial = tension(stiffnessLocal,theta,displacementSelected)
    R = [cos(theta) -sin(theta); sin(theta) cos(theta)];
    R4 = [R zeros(2,2); zeros(2,2) R];
    displacementLocal = R4'*displacementSelected;
    forceAxial = stiffnessLocal*displacementLocal;
end
% Looping through the entire structure.
A = 0; theta = 0; l = 0; stiffnessLocal = zeros(4,4); nodeAxialForces = zeros(nodeCount,1);
for element = 1:5 % For first three bars
    A = areaVector(element);
    theta = angleVector(element);
    l = lengthVector(element);
    stiffnessLocal = localStiffnessGenerator(E,A,l,theta);
    nodeCounter = element*2 - 1;
    stiffnessLocalGlobal = local2Global(stiffnessLocal,nodeVector(nodeCounter),nodeVector(nodeCounter+1),nodeCount);
    stiffnessLocal
    stiffnessLocalGlobal
    stiffnessGlobal += stiffnessLocalGlobal;
end

% Applying Boundary Conditions,
forceEval = forceVector([3 5:end]);
displacementEval = displacementVector([3 5:end]);
stiffnessEval = stiffnessGlobal([3 5:end],[3 5:end]);
displacementEval = stiffnessEval\forceEval;

displacementVector([3 5:end]) = displacementEval;
forceVector = stiffnessGlobal*displacementVector;

forceAxial = zeros(4,1);
% 2 Finding Axial Forces
for element = 1:elementCount
    A = areaVector(element);
    theta = angleVector(element);
    l = lengthVector(element);
    nodeCounter = element*2 - 1;
    stiffnessLocal = localStiffnessGenerator(E,A,l,theta);
    displacementVectorSelected = zeros(4,1);
    displacementVectorSelected(1:2) = displacementVector(nodeVector(nodeCounter):(nodeVector(nodeCounter))+1);
    displacementVectorSelected(3:4) = displacementVector(nodeVector(nodeCounter+1):(nodeVector(nodeCounter+1))+1);
    forceAxial = tension(stiffnessLocal,theta,displacementVectorSelected);
    nodeAxialForces(nodeVector(nodeCounter)) += forceAxial(1);
    nodeAxialForces(nodeVector(nodeCounter+1)) += forceAxial(3);
end

% Calculating Stresses And Strains
stressVector = nodeAxialForces/A;
strainVector = displacementVector/l;
\end{minted}

The calculated displacement vectors are:
\[
\begin{bmatrix}
0 \\
    0 \\
-0.0263 \\
    0 \\
-0.0789 \\
-0.4964 \\
0.0803 \\
-0.0307   \\
\end{bmatrix}
\]
The resulting forces are - 
\[
    \begin{bmatrix}
  1.0000e+03 \\
  -2.0000e+03 \\
  -9.0949e-13 \\
   3.5000e+03 \\
   4.5475e-13 \\
  -1.5000e+03 \\
  -1.0000e+03 \\
  -3.6676e-14    \\
    \end{bmatrix}
\]
\begin{figure}[h]
        \includegraphics[width=0.95\textwidth]{figures/plotq1.pdf}
\end{figure}
\section{Figure 2} % (fold)
\label{sec:figure_}
\begin{minted}{octave}
     % Input
E = 200e6;
elementCount = 10;
nodeCount = 7;

areaVector = 8e-3*ones(elementCount,1);
% AB, BC, CD, DE, EG, GH, AG, BG, BE, CE
% 1 m down from C, so E is (6,3)
AB = 3;
BC = 3;
CD = 2;
DE = CD*asecd(26.7);
DH = sqrt(8^2 + 4^2);
GH = sqrt(3^2 + 1.5^2);
EG = DH - DE - GH;
AG = 2.5*acscd(39.8);
BG = 2.5;
BE = 1*acscd(18.4);
CE = 1;
lengthVector = [AB;BC;CD;DE;EG;GH;AG;BG;BE;CE];
angleVector = [0;0;0;180+26.7;180+26.7;180+26.7;360-39.8;90;360-18.4;90];
positionVector = [0;4;3;4;6;4;8;4;6;3;3;1.5;0;0] % ABCDEGH

nodeVector = [1;2;2;3;3;4;4;5;5;6;6;7;1;6;6;2;2;5;5;3]; % AB, BC, CD, DA, DB
displacementVector = zeros(2*nodeCount,1);
forceVector = zeros(2*nodeCount,1);
forceVector(7) = 12e3; % F2
forceVector(9) = +1.5e3; % F1
% DE, EG,

stiffnessGlobal = zeros(2*nodeCount,2*nodeCount);

function stiffnessLocal = localStiffnessGenerator(E,A,l,theta);
    stiffnessConstant = E*A/l;
    R = [cosd(theta) -sind(theta); sind(theta) cosd(theta)];
    stiffnessMatrix = [stiffnessConstant 0 -stiffnessConstant 0; zeros(1,4);-stiffnessConstant 0 stiffnessConstant 0; zeros(1,4)];
    R4 = [R zeros(2,2); zeros(2,2) R];
    stiffnessLocal = R4*stiffnessMatrix*R4';
end
function stiffnessLocalGlobal = local2Global(stiffnessLocal,node1,node2,nodeCount)
    stiffnessLocalGlobal = zeros(2*nodeCount,2*nodeCount);
    i = 2*node1 - 1;
    j = 2*node2 - 1;
    stiffnessLocalGlobal(i:(i+1),i:(i+1)) = stiffnessLocal(1:2,1:2);
    stiffnessLocalGlobal(i:(i+1),j:(j+1)) = stiffnessLocal(1:2,3:4);
    stiffnessLocalGlobal(j:(j+1),i:(i+1)) = stiffnessLocal(3:4,1:2);
    stiffnessLocalGlobal(j:(j+1),j:(j+1)) = stiffnessLocal(3:4,3:4);
end
function forceAxial = tension(stiffnessLocal,theta,displacementSelected)
    R = [cos(theta) -sin(theta); sin(theta) cos(theta)];
    R4 = [R zeros(2,2); zeros(2,2) R];
    displacementLocal = R4'*displacementSelected;
    forceAxial = stiffnessLocal*displacementLocal;
end
% Looping through the entire structure.
A = 0; theta = 0; l = 0; stiffnessLocal = zeros(4,4);
for element = 1:elementCount % For first three bars
    A = areaVector(element);
    theta = angleVector(element);
    l = lengthVector(element);
    stiffnessLocal = localStiffnessGenerator(E,A,l,theta);
    nodeCounter = element*2 - 1;
    stiffnessLocalGlobal = local2Global(stiffnessLocal,nodeVector(nodeCounter),nodeVector(nodeCounter+1),nodeCount);
    stiffnessGlobal += stiffnessLocalGlobal;
end

% Applying Boundary Conditions,
range = [3:nodeCount*2-2]
forceEval = forceVector(range);
displacementEval = displacementVector(range);
stiffnessEval = stiffnessGlobal(range,range);
displacementEval = stiffnessEval\forceEval;

displacementVector(range) = displacementEval;
forceVector = stiffnessGlobal*displacementVector;

positionVectorNew = positionVector + displacementVector;
plot(positionVector(1:2:end),positionVector(2:2:end),'--or');
hold on;
plot(positionVectorNew(1:2:end),positionVectorNew(2:2:end),'--ob');

forceAxial = zeros(4,1);
% 2 Finding Axial Forces
nodeAxialForces =  zeros(nodeCount,1);
for element = 1:elementCount
    A = areaVector(element);
    theta = angleVector(element);
    l = lengthVector(element);
    nodeCounter = element*2 - 1;
    stiffnessLocal = localStiffnessGenerator(E,A,l,theta);
    displacementVectorSelected = zeros(4,1);
    displacementVectorSelected(1:2) = displacementVector(nodeVector(nodeCounter):(nodeVector(nodeCounter))+1);
    displacementVectorSelected(3:4) = displacementVector(nodeVector(nodeCounter+1):(nodeVector(nodeCounter+1))+1);
    forceAxial = tension(stiffnessLocal,theta,displacementVectorSelected);
    nodeAxialForces(nodeVector(nodeCounter)) += forceAxial(1);
    nodeAxialForces(nodeVector(nodeCounter+1)) += forceAxial(3);
end
% Calculating Stresses And Strains
stressVector = nodeAxialForces/A;
strainVector = displacementVector/l;
\end{minted}
The calculated displacement vectors are - 
\[
    \begin{bmatrix}
                0 \\
        0 \\
   0.0242 \\
  -0.0004 \\
   0.0467 \\
  -0.1255 \\
   0.0617 \\
  -0.2789 \\
  -0.0155 \\
  -0.1255 \\
   0.0009 \\
   0.0001 \\
        0 \\
        0 \\
    \end{bmatrix}
\]
The calculated force vectors are - 
\[
    \begin{bmatrix}
         -1.3128e+04 \\
   1.8728e+02 \\
   3.6380e-12 \\
            0 \\
  -1.0914e-11 \\
            0 \\
   1.2000e+04 \\
  -1.1369e-13 \\
   1.5000e+03 \\
  -2.9104e-11 \\
   8.6622e-15 \\
  -9.9476e-13 \\
  -3.7237e+02 \\
  -1.8728e+02 \\
    \end{bmatrix}
\]
% section Figure 2 (end)
\begin{figure}
        \includegraphics[width=0.95\textwidth]{figures/plotq1fig2.pdf}
\end{figure}
\section{Figure 3} % (fold)
\begin{minted}{octave}
     % Input
E = 200e7;
elementCount = 9;
nodeCount = 6;

areaVector = 2.5e-3*ones(elementCount,1);
% Following from the node order
% 12 23 34 45 56 13 35 24 46
lengthVector = [4;3;5;3;5;5;4;4;4];
angleVector = [0;90;360-atan(3/4);90;360-atan(3/4);45;0;0;0];
positionVector = [0;0;4;0;4;3;8;0;8;3;12;0] % ABCDEGH

nodeVector = [1;2;2;3;3;4;4;5;5;6;1;3;3;5;2;4;4;6]; % AB, BC, CD, DA, DB
displacementVector = zeros(2*nodeCount,1);
forceVector = zeros(2*nodeCount,1);
forceVector(8) = 400; % F2
forceVector(7) = -1200; % F1

% DE, EG,

stiffnessGlobal = zeros(2*nodeCount,2*nodeCount);

function stiffnessLocal = localStiffnessGenerator(E,A,l,theta);
    stiffnessConstant = E*A/l;
    R = [cosd(theta) -sind(theta); sind(theta) cosd(theta)];
    stiffnessMatrix = [stiffnessConstant 0 -stiffnessConstant 0; zeros(1,4);-stiffnessConstant 0 stiffnessConstant 0; zeros(1,4)];
    R4 = [R zeros(2,2); zeros(2,2) R];
    stiffnessLocal = R4*stiffnessMatrix*R4';
end
function stiffnessLocalGlobal = local2Global(stiffnessLocal,node1,node2,nodeCount)
    stiffnessLocalGlobal = zeros(2*nodeCount,2*nodeCount);
    i = 2*node1 - 1;
    j = 2*node2 - 1;
    stiffnessLocalGlobal(i:(i+1),i:(i+1)) = stiffnessLocal(1:2,1:2);
    stiffnessLocalGlobal(i:(i+1),j:(j+1)) = stiffnessLocal(1:2,3:4);
    stiffnessLocalGlobal(j:(j+1),i:(i+1)) = stiffnessLocal(3:4,1:2);
    stiffnessLocalGlobal(j:(j+1),j:(j+1)) = stiffnessLocal(3:4,3:4);
end

function forceAxial = tension(stiffnessLocal,theta,displacementSelected)
    R = [cos(theta) -sin(theta); sin(theta) cos(theta)];
    R4 = [R zeros(2,2); zeros(2,2) R];
    displacementLocal = R4'*displacementSelected;
    forceAxial = stiffnessLocal*displacementLocal;
end

% Looping through the entire structure.
A = 0; theta = 0; l = 0; stiffnessLocal = zeros(4,4);
for element = 1:elementCount % For first three bars
    A = areaVector(element);
    theta = angleVector(element);
    l = lengthVector(element);
    stiffnessLocal = localStiffnessGenerator(E,A,l,theta);
    nodeCounter = element*2 - 1;
    stiffnessLocalGlobal = local2Global(stiffnessLocal,nodeVector(nodeCounter),nodeVector(nodeCounter+1),nodeCount);
    stiffnessGlobal += stiffnessLocalGlobal;
end

% Applying Boundary Conditions,
range = [3:nodeCount*2-2]
forceEval = forceVector(range);
displacementEval = displacementVector(range);
stiffnessEval = stiffnessGlobal(range,range);
displacementEval = stiffnessEval\forceEval;


displacementVector(range) = displacementEval;
forceVector = stiffnessGlobal*displacementVector;


positionVectorNew = positionVector + displacementVector;
plot(positionVector(1:2:end),positionVector(2:2:end),'--or');
hold on;
plot(positionVectorNew(1:2:end),positionVectorNew(2:2:end),'--ob');
forceAxial = zeros(4,1);
nodeAxialForces = zeros(nodeCount,1);
% 2 Finding Axial Forces
for element = 1:elementCount
    A = areaVector(element);
    theta = angleVector(element);
    l = lengthVector(element);
    nodeCounter = element*2 - 1;
    stiffnessLocal = localStiffnessGenerator(E,A,l,theta);
    displacementVectorSelected = zeros(4,1);
    displacementVectorSelected(1:2) = displacementVector(nodeVector(nodeCounter):(nodeVector(nodeCounter))+1);
    displacementVectorSelected(3:4) = displacementVector(nodeVector(nodeCounter+1):(nodeVector(nodeCounter+1))+1);
    forceAxial = tension(stiffnessLocal,theta,displacementVectorSelected);
    nodeAxialForces(nodeVector(nodeCounter)) += forceAxial(1);
    nodeAxialForces(nodeVector(nodeCounter+1)) += forceAxial(3);
end

% Calculating Stresses And Strains
stressVector = nodeAxialForces/A;
strainVector = displacementVector/l;

% When maximum stress is 300 MPa, Forces need a new area,
stressVector = 300e6;
areaVector = stressVector./forceVector;
displacementVector = zeros(2*nodeCount,1);
forceVector = zeros(2*nodeCount,1);
forceVector(8) = 400; % F2
forceVector(7) = -1200; % F1

% DE, EG,

stiffnessGlobal = zeros(2*nodeCount,2*nodeCount);

function stiffnessLocal = localStiffnessGenerator(E,A,l,theta);
    stiffnessConstant = E*A/l;
    R = [cosd(theta) -sind(theta); sind(theta) cosd(theta)];
    stiffnessMatrix = [stiffnessConstant 0 -stiffnessConstant 0; zeros(1,4);-stiffnessConstant 0 stiffnessConstant 0; zeros(1,4)];
    R4 = [R zeros(2,2); zeros(2,2) R];
    stiffnessLocal = R4*stiffnessMatrix*R4';
end
function stiffnessLocalGlobal = local2Global(stiffnessLocal,node1,node2,nodeCount)
    stiffnessLocalGlobal = zeros(2*nodeCount,2*nodeCount);
    i = 2*node1 - 1;
    j = 2*node2 - 1;
    stiffnessLocalGlobal(i:(i+1),i:(i+1)) = stiffnessLocal(1:2,1:2);
    stiffnessLocalGlobal(i:(i+1),j:(j+1)) = stiffnessLocal(1:2,3:4);
    stiffnessLocalGlobal(j:(j+1),i:(i+1)) = stiffnessLocal(3:4,1:2);
    stiffnessLocalGlobal(j:(j+1),j:(j+1)) = stiffnessLocal(3:4,3:4);
end

function forceAxial = tension(stiffnessLocal,theta,displacementSelected)
    R = [cos(theta) -sin(theta); sin(theta) cos(theta)];
    R4 = [R zeros(2,2); zeros(2,2) R];
    displacementLocal = R4'*displacementSelected;
    forceAxial = stiffnessLocal*displacementLocal;
end

% Looping through the entire structure.
A = 0; theta = 0; l = 0; stiffnessLocal = zeros(4,4);
for element = 1:elementCount % For first three bars
    A = areaVector(element);
    theta = angleVector(element);
    l = lengthVector(element);
    stiffnessLocal = localStiffnessGenerator(E,A,l,theta);
    nodeCounter = element*2 - 1;
    stiffnessLocalGlobal = local2Global(stiffnessLocal,nodeVector(nodeCounter),nodeVector(nodeCounter+1),nodeCount);
    stiffnessGlobal += stiffnessLocalGlobal;
end

% Applying Boundary Conditions,
range = [3:nodeCount*2-2]
forceEval = forceVector(range);
displacementEval = displacementVector(range);
stiffnessEval = stiffnessGlobal(range,range);
displacementEval = stiffnessEval\forceEval;


displacementVector(range) = displacementEval;
forceVector = stiffnessGlobal*displacementVector;


positionVectorNew = positionVector + displacementVector;
plot(positionVector(1:2:end),positionVector(2:2:end),'--or');
hold on;
plot(positionVectorNew(1:2:end),positionVectorNew(2:2:end),'--ob');
forceAxial = zeros(4,1);
nodeAxialForces = zeros(nodeCount,1);
% 2 Finding Axial Forces
for element = 1:elementCount
    A = areaVector(element);
    theta = angleVector(element);
    l = lengthVector(element);
    nodeCounter = element*2 - 1;
    stiffnessLocal = localStiffnessGenerator(E,A,l,theta);
    displacementVectorSelected = zeros(4,1);
    displacementVectorSelected(1:2) = displacementVector(nodeVector(nodeCounter):(nodeVector(nodeCounter))+1);
    displacementVectorSelected(3:4) = displacementVector(nodeVector(nodeCounter+1):(nodeVector(nodeCounter+1))+1);
    forceAxial = tension(stiffnessLocal,theta,displacementVectorSelected);
    nodeAxialForces(nodeVector(nodeCounter)) += forceAxial(1);
    nodeAxialForces(nodeVector(nodeCounter+1)) += forceAxial(3);
end

% Calculating Stresses And Strains
stressVector = nodeAxialForces/A;
strainVector = displacementVector/l;

\end{minted}
% section Figure 3 (end)
The calculated force vectors are : - 

\begin{figure}
    \begin{center}
        \includegraphics[width=0.95\textwidth]{figures/plotq1fig3.pdf}
    \end{center}
\end{figure}

\end{document}
