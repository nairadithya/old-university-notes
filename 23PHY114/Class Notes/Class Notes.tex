\documentclass{report}
\input{preamble}
\title{\Huge{23PHY114}\\ Class Notes}
\author{\huge{Adithya Nair}}
\date{}
\begin{document}

\maketitle
\newpage% or \cleardoublepage
% \pdfbookmark[<level>]{<title>}{<dest>}
\tableofcontents
\chapter{Solids}
\createintro
\section{Moment Of Area}
The moment of inertia is used to help find the "resistance" to the force, given a specific axis or direction.

\section{Resisting Force And Moments From Supports} % (fold)
\label{sec:resisting_force_and_moments_from_supports}
There are three main kinds of supports - 
\begin{enumerate}
  \item Pin/Hinge - Fixes linear motion but leaves rotation free. 
  \item Roller - Fixes rotation but leaves motion free. 
  \item Clamp - Fixes both linear and rotational motion.
\end{enumerate}
% section Resisting Force And Moments From Supports (end)
Take this case, with a bunch of forces being applied to the given load. 

There are three main things we need to find for this figure. 
\begin{enumerate}
  \item The resultant force acting on this bar fixed to a hinge. 
  \item The support reaction force and moment.
  \item The moment on the object (maximum)
\end{enumerate}
The way to approach the problem is as always,
\begin{enumerate}
  \item Free Body Diagram first. 
  \item Assuming $\Sigma f = 0 $, because the object has no acceleration currently, because it is fixed.
  \item Assuming $\Sigma M = 0$
  \item The point at which the resultant force is applied is found by, 
    \[
      \frac{\int |r|dm}{\int dm}
    \]
\end{enumerate}
\section{Derivation Of The Uniaxial Formula}

\section{Code For Uniaxial Deformation} % (fold)


\label{sec:for_uniaxial_deformation}
Main Subroutines For Uniaxial Deformation
% section For Uniaxial Deformation (end)
\subsection{Finding The Local Stiffness Matrix} % (fold)
\begin{lstlisting}{octave}
function stiffnessLocal = localStiffnessGenerator(E,A,l,theta);
    stiffnessConstant = E*A/l;
    R = [cos(theta) -sin(theta); sin(theta) cos(theta)];
    stiffnessMatrix = [stiffnessConstant 0 -stiffnessConstant 0; zeros(1,4);-stiffnessConstant 0 stiffnessConstant 0; zeros(1,4)];
    R4 = [R zeros(2,2); zeros(2,2) R];
    stiffnessLocal = R4*stiffnessMatrix*R4';
end
\end{lstlisting}
\label{sec:finding_the_local_stiffness_matrix}
% subsubsection Finding The Local Stiffness Matrix (end)
\subsection{Converting The Local Stiffness To A Global Stiffness Matrix}
\begin{lstlisting}{octave}
function stiffnessLocalGlobal = local2Global(stiffnessLocal,node1,node2,nodeCount)
    stiffnessLocalGlobal = zeros(2*nodeCount,2*nodeCount);
    i = 2*node1 - 1;
    j = 2*node2 - 1;
    stiffnessLocalGlobal(i:(i+1),i:(i+1)) = stiffnessLocal(1:2,1:2);
    stiffnessLocalGlobal(i:(i+1),j:(j+1)) = stiffnessLocal(1:2,3:4);
    stiffnessLocalGlobal(j:(j+1),i:(i+1)) = stiffnessLocal(3:4,1:2);
    stiffnessLocalGlobal(j:(j+1),j:(j+1)) = stiffnessLocal(3:4,3:4);
end
\end{lstlisting}
\subsection{Main Loop Through Evaluating The Global Stiffness Matrix}
\begin{lstlisting}{octave}
 A = 0; theta = 0; l = 0; stiffnessLocal = zeros(4,4); nodeAxialForces = zeros(nodeCount,1);
for element = 1:elementCount % For first three bars
    A = areaVector(element);
    theta = angleVector(element);
    l = lengthVector(element);
    stiffnessLocal = localStiffnessGenerator(E,A,l,theta);
    nodeCounter = element*2 - 1;
    stiffnessLocalGlobal = local2Global(stiffnessLocal,nodeVector(nodeCounter),nodeVector(nodeCounter+1),nodeCount);
    stiffnessLocal
    stiffnessLocalGlobal
    stiffnessGlobal += stiffnessLocalGlobal;
end
\end{lstlisting}
\subsection{Applying Boundary Conditions}
\begin{lstlisting}{octave}
selectedVector = [3 5:end]
forceEval = forceVector(selectedVector);
displacementEval = displacementVector(selectedVector);
stiffnessEval = stiffnessGlobal(selectedVector);
displacementEval = stiffnessEval\forceEval;
\end{lstlisting}
\section{Deriving For Bending Deformation}
Taking this example of a bar fixed to the wall, and examining small components of the bar and examining the forces at work on the component, we get a Tension, Moment and Stress.

Taking things along the $\hat{i}$ direction, we get
$$
-T + T + \Delta T = 0
$$
 What does this tell us? Tension is constant. 

While $-S + S + \Delta S - w\Delta x = 0$

$$
\frac{dS}{dx} = \lim_{\Delta x \rightarrow 0} \frac{\Delta S}{\Delta x} = w(x)
$$
We get an expression for the effects of the distributed force on the shear force of an individual object.

Looking at the angular momentum balance,

$$
\sum M_{/P} = 0
$$
$$
-M + (M + \Delta M) \hat{k} + (-\Delta x \hat{i} \times(-T\hat{i} - S \hat{j}) + (\frac{-\Delta x}{2} \hat{i} \times (-w(x) \Delta x \hat{j}))
$$
$$
\implies \frac{\Delta M }{\Delta x} + S - w\frac{\Delta x}{2} = 0
$$
Taking the limit on the equations, we get,
$$
\frac{dM}{dx} + S = 0
$$

Thus we get,
$$
\frac{dT}{dx} = 0
$$
$$
\frac{dS}{dx} = w(x)
$$
$$
\frac{dM}{dx} = -S
$$
Now we get to the Finite Element Method for bending.
$$
\theta(x) = v'(x)
$$
We find by simplification,

$$
c_1 = v_l 
$$
$$
c_2 = \theta_l
$$
$$
c_3 = \frac{3}{l^2}(v_r-v_l) - \frac{1}{l} (2\theta_l + \theta_r)
$$
$$
c_4 = \frac{2}{l^3} (v_l-v_r) + \frac{\theta_l+\theta_r}{l^2}
$$

$$
v(x) = H_1(x) v_l + H_2(x)\theta_l (H_3(x)) v_r + (H_4(x)) \theta_r
$$
$$
H_1 = 2(\frac{x}{l}^3) - 3\frac{x}{l}^2 + 1
$$
$$
H_2 = x-2\frac{x^2}{l} + \frac{x^3}{l^2}
$$
$$
H_3 = 3(\frac{x}{l})^2 -2\frac{x}{l}^3
$$
$$
H_4 = \frac{x^3}{l^2} - \frac{x^2}{l}
$$



Since we know that,
$$
\int_0^l EIq''v''dx
$$
$$
v = \underline{H}(x)^T v
$$
$$
q = H(x)
$$

\section{Quick Reference Notes}
\subsection{Derivations}
The governing differential equation is 
\[
  EI v^{(4)} = w(x)
\]

Then we can form a linear relationship
\begin{align*}
  v(x) & - \text{deflection} \\
  v^{(2)}(x) = \theta(x) & - \text{slope} \\
  EIv^{(2)}(x) = M(x) & - \text{bending moment related to curvature} \\
  EIv^{(3)}(x) = \frac{dM}{dx} = V(x) & - \text{transverse shear} \\ 
  EIv^{(4)}(x) = \frac{dV}{dx} = w(x) & - \text{load}
\end{align*}

Where, v is the deflection or displacement of the beam. E is the Young's Modulus and I is 

\subsection{Deriving Stiffness Matrix}
Use Hermite's interpolation formula to derive cubic shape functions for the deflection of beams.

\section{Final Equation}
For a bar, with axial, bending and shear deformation... The final equation is, where R, S and M are Axial, Shear and Moment.

\[
  F = k\Delta x
\]

\[
  \begin{bmatrix}
    R_1 \\ 
    S_1 \\
    M_1\\
    R_2 \\ 
    S_2 \\
    M_2\\
  \end{bmatrix}
  = 
  \begin{bmatrix}
    C_1 & 0 & 0 & -C_1 & 0 & 0 \\ 
    0 & 12C_2 & 6C_2L & 0 & -12C_2 & 6C_2L \\
    0 & 6C_2L & 4C_2L^2 & 0 & -6C_2L & 2C_2L^2 \\
    -C_1 & 0 & 0 & C_1 & 0 & 0 \\ 
    0 & -12C_2 & -6C_2L & 0 & 12C_2 & -6C_2L \\
    0 & 6C_2L & 2C_2L^2 & 0 & -6C_2L & 2C_2L^2 \\
  \end{bmatrix}
  \begin{bmatrix}
    u_1 \\ 
    v_1 \\ 
    \theta_1 \\
    u_2 \\ 
    v_2 \\ 
    \theta_2 \\
  \end{bmatrix}
\]
Where, $C_1 = \frac{EA}{L}$ and $C_2 = \frac{EI}{L^3}$

Rotating the matrix by angle $\theta$

\[
  R = \begin{bmatrix}
    c & s & 0 & 0 & 0 & 0 \\ 
    -s & c & 0 & 0 & 0 & 0 \\ 
    0 & 0 & 1 & 0 & 0 & 0 \\
    0 & 0 & 0 & c & s & 0 \\ 
    0 & 0 & 0 & -s & c & 0 \\ 
    0 & 0 & 0 & 0 & 0 & 1 \\
  \end{bmatrix}
\]
\end{document}
