\documentclass{report}
\input{preamble}
\usepackage{fancyhdr}
\usepackage{minted}
\title{\Huge{23PHY114}\\ Class Notes}
\author{\huge{Adithya Nair}}
\date{}
\begin{document}

\maketitle
\newpage% or \cleardoublepage
% \pdfbookmark[<level>]{<title>}{<dest>}
\tableofcontents
\chapter{Solids}
\section{Moment Of Area}
The moment of inertia is used to help find the "resistance" to the force, given a specific axis or direction.

\section{Resisting Force And Moments From Supports} % (fold)
\label{sec:resisting_force_and_moments_from_supports}
There are three main kinds of supports - 
\begin{enumerate}
  \item Pin/Hinge - Fixes linear motion but leaves rotation free. 
  \item Roller - Fixes rotation but leaves motion free. 
  \item Clamp - Fixes both linear and rotational motion.
\end{enumerate}
% section Resisting Force And Moments From Supports (end)
Take this case, with a bunch of forces being applied to the given load. 
\begin{figure}[ht]
    \includegraphics[width=0.9\textwidth]{figures/ProblemExample1.pdf}
\end{figure}
There are three main things we need to find for this figure. 
\begin{enumerate}
  \item The resultant force acting on this bar fixed to a hinge. 
  \item The support reaction force and moment.
  \item The moment on the object (maximum)
\end{enumerate}
The way to approach the problem is as always,
\begin{enumerate}
  \item Free Body Diagram first. 
  \item Assuming $\Sigma f = 0 $, because the object has no acceleration currently, because it is fixed.
  \item Assuming $\Sigma M = 0$
  \item The point at which the resultant force is applied is found by, 
    \[
      \frac{\int |r|dm}{\int dm}
    \]
\end{enumerate}
\section{For Uniaxial Deformation} % (fold)

\label{sec:for_uniaxial_deformation}
Main Subroutines For Uniaxial Deformation
% section For Uniaxial Deformation (end)
\subsection{Finding The Local Stiffness Matrix} % (fold)
\begin{minted}{octave}
function stiffnessLocal = localStiffnessGenerator(E,A,l,theta);
    stiffnessConstant = E*A/l;
    R = [cos(theta) -sin(theta); sin(theta) cos(theta)];
    stiffnessMatrix = [stiffnessConstant 0 -stiffnessConstant 0; zeros(1,4);-stiffnessConstant 0 stiffnessConstant 0; zeros(1,4)];
    R4 = [R zeros(2,2); zeros(2,2) R];
    stiffnessLocal = R4*stiffnessMatrix*R4';
end
\end{minted}
\label{sec:finding_the_local_stiffness_matrix}
% subsubsection Finding The Local Stiffness Matrix (end)
\subsection{Converting The Local Stiffness To A Global Stiffness Matrix}
\begin{minted}{octave}
   function stiffnessLocalGlobal = local2Global(stiffnessLocal,node1,node2,nodeCount)
    stiffnessLocalGlobal = zeros(2*nodeCount,2*nodeCount);
    i = 2*node1 - 1;
    j = 2*node2 - 1;
    stiffnessLocalGlobal(i:(i+1),i:(i+1)) = stiffnessLocal(1:2,1:2);
    stiffnessLocalGlobal(i:(i+1),j:(j+1)) = stiffnessLocal(1:2,3:4);
    stiffnessLocalGlobal(j:(j+1),i:(i+1)) = stiffnessLocal(3:4,1:2);
    stiffnessLocalGlobal(j:(j+1),j:(j+1)) = stiffnessLocal(3:4,3:4);
end
\end{minted}
\subsection{Main Loop Through Evaluating The Global Stiffness Matrix}
\begin{minted}{octave}
   A = 0; theta = 0; l = 0; stiffnessLocal = zeros(4,4); nodeAxialForces = zeros(nodeCount,1);
for element = 1:5 % For first three bars
    A = areaVector(element);
    theta = angleVector(element);
    l = lengthVector(element);
    stiffnessLocal = localStiffnessGenerator(E,A,l,theta);
    nodeCounter = element*2 - 1;
    stiffnessLocalGlobal = local2Global(stiffnessLocal,nodeVector(nodeCounter),nodeVector(nodeCounter+1),nodeCount);
    stiffnessLocal
    stiffnessLocalGlobal
    stiffnessGlobal += stiffnessLocalGlobal;
end
\end{minted}
\subsection{Applying Boundary Conditions}
\begin{minted}{octave}
selectedVector = [3 5:end]
forceEval = forceVector(selectedVector);
displacementEval = displacementVector(selectedVector);
stiffnessEval = stiffnessGlobal(selectedVector);
displacementEval = stiffnessEval\forceEval;
\end{minted}
\end{document}
