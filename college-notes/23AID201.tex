% Created 2024-07-31 Wed 12:19
% Intended LaTeX compiler: pdflatex
\documentclass[11pt]{report}
\usepackage[utf8]{inputenc}
\usepackage[T1]{fontenc}
\usepackage{graphicx}
\usepackage{longtable}
\usepackage{wrapfig}
\usepackage{rotating}
\usepackage[normalem]{ulem}
\usepackage{amsmath}
\usepackage{amssymb}
\usepackage{capt-of}
\usepackage{hyperref}
\input{preamble}
\author{Adithya Nair}
\date{\today}
\title{Modelling, Simulation And Analysis.}
\hypersetup{
 pdfauthor={Adithya Nair},
 pdftitle={Modelling, Simulation And Analysis.},
 pdfkeywords={},
 pdfsubject={},
 pdfcreator={Emacs 29.4 (Org mode 9.8)}, 
 pdflang={English}}
\begin{document}

\maketitle
\tableofcontents

\part{Course Overview}
\label{sec:org99ef384}
\begin{enumerate}
\item Introduction to Modelling
\item Bond graph modelling
\item Basic System Models
\end{enumerate}

The textbooks to be followed is: ``System Dynamics - Modeling, Simulation And Control Systems'', ``Modern Control Engineering, Katsuhiko Ogata''

What is modelling? It's a mathematical representation of a system. These representations allow us to evaluate the results of that given system. This course is focused on equipping us with the working knowledge to develop such mathematical models for our systems.

Upon modelling the system, we can start controlling this system. With the equations we have made, we can now add our own inputs to those equations. This in a nutshell is what the course is about.
\end{document}
